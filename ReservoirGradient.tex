\documentclass[]{article}
\usepackage{lmodern}
\usepackage{amssymb,amsmath}
\usepackage{ifxetex,ifluatex}
\usepackage{fixltx2e} % provides \textsubscript
\ifnum 0\ifxetex 1\fi\ifluatex 1\fi=0 % if pdftex
  \usepackage[T1]{fontenc}
  \usepackage[utf8]{inputenc}
\else % if luatex or xelatex
  \ifxetex
    \usepackage{mathspec}
  \else
    \usepackage{fontspec}
  \fi
  \defaultfontfeatures{Ligatures=TeX,Scale=MatchLowercase}
\fi
% use upquote if available, for straight quotes in verbatim environments
\IfFileExists{upquote.sty}{\usepackage{upquote}}{}
% use microtype if available
\IfFileExists{microtype.sty}{%
\usepackage{microtype}
\UseMicrotypeSet[protrusion]{basicmath} % disable protrusion for tt fonts
}{}
\usepackage[margin=1in]{geometry}
\usepackage{hyperref}
\hypersetup{unicode=true,
            pdftitle={Dormancy and dispersal structure bacterial communities across ecosystem boundaries},
            pdfauthor={Nathan I. Wisnoski, Mario E. Muscarella, Megan L. Larsen, and Jay T. Lennon},
            pdfborder={0 0 0},
            breaklinks=true}
\urlstyle{same}  % don't use monospace font for urls
\usepackage{color}
\usepackage{fancyvrb}
\newcommand{\VerbBar}{|}
\newcommand{\VERB}{\Verb[commandchars=\\\{\}]}
\DefineVerbatimEnvironment{Highlighting}{Verbatim}{commandchars=\\\{\}}
% Add ',fontsize=\small' for more characters per line
\usepackage{framed}
\definecolor{shadecolor}{RGB}{248,248,248}
\newenvironment{Shaded}{\begin{snugshade}}{\end{snugshade}}
\newcommand{\KeywordTok}[1]{\textcolor[rgb]{0.13,0.29,0.53}{\textbf{#1}}}
\newcommand{\DataTypeTok}[1]{\textcolor[rgb]{0.13,0.29,0.53}{#1}}
\newcommand{\DecValTok}[1]{\textcolor[rgb]{0.00,0.00,0.81}{#1}}
\newcommand{\BaseNTok}[1]{\textcolor[rgb]{0.00,0.00,0.81}{#1}}
\newcommand{\FloatTok}[1]{\textcolor[rgb]{0.00,0.00,0.81}{#1}}
\newcommand{\ConstantTok}[1]{\textcolor[rgb]{0.00,0.00,0.00}{#1}}
\newcommand{\CharTok}[1]{\textcolor[rgb]{0.31,0.60,0.02}{#1}}
\newcommand{\SpecialCharTok}[1]{\textcolor[rgb]{0.00,0.00,0.00}{#1}}
\newcommand{\StringTok}[1]{\textcolor[rgb]{0.31,0.60,0.02}{#1}}
\newcommand{\VerbatimStringTok}[1]{\textcolor[rgb]{0.31,0.60,0.02}{#1}}
\newcommand{\SpecialStringTok}[1]{\textcolor[rgb]{0.31,0.60,0.02}{#1}}
\newcommand{\ImportTok}[1]{#1}
\newcommand{\CommentTok}[1]{\textcolor[rgb]{0.56,0.35,0.01}{\textit{#1}}}
\newcommand{\DocumentationTok}[1]{\textcolor[rgb]{0.56,0.35,0.01}{\textbf{\textit{#1}}}}
\newcommand{\AnnotationTok}[1]{\textcolor[rgb]{0.56,0.35,0.01}{\textbf{\textit{#1}}}}
\newcommand{\CommentVarTok}[1]{\textcolor[rgb]{0.56,0.35,0.01}{\textbf{\textit{#1}}}}
\newcommand{\OtherTok}[1]{\textcolor[rgb]{0.56,0.35,0.01}{#1}}
\newcommand{\FunctionTok}[1]{\textcolor[rgb]{0.00,0.00,0.00}{#1}}
\newcommand{\VariableTok}[1]{\textcolor[rgb]{0.00,0.00,0.00}{#1}}
\newcommand{\ControlFlowTok}[1]{\textcolor[rgb]{0.13,0.29,0.53}{\textbf{#1}}}
\newcommand{\OperatorTok}[1]{\textcolor[rgb]{0.81,0.36,0.00}{\textbf{#1}}}
\newcommand{\BuiltInTok}[1]{#1}
\newcommand{\ExtensionTok}[1]{#1}
\newcommand{\PreprocessorTok}[1]{\textcolor[rgb]{0.56,0.35,0.01}{\textit{#1}}}
\newcommand{\AttributeTok}[1]{\textcolor[rgb]{0.77,0.63,0.00}{#1}}
\newcommand{\RegionMarkerTok}[1]{#1}
\newcommand{\InformationTok}[1]{\textcolor[rgb]{0.56,0.35,0.01}{\textbf{\textit{#1}}}}
\newcommand{\WarningTok}[1]{\textcolor[rgb]{0.56,0.35,0.01}{\textbf{\textit{#1}}}}
\newcommand{\AlertTok}[1]{\textcolor[rgb]{0.94,0.16,0.16}{#1}}
\newcommand{\ErrorTok}[1]{\textcolor[rgb]{0.64,0.00,0.00}{\textbf{#1}}}
\newcommand{\NormalTok}[1]{#1}
\usepackage{longtable,booktabs}
\usepackage{graphicx,grffile}
\makeatletter
\def\maxwidth{\ifdim\Gin@nat@width>\linewidth\linewidth\else\Gin@nat@width\fi}
\def\maxheight{\ifdim\Gin@nat@height>\textheight\textheight\else\Gin@nat@height\fi}
\makeatother
% Scale images if necessary, so that they will not overflow the page
% margins by default, and it is still possible to overwrite the defaults
% using explicit options in \includegraphics[width, height, ...]{}
\setkeys{Gin}{width=\maxwidth,height=\maxheight,keepaspectratio}
\IfFileExists{parskip.sty}{%
\usepackage{parskip}
}{% else
\setlength{\parindent}{0pt}
\setlength{\parskip}{6pt plus 2pt minus 1pt}
}
\setlength{\emergencystretch}{3em}  % prevent overfull lines
\providecommand{\tightlist}{%
  \setlength{\itemsep}{0pt}\setlength{\parskip}{0pt}}
\setcounter{secnumdepth}{0}
% Redefines (sub)paragraphs to behave more like sections
\ifx\paragraph\undefined\else
\let\oldparagraph\paragraph
\renewcommand{\paragraph}[1]{\oldparagraph{#1}\mbox{}}
\fi
\ifx\subparagraph\undefined\else
\let\oldsubparagraph\subparagraph
\renewcommand{\subparagraph}[1]{\oldsubparagraph{#1}\mbox{}}
\fi

%%% Use protect on footnotes to avoid problems with footnotes in titles
\let\rmarkdownfootnote\footnote%
\def\footnote{\protect\rmarkdownfootnote}

%%% Change title format to be more compact
\usepackage{titling}

% Create subtitle command for use in maketitle
\newcommand{\subtitle}[1]{
  \posttitle{
    \begin{center}\large#1\end{center}
    }
}

\setlength{\droptitle}{-2em}

  \title{Dormancy and dispersal structure bacterial communities across ecosystem
boundaries}
    \pretitle{\vspace{\droptitle}\centering\huge}
  \posttitle{\par}
    \author{Nathan I. Wisnoski, Mario E. Muscarella, Megan L. Larsen, and Jay T.
Lennon}
    \preauthor{\centering\large\emph}
  \postauthor{\par}
      \predate{\centering\large\emph}
  \postdate{\par}
    \date{28 January, 2019}

\usepackage{array}
\usepackage{graphics}

\begin{document}
\maketitle

\section{Initial Setup}\label{initial-setup}

First, we'll load the packages we'll need for the analysis, as well as
some other functions.

\begin{Shaded}
\begin{Highlighting}[]
\CommentTok{# Import Required Packages}
\KeywordTok{library}\NormalTok{(}\StringTok{"png"}\NormalTok{)}
\KeywordTok{library}\NormalTok{(}\StringTok{"grid"}\NormalTok{)}
\KeywordTok{library}\NormalTok{(}\StringTok{"tidyverse"}\NormalTok{)   }
\KeywordTok{library}\NormalTok{(}\StringTok{"vegan"}\NormalTok{)}
\KeywordTok{library}\NormalTok{(}\StringTok{"xtable"}\NormalTok{)}
\KeywordTok{library}\NormalTok{(}\StringTok{"viridis"}\NormalTok{)}
\KeywordTok{library}\NormalTok{(}\StringTok{"cowplot"}\NormalTok{)}
\KeywordTok{library}\NormalTok{(}\StringTok{"adespatial"}\NormalTok{)}
\KeywordTok{library}\NormalTok{(}\StringTok{"ggrepel"}\NormalTok{)}
\KeywordTok{library}\NormalTok{(}\StringTok{"gganimate"}\NormalTok{)}
\KeywordTok{library}\NormalTok{(}\StringTok{"maps"}\NormalTok{)}
\KeywordTok{library}\NormalTok{(}\StringTok{"rgdal"}\NormalTok{)}
\KeywordTok{library}\NormalTok{(}\StringTok{"iNEXT"}\NormalTok{)}
\KeywordTok{library}\NormalTok{(}\StringTok{"officer"}\NormalTok{)}
\KeywordTok{library}\NormalTok{(}\StringTok{"flextable"}\NormalTok{) }\CommentTok{#must have gdtools installed also}
\KeywordTok{library}\NormalTok{(}\StringTok{"broom"}\NormalTok{)}
\KeywordTok{library}\NormalTok{(}\StringTok{"ggpmisc"}\NormalTok{)}
\KeywordTok{library}\NormalTok{(}\StringTok{"pander"}\NormalTok{)}

\KeywordTok{source}\NormalTok{(}\StringTok{"bin/mothur_tools.R"}\NormalTok{)}
\NormalTok{se <-}\StringTok{ }\ControlFlowTok{function}\NormalTok{(x, ...)\{}\KeywordTok{sd}\NormalTok{(x, }\DataTypeTok{na.rm =} \OtherTok{TRUE}\NormalTok{)}\OperatorTok{/}\KeywordTok{sqrt}\NormalTok{(}\KeywordTok{length}\NormalTok{(}\KeywordTok{na.omit}\NormalTok{(x)))\}}
\end{Highlighting}
\end{Shaded}

Next, we'll set the aesthetics of the figures we will produce.

\begin{Shaded}
\begin{Highlighting}[]
\NormalTok{my.cols <-}\StringTok{ }\NormalTok{RColorBrewer}\OperatorTok{::}\KeywordTok{brewer.pal}\NormalTok{(}\DataTypeTok{n =} \DecValTok{4}\NormalTok{, }\DataTypeTok{name =} \StringTok{"Greys"}\NormalTok{)[}\DecValTok{3}\OperatorTok{:}\DecValTok{4}\NormalTok{]}

\CommentTok{# Set theme for figures in the paper}
\KeywordTok{theme_set}\NormalTok{(}\KeywordTok{theme_classic}\NormalTok{() }\OperatorTok{+}\StringTok{ }
\StringTok{  }\KeywordTok{theme}\NormalTok{(}\DataTypeTok{axis.title =} \KeywordTok{element_text}\NormalTok{(}\DataTypeTok{size =} \DecValTok{16}\NormalTok{),}
        \DataTypeTok{axis.title.x =} \KeywordTok{element_text}\NormalTok{(}\DataTypeTok{margin =} \KeywordTok{margin}\NormalTok{(}\DataTypeTok{t =} \DecValTok{15}\NormalTok{, }\DataTypeTok{b =} \DecValTok{15}\NormalTok{)),}
        \DataTypeTok{axis.title.y =} \KeywordTok{element_text}\NormalTok{(}\DataTypeTok{margin =} \KeywordTok{margin}\NormalTok{(}\DataTypeTok{l =} \DecValTok{15}\NormalTok{, }\DataTypeTok{r =} \DecValTok{15}\NormalTok{)),}
        \DataTypeTok{axis.text =} \KeywordTok{element_text}\NormalTok{(}\DataTypeTok{size =} \DecValTok{14}\NormalTok{),}
        \DataTypeTok{axis.text.x =} \KeywordTok{element_text}\NormalTok{(}\DataTypeTok{margin =} \KeywordTok{margin}\NormalTok{(}\DataTypeTok{t =} \DecValTok{5}\NormalTok{)),}
        \DataTypeTok{axis.text.y =} \KeywordTok{element_text}\NormalTok{(}\DataTypeTok{margin =} \KeywordTok{margin}\NormalTok{(}\DataTypeTok{r =} \DecValTok{5}\NormalTok{)),}
        \CommentTok{#axis.line.x = element_line(size = 1),}
        \CommentTok{#axis.line.y = element_line(size = 1),}
        \DataTypeTok{axis.line.x =} \KeywordTok{element_blank}\NormalTok{(),}
        \DataTypeTok{axis.line.y =} \KeywordTok{element_blank}\NormalTok{(),}
        \DataTypeTok{axis.ticks.x =} \KeywordTok{element_line}\NormalTok{(}\DataTypeTok{size =} \DecValTok{1}\NormalTok{),}
        \DataTypeTok{axis.ticks.y =} \KeywordTok{element_line}\NormalTok{(}\DataTypeTok{size =} \DecValTok{1}\NormalTok{),}
        \DataTypeTok{axis.ticks.length =} \KeywordTok{unit}\NormalTok{(.}\DecValTok{1}\NormalTok{, }\StringTok{"in"}\NormalTok{),}
        \DataTypeTok{panel.border =} \KeywordTok{element_rect}\NormalTok{(}\DataTypeTok{color =} \StringTok{"black"}\NormalTok{, }\DataTypeTok{fill =} \OtherTok{NA}\NormalTok{, }\DataTypeTok{size =} \FloatTok{1.5}\NormalTok{),}
        \DataTypeTok{legend.title =} \KeywordTok{element_blank}\NormalTok{(),}
        \DataTypeTok{legend.text =} \KeywordTok{element_text}\NormalTok{(}\DataTypeTok{size =} \DecValTok{14}\NormalTok{),}
        \DataTypeTok{strip.text =} \KeywordTok{element_text}\NormalTok{(}\DataTypeTok{size =} \DecValTok{14}\NormalTok{),}
        \DataTypeTok{strip.background =} \KeywordTok{element_blank}\NormalTok{()}
\NormalTok{        ))}
\end{Highlighting}
\end{Shaded}

\subsection{Import Data}\label{import-data}

Here, we read in the processed sequence files from mothur (shared and
taxonomy) and a design of the sampling. We also load in the
environmental data. We then remove the mock community from the dataset
and ensure the the design and OTU table are aligned by row.

\begin{Shaded}
\begin{Highlighting}[]
\CommentTok{# Define Inputs}
\CommentTok{# Design = general design file for experiment}
\CommentTok{# shared = OTU table from mothur with sequence similarity clustering}
\CommentTok{# Taxonomy = Taxonomic information for each OTU}
\NormalTok{design <-}\StringTok{ "data/UL.design.txt"}
\NormalTok{shared <-}\StringTok{ "data/ul_resgrad.trim.contigs.good.unique.good.filter.unique.precluster.pick.pick.pick.opti_mcc.shared"}
\NormalTok{taxon  <-}\StringTok{ "data/ul_resgrad.trim.contigs.good.unique.good.filter.unique.precluster.pick.pick.pick.opti_mcc.0.03.cons.taxonomy"}

\CommentTok{# Import Design}
\NormalTok{design <-}\StringTok{ }\KeywordTok{read.delim}\NormalTok{(design, }\DataTypeTok{header=}\NormalTok{T, }\DataTypeTok{row.names=}\DecValTok{1}\NormalTok{)}

\CommentTok{# Import Shared Files}
\NormalTok{OTUs <-}\StringTok{ }\KeywordTok{read.otu}\NormalTok{(}\DataTypeTok{shared =}\NormalTok{ shared, }\DataTypeTok{cutoff =} \StringTok{"0.03"}\NormalTok{)    }\CommentTok{# 97% Similarity}

\CommentTok{# Import Taxonomy}
\NormalTok{OTU.tax <-}\StringTok{ }\KeywordTok{read.tax}\NormalTok{(}\DataTypeTok{taxonomy =}\NormalTok{ taxon, }\DataTypeTok{format =} \StringTok{"rdp"}\NormalTok{)}

\CommentTok{# Load environmental data}
\NormalTok{env.dat <-}\StringTok{ }\KeywordTok{read.csv}\NormalTok{(}\StringTok{"data/ResGrad_EnvDat.csv"}\NormalTok{, }\DataTypeTok{header =} \OtherTok{TRUE}\NormalTok{)}
\NormalTok{env.dat <-}\StringTok{ }\NormalTok{env.dat[}\OperatorTok{-}\DecValTok{16}\NormalTok{,]}

\CommentTok{# Subset to just the reservoir gradient sites}
\NormalTok{OTUs <-}\StringTok{ }\NormalTok{OTUs[}\KeywordTok{str_which}\NormalTok{(}\KeywordTok{rownames}\NormalTok{(OTUs), }\StringTok{"RG"}\NormalTok{),]}
\NormalTok{OTUs <-}\StringTok{ }\NormalTok{OTUs[}\OperatorTok{-}\KeywordTok{which}\NormalTok{(}\KeywordTok{rownames}\NormalTok{(OTUs) }\OperatorTok{==}\StringTok{ "RGMockComm"}\NormalTok{),]}

\CommentTok{# make sure OTU table matches up with design order}
\NormalTok{OTUs <-}\StringTok{ }\NormalTok{OTUs[}\KeywordTok{match}\NormalTok{(}\KeywordTok{rownames}\NormalTok{(design), }\KeywordTok{rownames}\NormalTok{(OTUs)),]}
\end{Highlighting}
\end{Shaded}

\subsection{Clean and transform OTU
table}\label{clean-and-transform-otu-table}

Here, we remove OTUs with low incidence across sites, we remove any
samples with low coverage, and we standardize the OTU table by
log-transforming the abundances and relativizing by site.

\begin{Shaded}
\begin{Highlighting}[]
\CommentTok{# Remove OTUs with less than two occurences across all sites}
\NormalTok{OTUs <-}\StringTok{ }\NormalTok{OTUs[, }\KeywordTok{which}\NormalTok{(}\KeywordTok{colSums}\NormalTok{(OTUs) }\OperatorTok{>=}\StringTok{ }\DecValTok{2}\NormalTok{)]}

\CommentTok{# Sequencing Coverage}
\NormalTok{coverage <-}\StringTok{ }\KeywordTok{rowSums}\NormalTok{(OTUs)}

\CommentTok{# Remove Low Coverage Samples (This code removes two sites: Site 5DNA, Site 6cDNA)}
\NormalTok{lows <-}\StringTok{ }\KeywordTok{which}\NormalTok{(coverage }\OperatorTok{<}\StringTok{ }\DecValTok{10000}\NormalTok{)}
\NormalTok{OTUs <-}\StringTok{ }\NormalTok{OTUs[}\OperatorTok{-}\KeywordTok{which}\NormalTok{(coverage }\OperatorTok{<}\StringTok{ }\DecValTok{10000}\NormalTok{), ]}
\NormalTok{design <-}\StringTok{ }\NormalTok{design[}\OperatorTok{-}\KeywordTok{which}\NormalTok{(coverage }\OperatorTok{<}\StringTok{ }\DecValTok{10000}\NormalTok{), ]}
\CommentTok{# Remove OTUs with less than two occurences across all sites}
\NormalTok{OTUs <-}\StringTok{ }\NormalTok{OTUs[, }\KeywordTok{which}\NormalTok{(}\KeywordTok{colSums}\NormalTok{(OTUs) }\OperatorTok{>=}\StringTok{ }\DecValTok{2}\NormalTok{)]}
\CommentTok{# OTUs <- rrarefy(OTUs, min(coverage))}

\CommentTok{# Make Relative Abundance Matrices}
\NormalTok{OTUsREL <-}\StringTok{ }\KeywordTok{decostand}\NormalTok{(OTUs, }\DataTypeTok{method =} \StringTok{"total"}\NormalTok{)}

\CommentTok{# Log Transform Relative Abundances}
\NormalTok{OTUsREL.log <-}\StringTok{ }\KeywordTok{decostand}\NormalTok{(OTUs, }\DataTypeTok{method =} \StringTok{"log"}\NormalTok{)}
\end{Highlighting}
\end{Shaded}

\section{Reservoir environmental
gradients}\label{reservoir-environmental-gradients}

Just to see if there are any strong underlying resource or nutrient
gradients in the reservoir, we'll plot them along the distance of the
reservoir.

\begin{Shaded}
\begin{Highlighting}[]
\NormalTok{facet.labs <-}\StringTok{ }\KeywordTok{c}\NormalTok{(}\StringTok{`}\DataTypeTok{chla}\StringTok{`}\NormalTok{ =}\StringTok{ "Chlorophyll-a"}\NormalTok{,}
                \StringTok{`}\DataTypeTok{color}\StringTok{`}\NormalTok{ =}\StringTok{ "Color"}\NormalTok{,}
                \StringTok{`}\DataTypeTok{DO}\StringTok{`}\NormalTok{ =}\StringTok{ "Dissolved Oxygen"}\NormalTok{,}
                \StringTok{`}\DataTypeTok{pH}\StringTok{`}\NormalTok{ =}\StringTok{ "pH"}\NormalTok{,}
                \StringTok{`}\DataTypeTok{TP}\StringTok{`}\NormalTok{ =}\StringTok{ "Total Phosphorus"}\NormalTok{)}

\NormalTok{env.dat }\OperatorTok\StringTok{ }\KeywordTok{select}\NormalTok{(dist.dam, DO, pH, TP, chla) }\OperatorTok\StringTok{ }
\StringTok{  }\KeywordTok{gather}\NormalTok{(variable, value, }\OperatorTok{-}\NormalTok{dist.dam) }\OperatorTok\StringTok{ }
\StringTok{  }\KeywordTok{ggplot}\NormalTok{(}\KeywordTok{aes}\NormalTok{(}\DataTypeTok{x =}\NormalTok{ dist.dam, }\DataTypeTok{y =}\NormalTok{ value)) }\OperatorTok{+}\StringTok{ }
\StringTok{  }\KeywordTok{geom_point}\NormalTok{() }\OperatorTok{+}\StringTok{ }
\StringTok{  }\KeywordTok{geom_smooth}\NormalTok{(}\DataTypeTok{method =} \StringTok{"lm"}\NormalTok{, }\DataTypeTok{color =} \StringTok{"black"}\NormalTok{) }\OperatorTok{+}\StringTok{ }
\StringTok{  }\KeywordTok{facet_grid}\NormalTok{(variable }\OperatorTok{~}\NormalTok{., }\DataTypeTok{scales =} \StringTok{"free"}\NormalTok{, }\DataTypeTok{switch =} \StringTok{"y"}\NormalTok{, }
             \DataTypeTok{labeller =} \KeywordTok{as_labeller}\NormalTok{(facet.labs)) }\OperatorTok{+}\StringTok{ }
\StringTok{  }\KeywordTok{theme}\NormalTok{(}\DataTypeTok{strip.background =} \KeywordTok{element_blank}\NormalTok{(), }
        \DataTypeTok{strip.text =} \KeywordTok{element_text}\NormalTok{(}\DataTypeTok{size =} \DecValTok{14}\NormalTok{),}
        \DataTypeTok{strip.placement =} \StringTok{"outside"}\NormalTok{) }\OperatorTok{+}\StringTok{ }
\StringTok{  }\KeywordTok{labs}\NormalTok{(}\DataTypeTok{x =} \StringTok{"Reservoir Transect (m)"}\NormalTok{,}
       \DataTypeTok{y =} \StringTok{""}\NormalTok{) }\OperatorTok{+}
\StringTok{  }\KeywordTok{scale_x_reverse}\NormalTok{() }\OperatorTok{+}
\StringTok{  }\KeywordTok{scale_y_continuous}\NormalTok{()}
\end{Highlighting}
\end{Shaded}

\begin{center}\includegraphics{ReservoirGradient_files/figure-latex/env_plot-1} \end{center}

So, there are some weak gradients, but nothing too prevailing.

\section{Analyze Diversity}\label{analyze-diversity}

Now, we will analyze the bacterial diversity in the reservoir and nearby
soils to figure out how well they support different mechanisms of
community assembly.

\subsection{\texorpdfstring{How does \(\alpha\)-diversity vary along the
reservoir?}{How does \textbackslash{}alpha-diversity vary along the reservoir?}}\label{how-does-alpha-diversity-vary-along-the-reservoir}

First, we use the method of rarefaction and extrapolation developed by
Chao et al. in the iNEXT package.

\begin{Shaded}
\begin{Highlighting}[]
\CommentTok{# Observed Richness}
\NormalTok{S.obs <-}\StringTok{ }\KeywordTok{rowSums}\NormalTok{((OTUs }\OperatorTok{>}\StringTok{ }\DecValTok{0}\NormalTok{) }\OperatorTok{*}\StringTok{ }\DecValTok{1}\NormalTok{)}

\CommentTok{# Simpson's Evenness}
\NormalTok{SimpE <-}\StringTok{ }\ControlFlowTok{function}\NormalTok{(}\DataTypeTok{x =} \StringTok{""}\NormalTok{)\{}
\NormalTok{  x <-}\StringTok{ }\KeywordTok{as.data.frame}\NormalTok{(x)}
\NormalTok{  D <-}\StringTok{ }\KeywordTok{diversity}\NormalTok{(x, }\StringTok{"inv"}\NormalTok{)}
\NormalTok{  S <-}\StringTok{ }\KeywordTok{sum}\NormalTok{((x }\OperatorTok{>}\StringTok{ }\DecValTok{0}\NormalTok{) }\OperatorTok{*}\StringTok{ }\DecValTok{1}\NormalTok{) }
\NormalTok{  E <-}\StringTok{ }\NormalTok{(D)}\OperatorTok{/}\NormalTok{S }
  \KeywordTok{return}\NormalTok{(E)}
\NormalTok{\}}
\NormalTok{simpsE <-}\StringTok{ }\KeywordTok{round}\NormalTok{(}\KeywordTok{apply}\NormalTok{(OTUs, }\DecValTok{1}\NormalTok{, SimpE), }\DecValTok{3}\NormalTok{)}
\NormalTok{shan <-}\StringTok{ }\KeywordTok{diversity}\NormalTok{(OTUs, }\DataTypeTok{index =} \StringTok{"shannon"}\NormalTok{)}
\NormalTok{exp.shan <-}\StringTok{ }\KeywordTok{exp}\NormalTok{(shan)}
\NormalTok{alpha.div <-}\StringTok{ }\KeywordTok{cbind}\NormalTok{(design, S.obs, simpsE, shan, exp.shan)}


\CommentTok{# # estimate asymptotic richness}
\CommentTok{# divestim <- iNEXT(t(OTUs), datatype = "abundance", nboot = 999)}
\CommentTok{# saveRDS(divestim, file = "intermediate-data/inext-output-999boots.rda")}
\NormalTok{divestim <-}\StringTok{ }\KeywordTok{read_rds}\NormalTok{(}\StringTok{"intermediate-data/inext-output-999boots.rda"}\NormalTok{)}
\NormalTok{divestim.df <-}\StringTok{ }\KeywordTok{fortify}\NormalTok{(divestim) }\OperatorTok\StringTok{ }
\StringTok{  }\KeywordTok{mutate}\NormalTok{(}\DataTypeTok{habitat =} \KeywordTok{str_to_title}\NormalTok{(design[}\KeywordTok{as.character}\NormalTok{(site),}\StringTok{"type"}\NormalTok{]))}
\end{Highlighting}
\end{Shaded}

Here is the resulting curve, showing the higher diversity in soil
samples relative to the lake samples.

\begin{Shaded}
\begin{Highlighting}[]
\NormalTok{divestim.df }\OperatorTok\StringTok{ }
\StringTok{  }\KeywordTok{ggplot}\NormalTok{(}\KeywordTok{aes}\NormalTok{(}\DataTypeTok{x =}\NormalTok{ x, }\DataTypeTok{y =}\NormalTok{ y, }
             \DataTypeTok{ymin =}\NormalTok{ y.lwr, }\DataTypeTok{ymax =}\NormalTok{ y.upr, }
             \DataTypeTok{color =}\NormalTok{ habitat, }\DataTypeTok{fill =}\NormalTok{ habitat, }\DataTypeTok{group =}\NormalTok{ site)) }\OperatorTok{+}
\StringTok{  }\KeywordTok{geom_ribbon}\NormalTok{(}\DataTypeTok{data=}\KeywordTok{subset}\NormalTok{(divestim.df, method }\OperatorTok{==}\StringTok{ "extrapolated"}\NormalTok{), }\DataTypeTok{alpha =} \FloatTok{0.3}\NormalTok{) }\OperatorTok{+}
\StringTok{  }\KeywordTok{geom_line}\NormalTok{(}\DataTypeTok{data=}\KeywordTok{subset}\NormalTok{(divestim.df, method }\OperatorTok{==}\StringTok{ "interpolated"}\NormalTok{), }\DataTypeTok{size =} \DecValTok{1}\NormalTok{, }\DataTypeTok{alpha =}\NormalTok{ .}\DecValTok{8}\NormalTok{) }\OperatorTok{+}
\StringTok{  }\KeywordTok{geom_line}\NormalTok{(}\DataTypeTok{alpha =} \DecValTok{1}\NormalTok{, }\DataTypeTok{linetype =} \StringTok{"dashed"}\NormalTok{) }\OperatorTok{+}
\StringTok{  }\KeywordTok{scale_x_continuous}\NormalTok{(}\DataTypeTok{labels =}\NormalTok{ scales}\OperatorTok{::}\NormalTok{comma) }\OperatorTok{+}
\StringTok{  }\KeywordTok{labs}\NormalTok{(}\DataTypeTok{x =} \StringTok{"Sample Size"}\NormalTok{, }\DataTypeTok{y =} \StringTok{"Estimated Richness"}\NormalTok{) }\OperatorTok{+}
\StringTok{  }\KeywordTok{theme}\NormalTok{(}\DataTypeTok{legend.position =}  \KeywordTok{c}\NormalTok{(.}\DecValTok{88}\NormalTok{,.}\DecValTok{5}\NormalTok{)) }\OperatorTok{+}
\StringTok{  }\KeywordTok{scale_color_grey}\NormalTok{(}\DataTypeTok{end =}\NormalTok{ .}\DecValTok{7}\NormalTok{) }\OperatorTok{+}
\StringTok{  }\KeywordTok{scale_fill_grey}\NormalTok{(}\DataTypeTok{end =}\NormalTok{ .}\DecValTok{7}\NormalTok{)}
\end{Highlighting}
\end{Shaded}

\begin{center}\includegraphics{ReservoirGradient_files/figure-latex/rare_extrap_plot-1} \end{center}

Next, we'll extract the estimates for the Hill numbers at different
levels of q, which differentially weight common versus rare species.

\begin{Shaded}
\begin{Highlighting}[]
\NormalTok{hill.estim <-}\StringTok{ }\NormalTok{divestim}\OperatorTok{$}\NormalTok{AsyEst }\OperatorTok\StringTok{ }\KeywordTok{filter}\NormalTok{(Diversity }\OperatorTok{==}\StringTok{ "Species richness"}\NormalTok{) }\OperatorTok\StringTok{ }
\StringTok{  }\KeywordTok{left_join}\NormalTok{(}\KeywordTok{rownames_to_column}\NormalTok{(alpha.div), }\DataTypeTok{by =} \KeywordTok{c}\NormalTok{(}\StringTok{"Observed"}\NormalTok{ =}\StringTok{ "S.obs"}\NormalTok{)) }\OperatorTok\StringTok{ }
\StringTok{  }\KeywordTok{select}\NormalTok{(Site, rowname, station, molecule, type, distance) }\OperatorTok\StringTok{ }
\StringTok{  }\KeywordTok{left_join}\NormalTok{(divestim}\OperatorTok{$}\NormalTok{AsyEst, }\DataTypeTok{by =} \StringTok{"Site"}\NormalTok{)}

\NormalTok{hill.water <-}\StringTok{ }\KeywordTok{as_tibble}\NormalTok{(hill.estim) }\OperatorTok\StringTok{ }\KeywordTok{filter}\NormalTok{(type }\OperatorTok{==}\StringTok{ "water"}\NormalTok{)}
\NormalTok{hill.water.rich <-}\StringTok{ }\KeywordTok{subset}\NormalTok{(hill.water, Diversity }\OperatorTok{==}\StringTok{ "Species richness"}\NormalTok{)}
\NormalTok{hill.water.shan <-}\StringTok{ }\KeywordTok{subset}\NormalTok{(hill.water, Diversity }\OperatorTok{==}\StringTok{ "Shannon diversity"}\NormalTok{)}
\NormalTok{hill.water.simp <-}\StringTok{ }\KeywordTok{subset}\NormalTok{(hill.water, Diversity }\OperatorTok{==}\StringTok{ "Simpson diversity"}\NormalTok{)}

\NormalTok{hill.water.mod.rich <-}\StringTok{ }\KeywordTok{lm}\NormalTok{(Estimator }\OperatorTok{~}\StringTok{ }\NormalTok{distance }\OperatorTok{*}\StringTok{ }\NormalTok{molecule, }\DataTypeTok{data =}\NormalTok{ hill.water.rich)}
\NormalTok{hill.water.mod.shan <-}\StringTok{ }\KeywordTok{lm}\NormalTok{(Estimator }\OperatorTok{~}\StringTok{ }\NormalTok{distance }\OperatorTok{*}\StringTok{ }\NormalTok{molecule, }\DataTypeTok{data =}\NormalTok{ hill.water.shan)}
\NormalTok{hill.water.mod.simp <-}\StringTok{ }\KeywordTok{lm}\NormalTok{(Estimator }\OperatorTok{~}\StringTok{ }\NormalTok{distance }\OperatorTok{*}\StringTok{ }\NormalTok{molecule, }\DataTypeTok{data =}\NormalTok{ hill.water.simp)}

\CommentTok{# summary(hill.water.mod.rich)}
\CommentTok{# summary(hill.water.mod.shan)}
\CommentTok{# summary(hill.water.mod.simp)}

\CommentTok{# tidy up the model output}
\NormalTok{hill.water.mods <-}\StringTok{ }\KeywordTok{as_tibble}\NormalTok{(}\KeywordTok{rbind.data.frame}\NormalTok{(}
  \KeywordTok{tidy}\NormalTok{(hill.water.mod.rich) }\OperatorTok\StringTok{ }\KeywordTok{add_column}\NormalTok{(}\DataTypeTok{Diversity =} \StringTok{"Richness"}\NormalTok{),}
  \KeywordTok{tidy}\NormalTok{(hill.water.mod.shan) }\OperatorTok\StringTok{ }\KeywordTok{add_column}\NormalTok{(}\DataTypeTok{Diversity =} \StringTok{"Shannon"}\NormalTok{),}
  \KeywordTok{tidy}\NormalTok{(hill.water.mod.simp) }\OperatorTok\StringTok{ }\KeywordTok{add_column}\NormalTok{(}\DataTypeTok{Diversity =} \StringTok{"Simpson"}\NormalTok{)}
\NormalTok{))}
\end{Highlighting}
\end{Shaded}

\begin{Shaded}
\begin{Highlighting}[]
\CommentTok{# Summary table of the model results. }
\NormalTok{hill.water.mods }\OperatorTok\StringTok{ }
\StringTok{  }\KeywordTok{group_by}\NormalTok{(Diversity) }\OperatorTok\StringTok{ }
\StringTok{  }\KeywordTok{rename}\NormalTok{(}\StringTok{"Term"}\NormalTok{ =}\StringTok{ }\NormalTok{term, }
         \StringTok{"Estimate"}\NormalTok{ =}\StringTok{ }\NormalTok{estimate, }
         \StringTok{"Std. Error"}\NormalTok{ =}\StringTok{ }\NormalTok{std.error, }
         \StringTok{"Statistic"}\NormalTok{ =}\StringTok{ }\NormalTok{statistic, }
         \StringTok{"p-value"}\NormalTok{ =}\StringTok{ }\NormalTok{p.value) }\OperatorTok\StringTok{ }
\StringTok{  }\KeywordTok{filter}\NormalTok{(Term }\OperatorTok{!=}\StringTok{ "(Intercept)"}\NormalTok{) }\OperatorTok\StringTok{ }
\StringTok{  }\KeywordTok{select}\NormalTok{(Diversity, }\KeywordTok{everything}\NormalTok{()) }\OperatorTok\StringTok{ }
\StringTok{  }\KeywordTok{pander}\NormalTok{(}\DataTypeTok{round =} \DecValTok{2}\NormalTok{)}
\end{Highlighting}
\end{Shaded}

\begin{longtable}[]{@{}cccccc@{}}
\toprule
\begin{minipage}[b]{0.12\columnwidth}\centering\strut
Diversity\strut
\end{minipage} & \begin{minipage}[b]{0.24\columnwidth}\centering\strut
Term\strut
\end{minipage} & \begin{minipage}[b]{0.11\columnwidth}\centering\strut
Estimate\strut
\end{minipage} & \begin{minipage}[b]{0.14\columnwidth}\centering\strut
Std. Error\strut
\end{minipage} & \begin{minipage}[b]{0.12\columnwidth}\centering\strut
Statistic\strut
\end{minipage} & \begin{minipage}[b]{0.09\columnwidth}\centering\strut
p-value\strut
\end{minipage}\tabularnewline
\midrule
\endhead
\begin{minipage}[t]{0.12\columnwidth}\centering\strut
Richness\strut
\end{minipage} & \begin{minipage}[t]{0.24\columnwidth}\centering\strut
distance\strut
\end{minipage} & \begin{minipage}[t]{0.11\columnwidth}\centering\strut
5.21\strut
\end{minipage} & \begin{minipage}[t]{0.14\columnwidth}\centering\strut
0.54\strut
\end{minipage} & \begin{minipage}[t]{0.12\columnwidth}\centering\strut
9.66\strut
\end{minipage} & \begin{minipage}[t]{0.09\columnwidth}\centering\strut
0\strut
\end{minipage}\tabularnewline
\begin{minipage}[t]{0.12\columnwidth}\centering\strut
Richness\strut
\end{minipage} & \begin{minipage}[t]{0.24\columnwidth}\centering\strut
moleculeRNA\strut
\end{minipage} & \begin{minipage}[t]{0.11\columnwidth}\centering\strut
104.9\strut
\end{minipage} & \begin{minipage}[t]{0.14\columnwidth}\centering\strut
166.2\strut
\end{minipage} & \begin{minipage}[t]{0.12\columnwidth}\centering\strut
0.63\strut
\end{minipage} & \begin{minipage}[t]{0.09\columnwidth}\centering\strut
0.53\strut
\end{minipage}\tabularnewline
\begin{minipage}[t]{0.12\columnwidth}\centering\strut
Richness\strut
\end{minipage} & \begin{minipage}[t]{0.24\columnwidth}\centering\strut
distance:moleculeRNA\strut
\end{minipage} & \begin{minipage}[t]{0.11\columnwidth}\centering\strut
-5.12\strut
\end{minipage} & \begin{minipage}[t]{0.14\columnwidth}\centering\strut
0.72\strut
\end{minipage} & \begin{minipage}[t]{0.12\columnwidth}\centering\strut
-7.15\strut
\end{minipage} & \begin{minipage}[t]{0.09\columnwidth}\centering\strut
0\strut
\end{minipage}\tabularnewline
\begin{minipage}[t]{0.12\columnwidth}\centering\strut
Shannon\strut
\end{minipage} & \begin{minipage}[t]{0.24\columnwidth}\centering\strut
distance\strut
\end{minipage} & \begin{minipage}[t]{0.11\columnwidth}\centering\strut
0.29\strut
\end{minipage} & \begin{minipage}[t]{0.14\columnwidth}\centering\strut
0.11\strut
\end{minipage} & \begin{minipage}[t]{0.12\columnwidth}\centering\strut
2.75\strut
\end{minipage} & \begin{minipage}[t]{0.09\columnwidth}\centering\strut
0.01\strut
\end{minipage}\tabularnewline
\begin{minipage}[t]{0.12\columnwidth}\centering\strut
Shannon\strut
\end{minipage} & \begin{minipage}[t]{0.24\columnwidth}\centering\strut
moleculeRNA\strut
\end{minipage} & \begin{minipage}[t]{0.11\columnwidth}\centering\strut
-35.51\strut
\end{minipage} & \begin{minipage}[t]{0.14\columnwidth}\centering\strut
32.34\strut
\end{minipage} & \begin{minipage}[t]{0.12\columnwidth}\centering\strut
-1.1\strut
\end{minipage} & \begin{minipage}[t]{0.09\columnwidth}\centering\strut
0.28\strut
\end{minipage}\tabularnewline
\begin{minipage}[t]{0.12\columnwidth}\centering\strut
Shannon\strut
\end{minipage} & \begin{minipage}[t]{0.24\columnwidth}\centering\strut
distance:moleculeRNA\strut
\end{minipage} & \begin{minipage}[t]{0.11\columnwidth}\centering\strut
-0.29\strut
\end{minipage} & \begin{minipage}[t]{0.14\columnwidth}\centering\strut
0.14\strut
\end{minipage} & \begin{minipage}[t]{0.12\columnwidth}\centering\strut
-2.08\strut
\end{minipage} & \begin{minipage}[t]{0.09\columnwidth}\centering\strut
0.05\strut
\end{minipage}\tabularnewline
\begin{minipage}[t]{0.12\columnwidth}\centering\strut
Simpson\strut
\end{minipage} & \begin{minipage}[t]{0.24\columnwidth}\centering\strut
distance\strut
\end{minipage} & \begin{minipage}[t]{0.11\columnwidth}\centering\strut
0.05\strut
\end{minipage} & \begin{minipage}[t]{0.14\columnwidth}\centering\strut
0.03\strut
\end{minipage} & \begin{minipage}[t]{0.12\columnwidth}\centering\strut
1.66\strut
\end{minipage} & \begin{minipage}[t]{0.09\columnwidth}\centering\strut
0.11\strut
\end{minipage}\tabularnewline
\begin{minipage}[t]{0.12\columnwidth}\centering\strut
Simpson\strut
\end{minipage} & \begin{minipage}[t]{0.24\columnwidth}\centering\strut
moleculeRNA\strut
\end{minipage} & \begin{minipage}[t]{0.11\columnwidth}\centering\strut
-22.19\strut
\end{minipage} & \begin{minipage}[t]{0.14\columnwidth}\centering\strut
9.65\strut
\end{minipage} & \begin{minipage}[t]{0.12\columnwidth}\centering\strut
-2.3\strut
\end{minipage} & \begin{minipage}[t]{0.09\columnwidth}\centering\strut
0.03\strut
\end{minipage}\tabularnewline
\begin{minipage}[t]{0.12\columnwidth}\centering\strut
Simpson\strut
\end{minipage} & \begin{minipage}[t]{0.24\columnwidth}\centering\strut
distance:moleculeRNA\strut
\end{minipage} & \begin{minipage}[t]{0.11\columnwidth}\centering\strut
-0.04\strut
\end{minipage} & \begin{minipage}[t]{0.14\columnwidth}\centering\strut
0.04\strut
\end{minipage} & \begin{minipage}[t]{0.12\columnwidth}\centering\strut
-1.06\strut
\end{minipage} & \begin{minipage}[t]{0.09\columnwidth}\centering\strut
0.3\strut
\end{minipage}\tabularnewline
\bottomrule
\end{longtable}

\begin{Shaded}
\begin{Highlighting}[]
\NormalTok{hill.estim }\OperatorTok\StringTok{ }\KeywordTok{filter}\NormalTok{(type }\OperatorTok{==}\StringTok{ "water"}\NormalTok{) }\OperatorTok\StringTok{ }
\StringTok{  }\KeywordTok{mutate}\NormalTok{(}\DataTypeTok{molecule =} \KeywordTok{ifelse}\NormalTok{(molecule }\OperatorTok{==}\StringTok{ "DNA"}\NormalTok{, }\StringTok{"Total"}\NormalTok{, }\StringTok{"Active"}\NormalTok{)) }\OperatorTok\StringTok{ }
\StringTok{  }\KeywordTok{ggplot}\NormalTok{(}\KeywordTok{aes}\NormalTok{(}\DataTypeTok{x =}\NormalTok{ distance, }\DataTypeTok{y =}\NormalTok{ Estimator, }
             \DataTypeTok{ymin =}\NormalTok{ LCL, }\DataTypeTok{ymax =}\NormalTok{ UCL,}
             \DataTypeTok{color =}\NormalTok{ molecule, }\DataTypeTok{fill =}\NormalTok{ molecule, }\DataTypeTok{shape =}\NormalTok{ molecule)) }\OperatorTok{+}\StringTok{ }
\StringTok{  }\KeywordTok{geom_point}\NormalTok{(}\DataTypeTok{size =}\DecValTok{3}\NormalTok{) }\OperatorTok{+}\StringTok{ }
\StringTok{  }\CommentTok{# geom_errorbar(size = .5, aes(ymin = Estimator - s.e., ymax = Estimator + s.e.), }
\StringTok{  }\CommentTok{#                width = 10, alpha = 0.5) +}
\StringTok{  }\KeywordTok{geom_smooth}\NormalTok{(}\DataTypeTok{method =} \StringTok{"lm"}\NormalTok{, }\KeywordTok{aes}\NormalTok{(}\DataTypeTok{linetype =}\NormalTok{ molecule)) }\OperatorTok{+}
\StringTok{  }\KeywordTok{labs}\NormalTok{(}\DataTypeTok{x =} \StringTok{"Reservoir Transect (m)"}\NormalTok{,}
       \DataTypeTok{y =} \StringTok{"Estimated Diversity (Species Equivalents)"}\NormalTok{) }\OperatorTok{+}
\StringTok{  }\KeywordTok{scale_color_manual}\NormalTok{(}\DataTypeTok{values =}\NormalTok{ my.cols) }\OperatorTok{+}
\StringTok{  }\KeywordTok{scale_fill_manual}\NormalTok{(}\DataTypeTok{values =}\NormalTok{ my.cols) }\OperatorTok{+}\StringTok{ }
\StringTok{  }\KeywordTok{theme}\NormalTok{(}\DataTypeTok{legend.position =} \KeywordTok{c}\NormalTok{(.}\DecValTok{88}\NormalTok{,.}\DecValTok{95}\NormalTok{)) }\OperatorTok{+}
\StringTok{  }\KeywordTok{scale_x_reverse}\NormalTok{() }\OperatorTok{+}
\StringTok{  }\KeywordTok{facet_wrap}\NormalTok{(}\OperatorTok{~}\StringTok{ }\NormalTok{Diversity, }\DataTypeTok{scales =} \StringTok{"free"}\NormalTok{, }\DataTypeTok{ncol =} \DecValTok{1}\NormalTok{)}
\end{Highlighting}
\end{Shaded}

\begin{center}\includegraphics{ReservoirGradient_files/figure-latex/hill_div_plot-1} \end{center}

\begin{Shaded}
\begin{Highlighting}[]
  \CommentTok{#facet_grid(Diversity ~ ., scales = "free")}
\end{Highlighting}
\end{Shaded}

So, from the basis of these results, we can make the following
conclusions. First, we note that diversity in the total community decays
from the stream inlet to the dam of the reservoir. That is, all the
lines have a negative slope. However, we do not see this decay in the
metabolically active community. Second, we note that the metabolically
actively community has much lower diversity than the total community
near the soils, but this difference decreases toward the dam. Last,
because we quantified diversity across three orders of Hill numbers (q =
0, 1, and 2), we can also say something about the relative importance of
rare versus common taxa along the reservoir transect. We see the the
significance of the distance-by-molecule interaction term decrease as
rare taxa are downweighted in favor of common taxa. This suggests that
the differences between the active and total communities along the
transect is driven primarily by rare taxa. However, the general trend of
higher Simpson diversity across the whole transect suggests that
low-activity, but relatively common, taxa are maintained in the
reservoir.

\subsection{How does community structure change along the
gradient?}\label{how-does-community-structure-change-along-the-gradient}

First, we'll just get an overview of how the communities look along the
aquatic transect.

\begin{Shaded}
\begin{Highlighting}[]
\NormalTok{ul.pcoa <-}\StringTok{ }\KeywordTok{cmdscale}\NormalTok{(}\KeywordTok{vegdist}\NormalTok{(OTUsREL.log, }\DataTypeTok{method=}\StringTok{"bray"}\NormalTok{), }\DecValTok{2}\NormalTok{, }\DataTypeTok{eig =}\NormalTok{ T, }\DataTypeTok{add =}\NormalTok{ T)}
\NormalTok{explainvars <-}\StringTok{ }\KeywordTok{round}\NormalTok{(}\KeywordTok{eigenvals}\NormalTok{(ul.pcoa)[}\KeywordTok{c}\NormalTok{(}\DecValTok{1}\NormalTok{,}\DecValTok{2}\NormalTok{)]}\OperatorTok{/}\KeywordTok{sum}\NormalTok{(}\KeywordTok{eigenvals}\NormalTok{(ul.pcoa)),}\DecValTok{3}\NormalTok{) }\OperatorTok{*}\DecValTok{100}
\NormalTok{water.pcvals <-}\StringTok{ }\KeywordTok{data.frame}\NormalTok{(}\KeywordTok{scores}\NormalTok{(ul.pcoa)) }\OperatorTok\StringTok{ }
\StringTok{  }\KeywordTok{rownames_to_column}\NormalTok{(}\StringTok{"name"}\NormalTok{) }\OperatorTok\StringTok{ }
\StringTok{  }\KeywordTok{left_join}\NormalTok{(}\KeywordTok{rownames_to_column}\NormalTok{(design, }\StringTok{"name"}\NormalTok{)) }\OperatorTok\StringTok{ }
\StringTok{  }\KeywordTok{arrange}\NormalTok{(}\KeywordTok{desc}\NormalTok{(distance)) }\OperatorTok\StringTok{ }\KeywordTok{filter}\NormalTok{(type }\OperatorTok{==}\StringTok{ "water"}\NormalTok{)}
\NormalTok{pc_dists <-}\StringTok{ }\KeywordTok{tibble}\NormalTok{(}
  \DataTypeTok{DNA_dim1 =} \KeywordTok{subset}\NormalTok{(water.pcvals, molecule }\OperatorTok{==}\StringTok{ "DNA"}\NormalTok{)}\OperatorTok{$}\NormalTok{Dim1,}
  \DataTypeTok{DNA_dim2 =} \KeywordTok{subset}\NormalTok{(water.pcvals, molecule }\OperatorTok{==}\StringTok{ "DNA"}\NormalTok{)}\OperatorTok{$}\NormalTok{Dim2,}
  \DataTypeTok{RNA_dim1 =} \KeywordTok{subset}\NormalTok{(water.pcvals, molecule }\OperatorTok{==}\StringTok{ "RNA"}\NormalTok{)}\OperatorTok{$}\NormalTok{Dim1,}
  \DataTypeTok{RNA_dim2 =} \KeywordTok{subset}\NormalTok{(water.pcvals, molecule }\OperatorTok{==}\StringTok{ "RNA"}\NormalTok{)}\OperatorTok{$}\NormalTok{Dim2)}
\KeywordTok{data.frame}\NormalTok{(}\KeywordTok{scores}\NormalTok{(ul.pcoa)) }\OperatorTok\StringTok{ }
\StringTok{  }\KeywordTok{rownames_to_column}\NormalTok{(}\StringTok{"name"}\NormalTok{) }\OperatorTok\StringTok{ }
\StringTok{  }\KeywordTok{left_join}\NormalTok{(}\KeywordTok{rownames_to_column}\NormalTok{(design, }\StringTok{"name"}\NormalTok{)) }\OperatorTok\StringTok{ }
\StringTok{  }\KeywordTok{arrange}\NormalTok{(}\KeywordTok{desc}\NormalTok{(distance)) }\OperatorTok\StringTok{ }\KeywordTok{filter}\NormalTok{(type }\OperatorTok{==}\StringTok{ "water"}\NormalTok{) }\OperatorTok\StringTok{ }
\StringTok{  }\KeywordTok{mutate}\NormalTok{(}\DataTypeTok{molecule =} \KeywordTok{ifelse}\NormalTok{(molecule }\OperatorTok{==}\StringTok{ "DNA"}\NormalTok{, }\StringTok{"Total"}\NormalTok{, }\StringTok{"Active"}\NormalTok{)) }\OperatorTok\StringTok{ }
\StringTok{  }\KeywordTok{ggplot}\NormalTok{(}\KeywordTok{aes}\NormalTok{(}\DataTypeTok{x =}\NormalTok{ Dim1, }\DataTypeTok{y =}\NormalTok{ Dim2)) }\OperatorTok{+}
\StringTok{  }\KeywordTok{geom_path}\NormalTok{(}\DataTypeTok{size =} \DecValTok{1}\NormalTok{, }\DataTypeTok{alpha =} \FloatTok{0.75}\NormalTok{, }\DataTypeTok{arrow =} \KeywordTok{arrow}\NormalTok{(}\DataTypeTok{angle =} \DecValTok{20}\NormalTok{,}
                          \DataTypeTok{length =} \KeywordTok{unit}\NormalTok{(}\FloatTok{0.35}\NormalTok{, }\StringTok{"cm"}\NormalTok{),}
                          \DataTypeTok{type =} \StringTok{"closed"}\NormalTok{), }\KeywordTok{aes}\NormalTok{(}\DataTypeTok{color =}\NormalTok{ molecule, }\DataTypeTok{linetype =}\NormalTok{ molecule)) }\OperatorTok{+}
\StringTok{  }\KeywordTok{geom_point}\NormalTok{(}\DataTypeTok{size =} \DecValTok{3}\NormalTok{, }\DataTypeTok{alpha =} \FloatTok{0.8}\NormalTok{, }\KeywordTok{aes}\NormalTok{(}\DataTypeTok{color =}\NormalTok{ molecule, }\DataTypeTok{shape =}\NormalTok{ molecule)) }\OperatorTok{+}\StringTok{ }
\StringTok{  }\KeywordTok{scale_color_manual}\NormalTok{(}\StringTok{"Community Subset"}\NormalTok{, }\DataTypeTok{values =}\NormalTok{ my.cols) }\OperatorTok{+}
\StringTok{  }\KeywordTok{geom_segment}\NormalTok{(}\DataTypeTok{data =}\NormalTok{ pc_dists,}
               \KeywordTok{aes}\NormalTok{(}\DataTypeTok{x =}\NormalTok{ DNA_dim1, }\DataTypeTok{y =}\NormalTok{ DNA_dim2,}
                   \DataTypeTok{xend =}\NormalTok{ RNA_dim1, }\DataTypeTok{yend =}\NormalTok{ RNA_dim2),}
               \DataTypeTok{alpha =} \DecValTok{0}\NormalTok{) }\OperatorTok{+}
\StringTok{  }\KeywordTok{coord_fixed}\NormalTok{() }\OperatorTok{+}
\StringTok{  }\KeywordTok{labs}\NormalTok{(}\DataTypeTok{x =} \KeywordTok{paste0}\NormalTok{(}\StringTok{"PCoA 1 ("}\NormalTok{, explainvars[}\DecValTok{1}\NormalTok{],}\StringTok{"%)"}\NormalTok{),}
       \DataTypeTok{y =} \KeywordTok{paste0}\NormalTok{(}\StringTok{"PCoA 2 ("}\NormalTok{, explainvars[}\DecValTok{2}\NormalTok{],}\StringTok{"%)"}\NormalTok{)) }\OperatorTok{+}
\StringTok{  }\KeywordTok{theme}\NormalTok{(}\DataTypeTok{legend.position =} \StringTok{"none"}\NormalTok{) }\OperatorTok{+}
\StringTok{  }\KeywordTok{annotate}\NormalTok{(}\DataTypeTok{geom =} \StringTok{"text"}\NormalTok{, }\DataTypeTok{x =}\NormalTok{ .}\DecValTok{2}\NormalTok{, }\DataTypeTok{y =} \OperatorTok{-}\NormalTok{.}\DecValTok{33}\NormalTok{, }\DataTypeTok{label =} \StringTok{"Active"}\NormalTok{, }\DataTypeTok{size =} \DecValTok{6}\NormalTok{) }\OperatorTok{+}
\StringTok{  }\KeywordTok{annotate}\NormalTok{(}\DataTypeTok{geom =} \StringTok{"text"}\NormalTok{, }\DataTypeTok{x =} \OperatorTok{-}\NormalTok{.}\DecValTok{3}\NormalTok{, }\DataTypeTok{y =}\NormalTok{ .}\DecValTok{1}\NormalTok{, }\DataTypeTok{label =} \StringTok{"Total"}\NormalTok{, }\DataTypeTok{size =} \DecValTok{6}\NormalTok{)}
\end{Highlighting}
\end{Shaded}

\begin{center}\includegraphics{ReservoirGradient_files/figure-latex/ordination-1} \end{center}

So, it appears that there is convergence in community structure along
the path from stream inlet to the dam. This could reflect a loss of
soil-derived taxa in the aquatic samples. To test this, we'll look at
\(\beta\)-diversity along the gradient with respect to the soil samples.
If we see a decay in similarity to soils, this suggests soil taxa are
having a comparatively lower influence with distance from the inlet.

\subsection{Similarity To Terrestrial Habitat Across Gradient
(Terrestrial
Influence)}\label{similarity-to-terrestrial-habitat-across-gradient-terrestrial-influence}

Here, we fit a linear model to the similarity of the aquatic community
to the soil community.

\begin{Shaded}
\begin{Highlighting}[]
\CommentTok{# Similarity to Soil Sample}
\NormalTok{UL.bray      <-}\StringTok{ }\DecValTok{1}\OperatorTok{-}\KeywordTok{as.matrix}\NormalTok{(}\KeywordTok{vegdist}\NormalTok{(OTUsREL.log, }\DataTypeTok{method=}\StringTok{"bray"}\NormalTok{))}
\NormalTok{UL.bray.lake <-}\StringTok{ }\NormalTok{UL.bray[}\OperatorTok{-}\KeywordTok{c}\NormalTok{(}\DecValTok{1}\OperatorTok{:}\DecValTok{3}\NormalTok{), }\DecValTok{1}\OperatorTok{:}\DecValTok{3}\NormalTok{] }
\NormalTok{bray.mean    <-}\StringTok{ }\KeywordTok{round}\NormalTok{(}\KeywordTok{apply}\NormalTok{(UL.bray.lake, }\DecValTok{1}\NormalTok{, mean), }\DecValTok{3}\NormalTok{)}
\NormalTok{bray.se      <-}\StringTok{ }\KeywordTok{round}\NormalTok{(}\KeywordTok{apply}\NormalTok{(UL.bray.lake, }\DecValTok{1}\NormalTok{, se), }\DecValTok{3}\NormalTok{)}
\NormalTok{UL.sim       <-}\StringTok{ }\KeywordTok{cbind}\NormalTok{(design[}\OperatorTok{-}\KeywordTok{c}\NormalTok{(}\DecValTok{1}\OperatorTok{:}\DecValTok{3}\NormalTok{), ], bray.mean, bray.se)}

\CommentTok{# Calculate Linear Model}
\NormalTok{model.terr <-}\StringTok{ }\KeywordTok{lm}\NormalTok{(bray.mean }\OperatorTok{~}\StringTok{ }\NormalTok{distance }\OperatorTok{*}\StringTok{ }\NormalTok{molecule, }\DataTypeTok{data =}\NormalTok{ UL.sim)}
\KeywordTok{pander}\NormalTok{(model.terr)}
\end{Highlighting}
\end{Shaded}

\begin{longtable}[]{@{}ccccc@{}}
\caption{Fitting linear model: bray.mean \textasciitilde{} distance *
molecule}\tabularnewline
\toprule
\begin{minipage}[b]{0.31\columnwidth}\centering\strut
~\strut
\end{minipage} & \begin{minipage}[b]{0.15\columnwidth}\centering\strut
Estimate\strut
\end{minipage} & \begin{minipage}[b]{0.15\columnwidth}\centering\strut
Std. Error\strut
\end{minipage} & \begin{minipage}[b]{0.12\columnwidth}\centering\strut
t value\strut
\end{minipage} & \begin{minipage}[b]{0.13\columnwidth}\centering\strut
Pr(\textgreater{}\textbar{}t\textbar{})\strut
\end{minipage}\tabularnewline
\midrule
\endfirsthead
\toprule
\begin{minipage}[b]{0.31\columnwidth}\centering\strut
~\strut
\end{minipage} & \begin{minipage}[b]{0.15\columnwidth}\centering\strut
Estimate\strut
\end{minipage} & \begin{minipage}[b]{0.15\columnwidth}\centering\strut
Std. Error\strut
\end{minipage} & \begin{minipage}[b]{0.12\columnwidth}\centering\strut
t value\strut
\end{minipage} & \begin{minipage}[b]{0.13\columnwidth}\centering\strut
Pr(\textgreater{}\textbar{}t\textbar{})\strut
\end{minipage}\tabularnewline
\midrule
\endhead
\begin{minipage}[t]{0.31\columnwidth}\centering\strut
\textbf{(Intercept)}\strut
\end{minipage} & \begin{minipage}[t]{0.15\columnwidth}\centering\strut
0.01524\strut
\end{minipage} & \begin{minipage}[t]{0.15\columnwidth}\centering\strut
0.01623\strut
\end{minipage} & \begin{minipage}[t]{0.12\columnwidth}\centering\strut
0.9392\strut
\end{minipage} & \begin{minipage}[t]{0.13\columnwidth}\centering\strut
0.3551\strut
\end{minipage}\tabularnewline
\begin{minipage}[t]{0.31\columnwidth}\centering\strut
\textbf{distance}\strut
\end{minipage} & \begin{minipage}[t]{0.15\columnwidth}\centering\strut
0.0004564\strut
\end{minipage} & \begin{minipage}[t]{0.15\columnwidth}\centering\strut
6.828e-05\strut
\end{minipage} & \begin{minipage}[t]{0.12\columnwidth}\centering\strut
6.684\strut
\end{minipage} & \begin{minipage}[t]{0.13\columnwidth}\centering\strut
2.097e-07\strut
\end{minipage}\tabularnewline
\begin{minipage}[t]{0.31\columnwidth}\centering\strut
\textbf{moleculeRNA}\strut
\end{minipage} & \begin{minipage}[t]{0.15\columnwidth}\centering\strut
0.01321\strut
\end{minipage} & \begin{minipage}[t]{0.15\columnwidth}\centering\strut
0.02281\strut
\end{minipage} & \begin{minipage}[t]{0.12\columnwidth}\centering\strut
0.579\strut
\end{minipage} & \begin{minipage}[t]{0.13\columnwidth}\centering\strut
0.5669\strut
\end{minipage}\tabularnewline
\begin{minipage}[t]{0.31\columnwidth}\centering\strut
\textbf{distance:moleculeRNA}\strut
\end{minipage} & \begin{minipage}[t]{0.15\columnwidth}\centering\strut
-0.0004269\strut
\end{minipage} & \begin{minipage}[t]{0.15\columnwidth}\centering\strut
9.608e-05\strut
\end{minipage} & \begin{minipage}[t]{0.12\columnwidth}\centering\strut
-4.443\strut
\end{minipage} & \begin{minipage}[t]{0.13\columnwidth}\centering\strut
0.0001117\strut
\end{minipage}\tabularnewline
\bottomrule
\end{longtable}

\begin{Shaded}
\begin{Highlighting}[]
\CommentTok{# # Calculate Confidance Intervals of Model}
\CommentTok{# newdata.terr <- data.frame(cbind(UL.sim$molecule, UL.sim$distance))}
\CommentTok{# conf95.terr <- predict(model.terr, newdata.terr, interval="confidence")}
\CommentTok{# }
\CommentTok{# # Dummy Variables Regression Model ("Terrestrial Influence")}
\CommentTok{# D2 <- (UL.sim$molecule == "RNA")*1}
\CommentTok{# fit.Fig.3b <- lm(UL.sim$bray.mean ~ UL.sim$distance + D2 + UL.sim$distance*D2)}
\CommentTok{# D2.R2 <- round(summary(fit.Fig.3b)$r.squared, 2)}
\CommentTok{# summary(fit.Fig.3b)}
\CommentTok{# }
\CommentTok{# }
\CommentTok{# DNA.int.3b <- fit.Fig.3b$coefficients[1]}
\CommentTok{# DNA.slp.3b <- fit.Fig.3b$coefficients[2]}
\CommentTok{# RNA.int.3b <- DNA.int.3b + fit.Fig.3b$coefficients[3]}
\CommentTok{# RNA.slp.3b <- DNA.slp.3b + fit.Fig.3b$coefficients[4]}
\end{Highlighting}
\end{Shaded}

\begin{Shaded}
\begin{Highlighting}[]
\NormalTok{UL.sim }\OperatorTok\StringTok{ }
\StringTok{  }\KeywordTok{mutate}\NormalTok{(}\DataTypeTok{molecule =} \KeywordTok{ifelse}\NormalTok{(UL.sim}\OperatorTok{$}\NormalTok{molecule }\OperatorTok{==}\StringTok{ "DNA"}\NormalTok{, }\StringTok{"Total"}\NormalTok{, }\StringTok{"Active"}\NormalTok{)) }\OperatorTok\StringTok{ }
\StringTok{  }\KeywordTok{ggplot}\NormalTok{(}\KeywordTok{aes}\NormalTok{(}\DataTypeTok{x =}\NormalTok{ distance, }\DataTypeTok{y =}\NormalTok{ bray.mean, }
             \DataTypeTok{color =}\NormalTok{ molecule, }\DataTypeTok{fill =}\NormalTok{ molecule, }\DataTypeTok{shape =}\NormalTok{ molecule)) }\OperatorTok{+}
\StringTok{  }\KeywordTok{geom_point}\NormalTok{(}\DataTypeTok{alpha =} \FloatTok{0.8}\NormalTok{, }\DataTypeTok{size =} \DecValTok{3}\NormalTok{, }\DataTypeTok{show.legend =}\NormalTok{ T) }\OperatorTok{+}\StringTok{ }
\StringTok{  }\KeywordTok{geom_smooth}\NormalTok{(}\DataTypeTok{method =} \StringTok{"lm"}\NormalTok{, }\DataTypeTok{show.legend =}\NormalTok{ T, }\KeywordTok{aes}\NormalTok{(}\DataTypeTok{linetype =}\NormalTok{ molecule)) }\OperatorTok{+}\StringTok{ }
\StringTok{  }\KeywordTok{labs}\NormalTok{(}\DataTypeTok{y =} \KeywordTok{str_wrap}\NormalTok{(}\StringTok{"Similarity to Soil Community"}\NormalTok{, }\DataTypeTok{width =} \DecValTok{20}\NormalTok{), }
       \DataTypeTok{x =} \StringTok{"Reservoir Transect (m)"}\NormalTok{) }\OperatorTok{+}\StringTok{ }
\StringTok{  }\KeywordTok{scale_color_manual}\NormalTok{(}\DataTypeTok{values =}\NormalTok{ my.cols) }\OperatorTok{+}\StringTok{ }
\StringTok{  }\KeywordTok{scale_fill_manual}\NormalTok{(}\DataTypeTok{values =}\NormalTok{ my.cols) }\OperatorTok{+}
\StringTok{  }\KeywordTok{theme}\NormalTok{(}\DataTypeTok{legend.position =} \KeywordTok{c}\NormalTok{(}\FloatTok{0.85}\NormalTok{, }\FloatTok{0.85}\NormalTok{)) }\OperatorTok{+}
\StringTok{  }\KeywordTok{scale_x_reverse}\NormalTok{(}\DataTypeTok{limits =} \KeywordTok{c}\NormalTok{(}\DecValTok{400}\NormalTok{,}\DecValTok{0}\NormalTok{)) }\OperatorTok{+}\StringTok{ }
\StringTok{  }\KeywordTok{annotate}\NormalTok{(}\DataTypeTok{geom =} \StringTok{"text"}\NormalTok{, }\DataTypeTok{x =} \DecValTok{50}\NormalTok{, }\DataTypeTok{y =} \FloatTok{0.15}\NormalTok{, }\DataTypeTok{size =} \DecValTok{6}\NormalTok{,}
           \DataTypeTok{label =} \KeywordTok{paste0}\NormalTok{(}\StringTok{"R^2== "}\NormalTok{,}\KeywordTok{round}\NormalTok{(}\KeywordTok{summary}\NormalTok{(model.terr)}\OperatorTok{$}\NormalTok{r.squared, }\DecValTok{2}\NormalTok{)), }\DataTypeTok{parse =}\NormalTok{ T)}
\end{Highlighting}
\end{Shaded}

\begin{center}\includegraphics{ReservoirGradient_files/figure-latex/plot_similarity_to_soils-1} \end{center}

We find that our model captures most of the variation in community
structure \((R^2 = 0.74986)\). We note a significant influence of
distance on community similarity and the presence of a significant
interaction between distance and whether the comparison is for active or
total bacterial communities. This indicates that total communities decay
faster with distance to soils than active communities do, which might be
explained by the large difference in initial intercept. Active
communities are always highly dissimilar to soil communities and remain
so across the lake, while total lake communities are initially similar
to soils, but this influence dissipates with distance into the
reservoir.

\section{Identifying the Soil
Bacteria}\label{identifying-the-soil-bacteria}

Now, we wish to determine whether soil-derived taxa are driving this
pattern, and then ask who these influential soil bacteria are.

To classify soil bacteria, we take an incidence-based approach and
classify OTUs as:\\
- present in the soil and present, but never active, in the reservoir\\
- present in the soil and active in the reservoir

\begin{Shaded}
\begin{Highlighting}[]
\CommentTok{# separate lake and soil samples}
\NormalTok{lake.total <-}\StringTok{ }\NormalTok{OTUs[}\KeywordTok{which}\NormalTok{(design}\OperatorTok{$}\NormalTok{molecule }\OperatorTok{==}\StringTok{ "DNA"}\NormalTok{, design}\OperatorTok{$}\NormalTok{type }\OperatorTok{==}\StringTok{ "water"}\NormalTok{),]}
\NormalTok{soil.total <-}\StringTok{ }\NormalTok{OTUs[}\KeywordTok{which}\NormalTok{(design}\OperatorTok{$}\NormalTok{molecule }\OperatorTok{==}\StringTok{ "DNA"}\NormalTok{, design}\OperatorTok{$}\NormalTok{type }\OperatorTok{==}\StringTok{ "soil"}\NormalTok{),]}

\CommentTok{# which otus are present in both lake and soil samples}
\NormalTok{lake.and.soil.total <-}\StringTok{ }\NormalTok{OTUs[}\KeywordTok{which}\NormalTok{(design}\OperatorTok{$}\NormalTok{molecule }\OperatorTok{==}\StringTok{ "DNA"}\NormalTok{, design}\OperatorTok{$}\NormalTok{type }\OperatorTok{==}\StringTok{ "water"}\NormalTok{),}
                            \KeywordTok{which}\NormalTok{(}\KeywordTok{colSums}\NormalTok{(lake.total) }\OperatorTok{>}\StringTok{ }\DecValTok{0} \OperatorTok{&}\StringTok{ }\KeywordTok{colSums}\NormalTok{(soil.total) }\OperatorTok{>}\StringTok{ }\DecValTok{0}\NormalTok{)]}

\CommentTok{# isolate just the dna and rna lake communities}
\NormalTok{w.dna <-}\StringTok{ }\NormalTok{OTUs[}\KeywordTok{which}\NormalTok{(design}\OperatorTok{$}\NormalTok{molecule }\OperatorTok{==}\StringTok{ "DNA"} \OperatorTok{&}\StringTok{ }\NormalTok{design}\OperatorTok{$}\NormalTok{type }\OperatorTok{==}\StringTok{ "water"}\NormalTok{), ]}
\NormalTok{w.rna <-}\StringTok{ }\NormalTok{OTUs[}\KeywordTok{which}\NormalTok{(design}\OperatorTok{$}\NormalTok{molecule }\OperatorTok{==}\StringTok{ "RNA"} \OperatorTok{&}\StringTok{ }\NormalTok{design}\OperatorTok{$}\NormalTok{type }\OperatorTok{==}\StringTok{ "water"}\NormalTok{), ]}

\CommentTok{# pull out the lake rna counts for otus found in lake and soil}
\NormalTok{lake.and.soil.act <-}\StringTok{ }\NormalTok{w.rna[,}\KeywordTok{colnames}\NormalTok{(lake.and.soil.total)]}

\CommentTok{# of these lake and soil taxa, which are never active?}
\NormalTok{nvr.act <-}\StringTok{ }\KeywordTok{which}\NormalTok{(}\KeywordTok{colSums}\NormalTok{(lake.and.soil.act) }\OperatorTok{==}\StringTok{ }\DecValTok{0}\NormalTok{)}
\end{Highlighting}
\end{Shaded}

We calculate the richness of the soil taxa that are never active in the
lake. We calculate richness from the DNA-based samples.

\begin{Shaded}
\begin{Highlighting}[]
\CommentTok{# pull out their dna abundances and calculate richness}
\NormalTok{terr.lake <-}\StringTok{ }\NormalTok{w.dna[ , }\KeywordTok{c}\NormalTok{(}\KeywordTok{names}\NormalTok{(nvr.act))]}
\NormalTok{terr.rich <-}\StringTok{ }\KeywordTok{rowSums}\NormalTok{((terr.lake }\OperatorTok{>}\StringTok{ }\DecValTok{0}\NormalTok{) }\OperatorTok{*}\StringTok{ }\DecValTok{1}\NormalTok{)}
\NormalTok{terr.REL <-}\StringTok{ }\KeywordTok{rowSums}\NormalTok{(terr.lake) }\OperatorTok{/}\StringTok{ }\KeywordTok{rowSums}\NormalTok{(w.dna) }
\NormalTok{design.dna <-}\StringTok{ }\NormalTok{design[}\KeywordTok{which}\NormalTok{(design}\OperatorTok{$}\NormalTok{molecule }\OperatorTok{==}\StringTok{ "DNA"} \OperatorTok{&}\StringTok{ }\NormalTok{design}\OperatorTok{$}\NormalTok{type }\OperatorTok{==}\StringTok{ "water"}\NormalTok{), ]}
\NormalTok{terr.rich.log <-}\StringTok{ }\KeywordTok{log10}\NormalTok{(terr.rich)}
\NormalTok{terr.REL.log <-}\StringTok{ }\KeywordTok{log10}\NormalTok{(terr.REL)}

\NormalTok{terr.mod1 <-}\StringTok{ }\KeywordTok{lm}\NormalTok{(terr.rich.log }\OperatorTok{~}\StringTok{ }\NormalTok{design.dna}\OperatorTok{$}\NormalTok{distance)}
\CommentTok{#summary(terr.mod1)}
\NormalTok{T1.R2 <-}\StringTok{ }\KeywordTok{round}\NormalTok{(}\KeywordTok{summary}\NormalTok{(terr.mod1)}\OperatorTok{$}\NormalTok{r.squared, }\DecValTok{2}\NormalTok{)}
\NormalTok{T1.int <-}\StringTok{ }\NormalTok{terr.mod1}\OperatorTok{$}\NormalTok{coefficients[}\DecValTok{1}\NormalTok{]}
\NormalTok{T1.slp <-}\StringTok{ }\NormalTok{terr.mod1}\OperatorTok{$}\NormalTok{coefficients[}\DecValTok{2}\NormalTok{]}
\KeywordTok{pander}\NormalTok{(terr.mod1)}
\end{Highlighting}
\end{Shaded}

\begin{longtable}[]{@{}ccccc@{}}
\caption{Fitting linear model: terr.rich.log \textasciitilde{}
design.dna\$distance We find distance is a highly significant predictor
of the richness of these soil-derived taxa (on a
log-scale).}\tabularnewline
\toprule
\begin{minipage}[b]{0.31\columnwidth}\centering\strut
~\strut
\end{minipage} & \begin{minipage}[b]{0.13\columnwidth}\centering\strut
Estimate\strut
\end{minipage} & \begin{minipage}[b]{0.16\columnwidth}\centering\strut
Std. Error\strut
\end{minipage} & \begin{minipage}[b]{0.12\columnwidth}\centering\strut
t value\strut
\end{minipage} & \begin{minipage}[b]{0.13\columnwidth}\centering\strut
Pr(\textgreater{}\textbar{}t\textbar{})\strut
\end{minipage}\tabularnewline
\midrule
\endfirsthead
\toprule
\begin{minipage}[b]{0.31\columnwidth}\centering\strut
~\strut
\end{minipage} & \begin{minipage}[b]{0.13\columnwidth}\centering\strut
Estimate\strut
\end{minipage} & \begin{minipage}[b]{0.16\columnwidth}\centering\strut
Std. Error\strut
\end{minipage} & \begin{minipage}[b]{0.12\columnwidth}\centering\strut
t value\strut
\end{minipage} & \begin{minipage}[b]{0.13\columnwidth}\centering\strut
Pr(\textgreater{}\textbar{}t\textbar{})\strut
\end{minipage}\tabularnewline
\midrule
\endhead
\begin{minipage}[t]{0.31\columnwidth}\centering\strut
\textbf{(Intercept)}\strut
\end{minipage} & \begin{minipage}[t]{0.13\columnwidth}\centering\strut
2.016\strut
\end{minipage} & \begin{minipage}[t]{0.16\columnwidth}\centering\strut
0.07715\strut
\end{minipage} & \begin{minipage}[t]{0.12\columnwidth}\centering\strut
26.13\strut
\end{minipage} & \begin{minipage}[t]{0.13\columnwidth}\centering\strut
6.362e-14\strut
\end{minipage}\tabularnewline
\begin{minipage}[t]{0.31\columnwidth}\centering\strut
\textbf{design.dna\$distance}\strut
\end{minipage} & \begin{minipage}[t]{0.13\columnwidth}\centering\strut
0.003161\strut
\end{minipage} & \begin{minipage}[t]{0.16\columnwidth}\centering\strut
0.0003245\strut
\end{minipage} & \begin{minipage}[t]{0.12\columnwidth}\centering\strut
9.74\strut
\end{minipage} & \begin{minipage}[t]{0.13\columnwidth}\centering\strut
7.059e-08\strut
\end{minipage}\tabularnewline
\bottomrule
\end{longtable}

\begin{Shaded}
\begin{Highlighting}[]
\KeywordTok{tibble}\NormalTok{(}\DataTypeTok{transient_rich =}\NormalTok{ terr.rich, }\DataTypeTok{distance =}\NormalTok{ design.dna}\OperatorTok{$}\NormalTok{distance) }\OperatorTok\StringTok{ }
\StringTok{  }\KeywordTok{ggplot}\NormalTok{(}\KeywordTok{aes}\NormalTok{(}\DataTypeTok{x =}\NormalTok{ distance, }\DataTypeTok{y =}\NormalTok{ transient_rich)) }\OperatorTok{+}\StringTok{ }
\StringTok{  }\KeywordTok{geom_smooth}\NormalTok{(}\DataTypeTok{method =} \StringTok{"lm"}\NormalTok{, }\DataTypeTok{color =} \StringTok{"black"}\NormalTok{, }\DataTypeTok{fill =} \StringTok{"grey"}\NormalTok{) }\OperatorTok{+}
\StringTok{  }\KeywordTok{geom_point}\NormalTok{(}\DataTypeTok{alpha =} \DecValTok{1}\NormalTok{, }\DataTypeTok{color =} \StringTok{"black"}\NormalTok{) }\OperatorTok{+}\StringTok{ }
\StringTok{  }\KeywordTok{scale_x_reverse}\NormalTok{(}\DataTypeTok{limits =} \KeywordTok{c}\NormalTok{(}\DecValTok{400}\NormalTok{,}\DecValTok{0}\NormalTok{)) }\OperatorTok{+}
\StringTok{  }\KeywordTok{scale_y_log10}\NormalTok{() }\OperatorTok{+}
\StringTok{  }\KeywordTok{annotation_logticks}\NormalTok{(}\DataTypeTok{sides =} \StringTok{"l"}\NormalTok{) }\OperatorTok{+}
\StringTok{  }\KeywordTok{labs}\NormalTok{(}\DataTypeTok{x =} \StringTok{"Reservoir Transect (m)"}\NormalTok{,}
       \DataTypeTok{y =} \StringTok{"Richness of }\CharTok{\textbackslash{}n}\StringTok{Soil-derived Taxa"}\NormalTok{) }\OperatorTok{+}
\StringTok{  }\KeywordTok{annotate}\NormalTok{(}\StringTok{"text"}\NormalTok{, }\DataTypeTok{x =} \DecValTok{50}\NormalTok{, }\DataTypeTok{y =} \DecValTok{750}\NormalTok{, }\DataTypeTok{size =} \DecValTok{6}\NormalTok{, }\DataTypeTok{label =} \KeywordTok{paste0}\NormalTok{(}\StringTok{"R^2== "}\NormalTok{,T1.R2), }\DataTypeTok{parse =}\NormalTok{ T) }
\end{Highlighting}
\end{Shaded}

\begin{center}\includegraphics{ReservoirGradient_files/figure-latex/plot_transient-1} \end{center}

\section{What is the fate of soil-derived taxa in the
reservoir?}\label{what-is-the-fate-of-soil-derived-taxa-in-the-reservoir}

So, we observe that most soil-derived taxa appear to decay once they
enter the reservoir. Do any soil-derived taxa persist in the active
bacterial community of the reservoir and do they rise to high relative
abundances?

\begin{Shaded}
\begin{Highlighting}[]
\CommentTok{# identify otus in soil samples and lake samples}
\NormalTok{in.soil <-}\StringTok{ }\NormalTok{OTUs[, }\KeywordTok{which}\NormalTok{(}\KeywordTok{colSums}\NormalTok{(OTUs[}\KeywordTok{c}\NormalTok{(}\DecValTok{1}\OperatorTok{:}\DecValTok{3}\NormalTok{),]) }\OperatorTok{>}\StringTok{ }\DecValTok{0}\NormalTok{ )]}
\CommentTok{#in.lake <- OTUs[, which(colSums(OTUs[-c(1:3),]) > 0)]}

\CommentTok{# isolate just the rna water samples and convert to presence-absence}
\NormalTok{in.lake.rna <-}\StringTok{ }\NormalTok{OTUs[}\KeywordTok{which}\NormalTok{(design}\OperatorTok{$}\NormalTok{molecule }\OperatorTok{==}\StringTok{ "RNA"} \OperatorTok{&}\StringTok{ }\NormalTok{design}\OperatorTok{$}\NormalTok{type }\OperatorTok{==}\StringTok{ "water"}\NormalTok{), ]}
\NormalTok{in.lake.rna.pa <-}\StringTok{ }\NormalTok{(in.lake.rna }\OperatorTok{>}\StringTok{ }\DecValTok{0}\NormalTok{) }\OperatorTok{*}\StringTok{ }\DecValTok{1}

\CommentTok{# define the 'core' taxa as otus present in 80% of samples}
\NormalTok{in.lake.core <-}\StringTok{ }\NormalTok{w.dna[, }\KeywordTok{which}\NormalTok{((}\KeywordTok{colSums}\NormalTok{(in.lake.rna.pa) }\OperatorTok{/}\StringTok{ }\KeywordTok{nrow}\NormalTok{(in.lake.rna.pa)) }\OperatorTok{>=}\StringTok{ }\FloatTok{0.8}\NormalTok{)]}

\CommentTok{# of the core, how many are also in the soil samples?}
\NormalTok{in.lake.core.from.soils <-}\StringTok{ }\NormalTok{in.lake.core[, }\KeywordTok{intersect}\NormalTok{(}\KeywordTok{colnames}\NormalTok{(in.lake.core), }\KeywordTok{colnames}\NormalTok{(in.soil))]}

\CommentTok{# of the core which are not in the soil samples}
\NormalTok{in.lake.core.not.soils <-}\StringTok{ }\NormalTok{in.lake.core[, }\KeywordTok{setdiff}\NormalTok{(}\KeywordTok{colnames}\NormalTok{(in.lake.core), }\KeywordTok{colnames}\NormalTok{(in.soil))]}

\CommentTok{# Find the relative abundance of the core taxa and prepare data frame to plot}
\NormalTok{in.lake.core.from.soils.REL <-}\StringTok{ }\NormalTok{in.lake.core.from.soils }\OperatorTok{/}\StringTok{ }\KeywordTok{rowSums}\NormalTok{(w.dna)}

\NormalTok{in.soil.to.plot <-}\StringTok{ }\KeywordTok{as.data.frame}\NormalTok{(in.lake.core.from.soils.REL) }\OperatorTok\StringTok{ }
\StringTok{  }\KeywordTok{rownames_to_column}\NormalTok{(}\StringTok{"sample_ID"}\NormalTok{) }\OperatorTok\StringTok{ }
\StringTok{  }\KeywordTok{gather}\NormalTok{(otu_id, rel_abundance, }\OperatorTok{-}\NormalTok{sample_ID) }\OperatorTok\StringTok{ }
\StringTok{  }\KeywordTok{left_join}\NormalTok{(}\KeywordTok{rownames_to_column}\NormalTok{(design.dna, }\StringTok{"sample_ID"}\NormalTok{)) }\OperatorTok\StringTok{ }
\StringTok{  }\KeywordTok{add_column}\NormalTok{(}\DataTypeTok{found =} \StringTok{"soils"}\NormalTok{)}

\NormalTok{in.lake.core.not.soils.REL <-}\StringTok{ }\NormalTok{in.lake.core.not.soils }\OperatorTok{/}\StringTok{ }\KeywordTok{rowSums}\NormalTok{(w.dna)}

\NormalTok{in.lake.to.plot <-}\StringTok{ }\KeywordTok{as.data.frame}\NormalTok{(in.lake.core.not.soils.REL) }\OperatorTok\StringTok{ }
\StringTok{  }\KeywordTok{rownames_to_column}\NormalTok{(}\StringTok{"sample_ID"}\NormalTok{) }\OperatorTok\StringTok{ }
\StringTok{  }\KeywordTok{gather}\NormalTok{(otu_id, rel_abundance, }\OperatorTok{-}\NormalTok{sample_ID) }\OperatorTok\StringTok{ }
\StringTok{  }\KeywordTok{left_join}\NormalTok{(}\KeywordTok{rownames_to_column}\NormalTok{(design.dna, }\StringTok{"sample_ID"}\NormalTok{)) }\OperatorTok\StringTok{ }
\StringTok{  }\KeywordTok{add_column}\NormalTok{(}\DataTypeTok{found =} \StringTok{"lake"}\NormalTok{)}
\end{Highlighting}
\end{Shaded}

Now, lets plot the abundances of the OTUs across the reservoir and split
them up into whether they were recovered in soils or not.

\begin{Shaded}
\begin{Highlighting}[]
\KeywordTok{bind_rows}\NormalTok{(in.soil.to.plot, in.lake.to.plot) }\OperatorTok\StringTok{ }
\StringTok{  }\KeywordTok{ggplot}\NormalTok{(}\KeywordTok{aes}\NormalTok{(}\DataTypeTok{x =}\NormalTok{ distance, }\DataTypeTok{y =}\NormalTok{ rel_abundance, }\DataTypeTok{color =}\NormalTok{ otu_id)) }\OperatorTok{+}\StringTok{ }
\StringTok{  }\KeywordTok{labs}\NormalTok{(}\DataTypeTok{x =} \StringTok{"Reservoir Transect (m)"}\NormalTok{, }
       \DataTypeTok{y =} \StringTok{"OTU Relative Abundance"}\NormalTok{) }\OperatorTok{+}
\StringTok{  }\KeywordTok{geom_line}\NormalTok{(}\DataTypeTok{alpha =} \FloatTok{0.5}\NormalTok{, }\DataTypeTok{stat =} \StringTok{"smooth"}\NormalTok{, }\DataTypeTok{method =} \StringTok{"lm"}\NormalTok{, }\DataTypeTok{se =}\NormalTok{ F, }\DataTypeTok{show.legend =}\NormalTok{ F) }\OperatorTok{+}
\StringTok{  }\KeywordTok{scale_y_log10}\NormalTok{() }\OperatorTok{+}
\StringTok{  }\KeywordTok{scale_x_reverse}\NormalTok{() }\OperatorTok{+}\StringTok{ }
\StringTok{  }\KeywordTok{facet_wrap}\NormalTok{(}\OperatorTok{~}\StringTok{ }\NormalTok{found, }\DataTypeTok{ncol =} \DecValTok{1}\NormalTok{, }
             \DataTypeTok{labeller =} \KeywordTok{as_labeller}\NormalTok{(}\KeywordTok{c}\NormalTok{(}
               \StringTok{`}\DataTypeTok{lake}\StringTok{`}\NormalTok{ =}\StringTok{ "Undetected in soils"}\NormalTok{,}
               \StringTok{`}\DataTypeTok{soils}\StringTok{`}\NormalTok{ =}\StringTok{ "Present in soils"}\NormalTok{)))}
\end{Highlighting}
\end{Shaded}

\begin{verbatim}
## Warning: Transformation introduced infinite values in continuous y-axis
\end{verbatim}

\begin{verbatim}
## Warning: Removed 104 rows containing non-finite values (stat_smooth).
\end{verbatim}

\begin{center}\includegraphics{ReservoirGradient_files/figure-latex/coreplot-1} \end{center}

From this figure, we note a few important points. First, we observe that
core reservoir taxa that are not detected in the soil samples tend to
increase in relative abundance along the reservoir transect. We also
note that for the taxa that are present in the soil samples, some tend
to increase drastically, while others tend to increase, along the
transect. This suggests that there may be two classes of soil-derived
OTUs that contribute to reservoir bacterial diversity:\\
- taxa where the reservoir is a sink (i.e., maintained via mass effects
from the soils) - aquatic taxa seeded by populations stored in the soils

\begin{Shaded}
\begin{Highlighting}[]
\CommentTok{# model distance effect on rel abundance to get slope and pval}
\NormalTok{soil.core.mods <-}\StringTok{ }\KeywordTok{apply}\NormalTok{(in.lake.core.from.soils.REL, }\DataTypeTok{MARGIN =} \DecValTok{2}\NormalTok{, }
    \DataTypeTok{FUN =} \ControlFlowTok{function}\NormalTok{(x) }\KeywordTok{summary}\NormalTok{(}\KeywordTok{lm}\NormalTok{(x }\OperatorTok{~}\StringTok{ }\NormalTok{design.dna}\OperatorTok{$}\NormalTok{distance))}\OperatorTok{$}\NormalTok{coefficients[}\DecValTok{2}\NormalTok{,}\KeywordTok{c}\NormalTok{(}\DecValTok{1}\NormalTok{,}\DecValTok{4}\NormalTok{)])}
\KeywordTok{rownames}\NormalTok{(soil.core.mods) <-}\StringTok{ }\KeywordTok{c}\NormalTok{(}\StringTok{"slope"}\NormalTok{, }\StringTok{"pval"}\NormalTok{)}

\CommentTok{# classify otus as significantly increasing or decreasing along reservoir}
\NormalTok{soil.core.decreasing <-}\StringTok{ }\KeywordTok{as.data.frame}\NormalTok{(}\KeywordTok{t}\NormalTok{(soil.core.mods)) }\OperatorTok\StringTok{ }
\StringTok{  }\KeywordTok{rownames_to_column}\NormalTok{(}\StringTok{"OTU"}\NormalTok{) }\OperatorTok\StringTok{ }
\StringTok{  }\KeywordTok{filter}\NormalTok{(pval }\OperatorTok{<}\StringTok{ }\FloatTok{0.05} \OperatorTok{&}\StringTok{ }\NormalTok{slope }\OperatorTok{>}\StringTok{ }\DecValTok{0}\NormalTok{) }\OperatorTok\StringTok{   }\CommentTok{# rel abund decreases toward dam}
\StringTok{  }\KeywordTok{left_join}\NormalTok{(OTU.tax)}
\end{Highlighting}
\end{Shaded}

\begin{verbatim}
## Warning: Column `OTU` joining character vector and factor, coercing into
## character vector
\end{verbatim}

\begin{Shaded}
\begin{Highlighting}[]
\NormalTok{soil.core.increasing <-}\StringTok{ }\KeywordTok{as.data.frame}\NormalTok{(}\KeywordTok{t}\NormalTok{(soil.core.mods)) }\OperatorTok\StringTok{ }
\StringTok{  }\KeywordTok{rownames_to_column}\NormalTok{(}\StringTok{"OTU"}\NormalTok{) }\OperatorTok\StringTok{ }
\StringTok{  }\KeywordTok{filter}\NormalTok{(pval }\OperatorTok{<}\StringTok{ }\FloatTok{0.05} \OperatorTok{&}\StringTok{ }\NormalTok{slope }\OperatorTok{<}\StringTok{ }\DecValTok{0}\NormalTok{) }\OperatorTok\StringTok{   }\CommentTok{# rel abund increases toward dam}
\StringTok{  }\KeywordTok{left_join}\NormalTok{(OTU.tax)}
\end{Highlighting}
\end{Shaded}

\begin{verbatim}
## Warning: Column `OTU` joining character vector and factor, coercing into
## character vector
\end{verbatim}

\begin{Shaded}
\begin{Highlighting}[]
\NormalTok{nonsoil.core.mods <-}\StringTok{ }\KeywordTok{apply}\NormalTok{(in.lake.core.not.soils.REL, }\DataTypeTok{MARGIN =} \DecValTok{2}\NormalTok{, }
    \DataTypeTok{FUN =} \ControlFlowTok{function}\NormalTok{(x) }\KeywordTok{summary}\NormalTok{(}\KeywordTok{lm}\NormalTok{(x }\OperatorTok{~}\StringTok{ }\NormalTok{design.dna}\OperatorTok{$}\NormalTok{distance))}\OperatorTok{$}\NormalTok{coefficients[}\DecValTok{2}\NormalTok{,}\KeywordTok{c}\NormalTok{(}\DecValTok{1}\NormalTok{,}\DecValTok{4}\NormalTok{)])}
\KeywordTok{rownames}\NormalTok{(nonsoil.core.mods) <-}\StringTok{ }\KeywordTok{c}\NormalTok{(}\StringTok{"slope"}\NormalTok{, }\StringTok{"pval"}\NormalTok{)}
\NormalTok{nonsoil.core.decreasing <-}\StringTok{ }\KeywordTok{as.data.frame}\NormalTok{(}\KeywordTok{t}\NormalTok{(nonsoil.core.mods)) }\OperatorTok\StringTok{ }
\StringTok{  }\KeywordTok{rownames_to_column}\NormalTok{(}\StringTok{"OTU"}\NormalTok{) }\OperatorTok\StringTok{ }
\StringTok{  }\KeywordTok{filter}\NormalTok{(pval }\OperatorTok{<}\StringTok{ }\FloatTok{0.05} \OperatorTok{&}\StringTok{ }\NormalTok{slope }\OperatorTok{>}\StringTok{ }\DecValTok{0}\NormalTok{) }\OperatorTok\StringTok{   }\CommentTok{# rel abund decreases toward dam}
\StringTok{  }\KeywordTok{left_join}\NormalTok{(OTU.tax)}
\end{Highlighting}
\end{Shaded}

\begin{verbatim}
## Warning: Column `OTU` joining character vector and factor, coercing into
## character vector
\end{verbatim}

\begin{Shaded}
\begin{Highlighting}[]
\NormalTok{nonsoil.core.increasing <-}\StringTok{ }\KeywordTok{as.data.frame}\NormalTok{(}\KeywordTok{t}\NormalTok{(nonsoil.core.mods)) }\OperatorTok\StringTok{ }
\StringTok{  }\KeywordTok{rownames_to_column}\NormalTok{(}\StringTok{"OTU"}\NormalTok{) }\OperatorTok\StringTok{ }
\StringTok{  }\KeywordTok{filter}\NormalTok{(pval }\OperatorTok{<}\StringTok{ }\FloatTok{0.05} \OperatorTok{&}\StringTok{ }\NormalTok{slope }\OperatorTok{<}\StringTok{ }\DecValTok{0}\NormalTok{) }\OperatorTok\StringTok{   }\CommentTok{# rel abund increases toward dam}
\StringTok{  }\KeywordTok{left_join}\NormalTok{(OTU.tax)}
\end{Highlighting}
\end{Shaded}

\begin{verbatim}
## Warning: Column `OTU` joining character vector and factor, coercing into
## character vector
\end{verbatim}

Now we will visualize the significant taxa

\begin{Shaded}
\begin{Highlighting}[]
\KeywordTok{pander}\NormalTok{(nonsoil.core.decreasing, }\DataTypeTok{caption =} \StringTok{"Core taxa not found in soils that get rarer along the transect."}\NormalTok{)}
\end{Highlighting}
\end{Shaded}

\begin{longtable}[]{@{}cccccc@{}}
\caption{Core taxa not found in soils that get rarer along the transect.
(continued below)}\tabularnewline
\toprule
\begin{minipage}[t]{0.12\columnwidth}\centering\strut
\textbf{OTU}\strut
\end{minipage} & \begin{minipage}[t]{0.14\columnwidth}\centering\strut
\textbf{slope}\strut
\end{minipage} & \begin{minipage}[t]{0.13\columnwidth}\centering\strut
\textbf{pval}\strut
\end{minipage} & \begin{minipage}[t]{0.15\columnwidth}\centering\strut
\textbf{Domain}\strut
\end{minipage} & \begin{minipage}[t]{0.15\columnwidth}\centering\strut
\textbf{Phylum}\strut
\end{minipage} & \begin{minipage}[t]{0.15\columnwidth}\centering\strut
\textbf{Class}\strut
\end{minipage}\tabularnewline
\bottomrule
\end{longtable}

\begin{longtable}[]{@{}ccc@{}}
\toprule
\begin{minipage}[t]{0.15\columnwidth}\centering\strut
\textbf{Order}\strut
\end{minipage} & \begin{minipage}[t]{0.16\columnwidth}\centering\strut
\textbf{Family}\strut
\end{minipage} & \begin{minipage}[t]{0.16\columnwidth}\centering\strut
\textbf{Genus}\strut
\end{minipage}\tabularnewline
\bottomrule
\end{longtable}

\begin{Shaded}
\begin{Highlighting}[]
\KeywordTok{pander}\NormalTok{(nonsoil.core.increasing, }\DataTypeTok{caption =} \StringTok{"Core taxa not found in soils that get more common along the transect."}\NormalTok{)}
\end{Highlighting}
\end{Shaded}

\begin{longtable}[]{@{}ccccc@{}}
\caption{Core taxa not found in soils that get more common along the
transect. (continued below)}\tabularnewline
\toprule
\begin{minipage}[b]{0.13\columnwidth}\centering\strut
OTU\strut
\end{minipage} & \begin{minipage}[b]{0.16\columnwidth}\centering\strut
slope\strut
\end{minipage} & \begin{minipage}[b]{0.14\columnwidth}\centering\strut
pval\strut
\end{minipage} & \begin{minipage}[b]{0.13\columnwidth}\centering\strut
Domain\strut
\end{minipage} & \begin{minipage}[b]{0.27\columnwidth}\centering\strut
Phylum\strut
\end{minipage}\tabularnewline
\midrule
\endfirsthead
\toprule
\begin{minipage}[b]{0.13\columnwidth}\centering\strut
OTU\strut
\end{minipage} & \begin{minipage}[b]{0.16\columnwidth}\centering\strut
slope\strut
\end{minipage} & \begin{minipage}[b]{0.14\columnwidth}\centering\strut
pval\strut
\end{minipage} & \begin{minipage}[b]{0.13\columnwidth}\centering\strut
Domain\strut
\end{minipage} & \begin{minipage}[b]{0.27\columnwidth}\centering\strut
Phylum\strut
\end{minipage}\tabularnewline
\midrule
\endhead
\begin{minipage}[t]{0.13\columnwidth}\centering\strut
Otu00004\strut
\end{minipage} & \begin{minipage}[t]{0.16\columnwidth}\centering\strut
-0.0001388\strut
\end{minipage} & \begin{minipage}[t]{0.14\columnwidth}\centering\strut
2.314e-06\strut
\end{minipage} & \begin{minipage}[t]{0.13\columnwidth}\centering\strut
Bacteria\strut
\end{minipage} & \begin{minipage}[t]{0.27\columnwidth}\centering\strut
Actinobacteria\strut
\end{minipage}\tabularnewline
\begin{minipage}[t]{0.13\columnwidth}\centering\strut
Otu00038\strut
\end{minipage} & \begin{minipage}[t]{0.16\columnwidth}\centering\strut
-4.052e-05\strut
\end{minipage} & \begin{minipage}[t]{0.14\columnwidth}\centering\strut
0.0004807\strut
\end{minipage} & \begin{minipage}[t]{0.13\columnwidth}\centering\strut
Bacteria\strut
\end{minipage} & \begin{minipage}[t]{0.27\columnwidth}\centering\strut
Actinobacteria\strut
\end{minipage}\tabularnewline
\begin{minipage}[t]{0.13\columnwidth}\centering\strut
Otu00118\strut
\end{minipage} & \begin{minipage}[t]{0.16\columnwidth}\centering\strut
-6.994e-06\strut
\end{minipage} & \begin{minipage}[t]{0.14\columnwidth}\centering\strut
0.01551\strut
\end{minipage} & \begin{minipage}[t]{0.13\columnwidth}\centering\strut
Bacteria\strut
\end{minipage} & \begin{minipage}[t]{0.27\columnwidth}\centering\strut
Actinobacteria\strut
\end{minipage}\tabularnewline
\begin{minipage}[t]{0.13\columnwidth}\centering\strut
Otu00156\strut
\end{minipage} & \begin{minipage}[t]{0.16\columnwidth}\centering\strut
-9.037e-06\strut
\end{minipage} & \begin{minipage}[t]{0.14\columnwidth}\centering\strut
0.0002416\strut
\end{minipage} & \begin{minipage}[t]{0.13\columnwidth}\centering\strut
Bacteria\strut
\end{minipage} & \begin{minipage}[t]{0.27\columnwidth}\centering\strut
Bacteria\_unclassified\strut
\end{minipage}\tabularnewline
\bottomrule
\end{longtable}

\begin{longtable}[]{@{}cc@{}}
\caption{Table continues below}\tabularnewline
\toprule
\begin{minipage}[b]{0.31\columnwidth}\centering\strut
Class\strut
\end{minipage} & \begin{minipage}[b]{0.38\columnwidth}\centering\strut
Order\strut
\end{minipage}\tabularnewline
\midrule
\endfirsthead
\toprule
\begin{minipage}[b]{0.31\columnwidth}\centering\strut
Class\strut
\end{minipage} & \begin{minipage}[b]{0.38\columnwidth}\centering\strut
Order\strut
\end{minipage}\tabularnewline
\midrule
\endhead
\begin{minipage}[t]{0.31\columnwidth}\centering\strut
Actinobacteria\strut
\end{minipage} & \begin{minipage}[t]{0.38\columnwidth}\centering\strut
Actinomycetales\strut
\end{minipage}\tabularnewline
\begin{minipage}[t]{0.31\columnwidth}\centering\strut
Actinobacteria\strut
\end{minipage} & \begin{minipage}[t]{0.38\columnwidth}\centering\strut
Actinomycetales\strut
\end{minipage}\tabularnewline
\begin{minipage}[t]{0.31\columnwidth}\centering\strut
Actinobacteria\strut
\end{minipage} & \begin{minipage}[t]{0.38\columnwidth}\centering\strut
Actinobacteria\_unclassified\strut
\end{minipage}\tabularnewline
\begin{minipage}[t]{0.31\columnwidth}\centering\strut
Bacteria\_unclassified\strut
\end{minipage} & \begin{minipage}[t]{0.38\columnwidth}\centering\strut
Bacteria\_unclassified\strut
\end{minipage}\tabularnewline
\bottomrule
\end{longtable}

\begin{longtable}[]{@{}cc@{}}
\toprule
\begin{minipage}[b]{0.41\columnwidth}\centering\strut
Family\strut
\end{minipage} & \begin{minipage}[b]{0.41\columnwidth}\centering\strut
Genus\strut
\end{minipage}\tabularnewline
\midrule
\endhead
\begin{minipage}[t]{0.41\columnwidth}\centering\strut
Actinomycetales\_unclassified\strut
\end{minipage} & \begin{minipage}[t]{0.41\columnwidth}\centering\strut
Actinomycetales\_unclassified\strut
\end{minipage}\tabularnewline
\begin{minipage}[t]{0.41\columnwidth}\centering\strut
Actinomycetales\_unclassified\strut
\end{minipage} & \begin{minipage}[t]{0.41\columnwidth}\centering\strut
Actinomycetales\_unclassified\strut
\end{minipage}\tabularnewline
\begin{minipage}[t]{0.41\columnwidth}\centering\strut
Actinobacteria\_unclassified\strut
\end{minipage} & \begin{minipage}[t]{0.41\columnwidth}\centering\strut
Actinobacteria\_unclassified\strut
\end{minipage}\tabularnewline
\begin{minipage}[t]{0.41\columnwidth}\centering\strut
Bacteria\_unclassified\strut
\end{minipage} & \begin{minipage}[t]{0.41\columnwidth}\centering\strut
Bacteria\_unclassified\strut
\end{minipage}\tabularnewline
\bottomrule
\end{longtable}

\begin{Shaded}
\begin{Highlighting}[]
\KeywordTok{pander}\NormalTok{(soil.core.decreasing, }\DataTypeTok{caption =} \StringTok{"Core taxa found in soils that get rarer along the transect."}\NormalTok{)}
\end{Highlighting}
\end{Shaded}

\begin{longtable}[]{@{}ccccc@{}}
\caption{Core taxa found in soils that get rarer along the transect.
(continued below)}\tabularnewline
\toprule
\begin{minipage}[b]{0.13\columnwidth}\centering\strut
OTU\strut
\end{minipage} & \begin{minipage}[b]{0.14\columnwidth}\centering\strut
slope\strut
\end{minipage} & \begin{minipage}[b]{0.12\columnwidth}\centering\strut
pval\strut
\end{minipage} & \begin{minipage}[b]{0.13\columnwidth}\centering\strut
Domain\strut
\end{minipage} & \begin{minipage}[b]{0.19\columnwidth}\centering\strut
Phylum\strut
\end{minipage}\tabularnewline
\midrule
\endfirsthead
\toprule
\begin{minipage}[b]{0.13\columnwidth}\centering\strut
OTU\strut
\end{minipage} & \begin{minipage}[b]{0.14\columnwidth}\centering\strut
slope\strut
\end{minipage} & \begin{minipage}[b]{0.12\columnwidth}\centering\strut
pval\strut
\end{minipage} & \begin{minipage}[b]{0.13\columnwidth}\centering\strut
Domain\strut
\end{minipage} & \begin{minipage}[b]{0.19\columnwidth}\centering\strut
Phylum\strut
\end{minipage}\tabularnewline
\midrule
\endhead
\begin{minipage}[t]{0.13\columnwidth}\centering\strut
Otu00018\strut
\end{minipage} & \begin{minipage}[t]{0.14\columnwidth}\centering\strut
4.776e-05\strut
\end{minipage} & \begin{minipage}[t]{0.12\columnwidth}\centering\strut
0.02307\strut
\end{minipage} & \begin{minipage}[t]{0.13\columnwidth}\centering\strut
Bacteria\strut
\end{minipage} & \begin{minipage}[t]{0.19\columnwidth}\centering\strut
Proteobacteria\strut
\end{minipage}\tabularnewline
\begin{minipage}[t]{0.13\columnwidth}\centering\strut
Otu00057\strut
\end{minipage} & \begin{minipage}[t]{0.14\columnwidth}\centering\strut
2.416e-05\strut
\end{minipage} & \begin{minipage}[t]{0.12\columnwidth}\centering\strut
0.03413\strut
\end{minipage} & \begin{minipage}[t]{0.13\columnwidth}\centering\strut
Bacteria\strut
\end{minipage} & \begin{minipage}[t]{0.19\columnwidth}\centering\strut
Proteobacteria\strut
\end{minipage}\tabularnewline
\begin{minipage}[t]{0.13\columnwidth}\centering\strut
Otu00077\strut
\end{minipage} & \begin{minipage}[t]{0.14\columnwidth}\centering\strut
5.224e-05\strut
\end{minipage} & \begin{minipage}[t]{0.12\columnwidth}\centering\strut
0.04372\strut
\end{minipage} & \begin{minipage}[t]{0.13\columnwidth}\centering\strut
Bacteria\strut
\end{minipage} & \begin{minipage}[t]{0.19\columnwidth}\centering\strut
Bacteroidetes\strut
\end{minipage}\tabularnewline
\begin{minipage}[t]{0.13\columnwidth}\centering\strut
Otu00081\strut
\end{minipage} & \begin{minipage}[t]{0.14\columnwidth}\centering\strut
2.098e-05\strut
\end{minipage} & \begin{minipage}[t]{0.12\columnwidth}\centering\strut
0.04653\strut
\end{minipage} & \begin{minipage}[t]{0.13\columnwidth}\centering\strut
Bacteria\strut
\end{minipage} & \begin{minipage}[t]{0.19\columnwidth}\centering\strut
Proteobacteria\strut
\end{minipage}\tabularnewline
\begin{minipage}[t]{0.13\columnwidth}\centering\strut
Otu00138\strut
\end{minipage} & \begin{minipage}[t]{0.14\columnwidth}\centering\strut
3.148e-05\strut
\end{minipage} & \begin{minipage}[t]{0.12\columnwidth}\centering\strut
0.04819\strut
\end{minipage} & \begin{minipage}[t]{0.13\columnwidth}\centering\strut
Bacteria\strut
\end{minipage} & \begin{minipage}[t]{0.19\columnwidth}\centering\strut
Firmicutes\strut
\end{minipage}\tabularnewline
\begin{minipage}[t]{0.13\columnwidth}\centering\strut
Otu00260\strut
\end{minipage} & \begin{minipage}[t]{0.14\columnwidth}\centering\strut
9.499e-06\strut
\end{minipage} & \begin{minipage}[t]{0.12\columnwidth}\centering\strut
0.04541\strut
\end{minipage} & \begin{minipage}[t]{0.13\columnwidth}\centering\strut
Bacteria\strut
\end{minipage} & \begin{minipage}[t]{0.19\columnwidth}\centering\strut
Proteobacteria\strut
\end{minipage}\tabularnewline
\begin{minipage}[t]{0.13\columnwidth}\centering\strut
Otu00545\strut
\end{minipage} & \begin{minipage}[t]{0.14\columnwidth}\centering\strut
2.148e-06\strut
\end{minipage} & \begin{minipage}[t]{0.12\columnwidth}\centering\strut
0.04557\strut
\end{minipage} & \begin{minipage}[t]{0.13\columnwidth}\centering\strut
Bacteria\strut
\end{minipage} & \begin{minipage}[t]{0.19\columnwidth}\centering\strut
Actinobacteria\strut
\end{minipage}\tabularnewline
\bottomrule
\end{longtable}

\begin{longtable}[]{@{}ccc@{}}
\caption{Table continues below}\tabularnewline
\toprule
\begin{minipage}[b]{0.28\columnwidth}\centering\strut
Class\strut
\end{minipage} & \begin{minipage}[b]{0.28\columnwidth}\centering\strut
Order\strut
\end{minipage} & \begin{minipage}[b]{0.28\columnwidth}\centering\strut
Family\strut
\end{minipage}\tabularnewline
\midrule
\endfirsthead
\toprule
\begin{minipage}[b]{0.28\columnwidth}\centering\strut
Class\strut
\end{minipage} & \begin{minipage}[b]{0.28\columnwidth}\centering\strut
Order\strut
\end{minipage} & \begin{minipage}[b]{0.28\columnwidth}\centering\strut
Family\strut
\end{minipage}\tabularnewline
\midrule
\endhead
\begin{minipage}[t]{0.28\columnwidth}\centering\strut
Gammaproteobacteria\strut
\end{minipage} & \begin{minipage}[t]{0.28\columnwidth}\centering\strut
Pseudomonadales\strut
\end{minipage} & \begin{minipage}[t]{0.28\columnwidth}\centering\strut
Pseudomonadaceae\strut
\end{minipage}\tabularnewline
\begin{minipage}[t]{0.28\columnwidth}\centering\strut
Gammaproteobacteria\strut
\end{minipage} & \begin{minipage}[t]{0.28\columnwidth}\centering\strut
Methylococcales\strut
\end{minipage} & \begin{minipage}[t]{0.28\columnwidth}\centering\strut
Methylococcaceae\strut
\end{minipage}\tabularnewline
\begin{minipage}[t]{0.28\columnwidth}\centering\strut
Flavobacteriia\strut
\end{minipage} & \begin{minipage}[t]{0.28\columnwidth}\centering\strut
Flavobacteriales\strut
\end{minipage} & \begin{minipage}[t]{0.28\columnwidth}\centering\strut
Flavobacteriaceae\strut
\end{minipage}\tabularnewline
\begin{minipage}[t]{0.28\columnwidth}\centering\strut
Betaproteobacteria\strut
\end{minipage} & \begin{minipage}[t]{0.28\columnwidth}\centering\strut
Burkholderiales\strut
\end{minipage} & \begin{minipage}[t]{0.28\columnwidth}\centering\strut
Oxalobacteraceae\strut
\end{minipage}\tabularnewline
\begin{minipage}[t]{0.28\columnwidth}\centering\strut
Bacilli\strut
\end{minipage} & \begin{minipage}[t]{0.28\columnwidth}\centering\strut
Bacillales\strut
\end{minipage} & \begin{minipage}[t]{0.28\columnwidth}\centering\strut
Bacillaceae\_1\strut
\end{minipage}\tabularnewline
\begin{minipage}[t]{0.28\columnwidth}\centering\strut
Gammaproteobacteria\strut
\end{minipage} & \begin{minipage}[t]{0.28\columnwidth}\centering\strut
Enterobacteriales\strut
\end{minipage} & \begin{minipage}[t]{0.28\columnwidth}\centering\strut
Enterobacteriaceae\strut
\end{minipage}\tabularnewline
\begin{minipage}[t]{0.28\columnwidth}\centering\strut
Actinobacteria\strut
\end{minipage} & \begin{minipage}[t]{0.28\columnwidth}\centering\strut
Solirubrobacterales\strut
\end{minipage} & \begin{minipage}[t]{0.28\columnwidth}\centering\strut
Solirubrobacteraceae\strut
\end{minipage}\tabularnewline
\bottomrule
\end{longtable}

\begin{longtable}[]{@{}c@{}}
\toprule
\begin{minipage}[b]{0.42\columnwidth}\centering\strut
Genus\strut
\end{minipage}\tabularnewline
\midrule
\endhead
\begin{minipage}[t]{0.42\columnwidth}\centering\strut
Pseudomonas\strut
\end{minipage}\tabularnewline
\begin{minipage}[t]{0.42\columnwidth}\centering\strut
Methylococcaceae\_unclassified\strut
\end{minipage}\tabularnewline
\begin{minipage}[t]{0.42\columnwidth}\centering\strut
Flavobacterium\strut
\end{minipage}\tabularnewline
\begin{minipage}[t]{0.42\columnwidth}\centering\strut
Janthinobacterium\strut
\end{minipage}\tabularnewline
\begin{minipage}[t]{0.42\columnwidth}\centering\strut
Bacillus\strut
\end{minipage}\tabularnewline
\begin{minipage}[t]{0.42\columnwidth}\centering\strut
Yersinia\strut
\end{minipage}\tabularnewline
\begin{minipage}[t]{0.42\columnwidth}\centering\strut
Solirubrobacter\strut
\end{minipage}\tabularnewline
\bottomrule
\end{longtable}

\begin{Shaded}
\begin{Highlighting}[]
\KeywordTok{pander}\NormalTok{(soil.core.increasing, }\DataTypeTok{caption =} \StringTok{"Core taxa found in soils that get more common along the transect."}\NormalTok{)}
\end{Highlighting}
\end{Shaded}

\begin{longtable}[]{@{}ccccc@{}}
\caption{Core taxa found in soils that get more common along the
transect. (continued below)}\tabularnewline
\toprule
\begin{minipage}[b]{0.13\columnwidth}\centering\strut
OTU\strut
\end{minipage} & \begin{minipage}[b]{0.16\columnwidth}\centering\strut
slope\strut
\end{minipage} & \begin{minipage}[b]{0.14\columnwidth}\centering\strut
pval\strut
\end{minipage} & \begin{minipage}[b]{0.13\columnwidth}\centering\strut
Domain\strut
\end{minipage} & \begin{minipage}[b]{0.20\columnwidth}\centering\strut
Phylum\strut
\end{minipage}\tabularnewline
\midrule
\endfirsthead
\toprule
\begin{minipage}[b]{0.13\columnwidth}\centering\strut
OTU\strut
\end{minipage} & \begin{minipage}[b]{0.16\columnwidth}\centering\strut
slope\strut
\end{minipage} & \begin{minipage}[b]{0.14\columnwidth}\centering\strut
pval\strut
\end{minipage} & \begin{minipage}[b]{0.13\columnwidth}\centering\strut
Domain\strut
\end{minipage} & \begin{minipage}[b]{0.20\columnwidth}\centering\strut
Phylum\strut
\end{minipage}\tabularnewline
\midrule
\endhead
\begin{minipage}[t]{0.13\columnwidth}\centering\strut
Otu00001\strut
\end{minipage} & \begin{minipage}[t]{0.16\columnwidth}\centering\strut
-2.271e-05\strut
\end{minipage} & \begin{minipage}[t]{0.14\columnwidth}\centering\strut
0.02914\strut
\end{minipage} & \begin{minipage}[t]{0.13\columnwidth}\centering\strut
Bacteria\strut
\end{minipage} & \begin{minipage}[t]{0.20\columnwidth}\centering\strut
Proteobacteria\strut
\end{minipage}\tabularnewline
\begin{minipage}[t]{0.13\columnwidth}\centering\strut
Otu00002\strut
\end{minipage} & \begin{minipage}[t]{0.16\columnwidth}\centering\strut
-0.0002387\strut
\end{minipage} & \begin{minipage}[t]{0.14\columnwidth}\centering\strut
0.0004954\strut
\end{minipage} & \begin{minipage}[t]{0.13\columnwidth}\centering\strut
Bacteria\strut
\end{minipage} & \begin{minipage}[t]{0.20\columnwidth}\centering\strut
Actinobacteria\strut
\end{minipage}\tabularnewline
\begin{minipage}[t]{0.13\columnwidth}\centering\strut
Otu00003\strut
\end{minipage} & \begin{minipage}[t]{0.16\columnwidth}\centering\strut
-0.0001089\strut
\end{minipage} & \begin{minipage}[t]{0.14\columnwidth}\centering\strut
0.0004266\strut
\end{minipage} & \begin{minipage}[t]{0.13\columnwidth}\centering\strut
Bacteria\strut
\end{minipage} & \begin{minipage}[t]{0.20\columnwidth}\centering\strut
Verrucomicrobia\strut
\end{minipage}\tabularnewline
\begin{minipage}[t]{0.13\columnwidth}\centering\strut
Otu00005\strut
\end{minipage} & \begin{minipage}[t]{0.16\columnwidth}\centering\strut
-5.373e-05\strut
\end{minipage} & \begin{minipage}[t]{0.14\columnwidth}\centering\strut
0.001508\strut
\end{minipage} & \begin{minipage}[t]{0.13\columnwidth}\centering\strut
Bacteria\strut
\end{minipage} & \begin{minipage}[t]{0.20\columnwidth}\centering\strut
Bacteroidetes\strut
\end{minipage}\tabularnewline
\begin{minipage}[t]{0.13\columnwidth}\centering\strut
Otu00006\strut
\end{minipage} & \begin{minipage}[t]{0.16\columnwidth}\centering\strut
-9.301e-06\strut
\end{minipage} & \begin{minipage}[t]{0.14\columnwidth}\centering\strut
0.02246\strut
\end{minipage} & \begin{minipage}[t]{0.13\columnwidth}\centering\strut
Bacteria\strut
\end{minipage} & \begin{minipage}[t]{0.20\columnwidth}\centering\strut
Bacteroidetes\strut
\end{minipage}\tabularnewline
\begin{minipage}[t]{0.13\columnwidth}\centering\strut
Otu00008\strut
\end{minipage} & \begin{minipage}[t]{0.16\columnwidth}\centering\strut
-4.31e-05\strut
\end{minipage} & \begin{minipage}[t]{0.14\columnwidth}\centering\strut
0.00383\strut
\end{minipage} & \begin{minipage}[t]{0.13\columnwidth}\centering\strut
Bacteria\strut
\end{minipage} & \begin{minipage}[t]{0.20\columnwidth}\centering\strut
Actinobacteria\strut
\end{minipage}\tabularnewline
\begin{minipage}[t]{0.13\columnwidth}\centering\strut
Otu00014\strut
\end{minipage} & \begin{minipage}[t]{0.16\columnwidth}\centering\strut
-0.0001028\strut
\end{minipage} & \begin{minipage}[t]{0.14\columnwidth}\centering\strut
0.000182\strut
\end{minipage} & \begin{minipage}[t]{0.13\columnwidth}\centering\strut
Bacteria\strut
\end{minipage} & \begin{minipage}[t]{0.20\columnwidth}\centering\strut
Actinobacteria\strut
\end{minipage}\tabularnewline
\begin{minipage}[t]{0.13\columnwidth}\centering\strut
Otu00015\strut
\end{minipage} & \begin{minipage}[t]{0.16\columnwidth}\centering\strut
-0.0001462\strut
\end{minipage} & \begin{minipage}[t]{0.14\columnwidth}\centering\strut
4.754e-05\strut
\end{minipage} & \begin{minipage}[t]{0.13\columnwidth}\centering\strut
Bacteria\strut
\end{minipage} & \begin{minipage}[t]{0.20\columnwidth}\centering\strut
Actinobacteria\strut
\end{minipage}\tabularnewline
\begin{minipage}[t]{0.13\columnwidth}\centering\strut
Otu00016\strut
\end{minipage} & \begin{minipage}[t]{0.16\columnwidth}\centering\strut
-5.778e-05\strut
\end{minipage} & \begin{minipage}[t]{0.14\columnwidth}\centering\strut
0.0002333\strut
\end{minipage} & \begin{minipage}[t]{0.13\columnwidth}\centering\strut
Bacteria\strut
\end{minipage} & \begin{minipage}[t]{0.20\columnwidth}\centering\strut
Actinobacteria\strut
\end{minipage}\tabularnewline
\begin{minipage}[t]{0.13\columnwidth}\centering\strut
Otu00017\strut
\end{minipage} & \begin{minipage}[t]{0.16\columnwidth}\centering\strut
-3.333e-05\strut
\end{minipage} & \begin{minipage}[t]{0.14\columnwidth}\centering\strut
2.84e-05\strut
\end{minipage} & \begin{minipage}[t]{0.13\columnwidth}\centering\strut
Bacteria\strut
\end{minipage} & \begin{minipage}[t]{0.20\columnwidth}\centering\strut
Actinobacteria\strut
\end{minipage}\tabularnewline
\begin{minipage}[t]{0.13\columnwidth}\centering\strut
Otu00025\strut
\end{minipage} & \begin{minipage}[t]{0.16\columnwidth}\centering\strut
-5.211e-05\strut
\end{minipage} & \begin{minipage}[t]{0.14\columnwidth}\centering\strut
0.0005492\strut
\end{minipage} & \begin{minipage}[t]{0.13\columnwidth}\centering\strut
Bacteria\strut
\end{minipage} & \begin{minipage}[t]{0.20\columnwidth}\centering\strut
Actinobacteria\strut
\end{minipage}\tabularnewline
\begin{minipage}[t]{0.13\columnwidth}\centering\strut
Otu00031\strut
\end{minipage} & \begin{minipage}[t]{0.16\columnwidth}\centering\strut
-5.931e-05\strut
\end{minipage} & \begin{minipage}[t]{0.14\columnwidth}\centering\strut
0.0002164\strut
\end{minipage} & \begin{minipage}[t]{0.13\columnwidth}\centering\strut
Bacteria\strut
\end{minipage} & \begin{minipage}[t]{0.20\columnwidth}\centering\strut
Bacteroidetes\strut
\end{minipage}\tabularnewline
\begin{minipage}[t]{0.13\columnwidth}\centering\strut
Otu00040\strut
\end{minipage} & \begin{minipage}[t]{0.16\columnwidth}\centering\strut
-3.772e-05\strut
\end{minipage} & \begin{minipage}[t]{0.14\columnwidth}\centering\strut
8.166e-06\strut
\end{minipage} & \begin{minipage}[t]{0.13\columnwidth}\centering\strut
Bacteria\strut
\end{minipage} & \begin{minipage}[t]{0.20\columnwidth}\centering\strut
Proteobacteria\strut
\end{minipage}\tabularnewline
\begin{minipage}[t]{0.13\columnwidth}\centering\strut
Otu00080\strut
\end{minipage} & \begin{minipage}[t]{0.16\columnwidth}\centering\strut
-2.2e-05\strut
\end{minipage} & \begin{minipage}[t]{0.14\columnwidth}\centering\strut
0.03337\strut
\end{minipage} & \begin{minipage}[t]{0.13\columnwidth}\centering\strut
Bacteria\strut
\end{minipage} & \begin{minipage}[t]{0.20\columnwidth}\centering\strut
Bacteroidetes\strut
\end{minipage}\tabularnewline
\bottomrule
\end{longtable}

\begin{longtable}[]{@{}cc@{}}
\caption{Table continues below}\tabularnewline
\toprule
\begin{minipage}[b]{0.29\columnwidth}\centering\strut
Class\strut
\end{minipage} & \begin{minipage}[b]{0.38\columnwidth}\centering\strut
Order\strut
\end{minipage}\tabularnewline
\midrule
\endfirsthead
\toprule
\begin{minipage}[b]{0.29\columnwidth}\centering\strut
Class\strut
\end{minipage} & \begin{minipage}[b]{0.38\columnwidth}\centering\strut
Order\strut
\end{minipage}\tabularnewline
\midrule
\endhead
\begin{minipage}[t]{0.29\columnwidth}\centering\strut
Betaproteobacteria\strut
\end{minipage} & \begin{minipage}[t]{0.38\columnwidth}\centering\strut
Burkholderiales\strut
\end{minipage}\tabularnewline
\begin{minipage}[t]{0.29\columnwidth}\centering\strut
Actinobacteria\strut
\end{minipage} & \begin{minipage}[t]{0.38\columnwidth}\centering\strut
Actinomycetales\strut
\end{minipage}\tabularnewline
\begin{minipage}[t]{0.29\columnwidth}\centering\strut
Spartobacteria\strut
\end{minipage} & \begin{minipage}[t]{0.38\columnwidth}\centering\strut
Spartobacteria\_unclassified\strut
\end{minipage}\tabularnewline
\begin{minipage}[t]{0.29\columnwidth}\centering\strut
Sphingobacteriia\strut
\end{minipage} & \begin{minipage}[t]{0.38\columnwidth}\centering\strut
Sphingobacteriales\strut
\end{minipage}\tabularnewline
\begin{minipage}[t]{0.29\columnwidth}\centering\strut
Sphingobacteriia\strut
\end{minipage} & \begin{minipage}[t]{0.38\columnwidth}\centering\strut
Sphingobacteriales\strut
\end{minipage}\tabularnewline
\begin{minipage}[t]{0.29\columnwidth}\centering\strut
Actinobacteria\strut
\end{minipage} & \begin{minipage}[t]{0.38\columnwidth}\centering\strut
Actinomycetales\strut
\end{minipage}\tabularnewline
\begin{minipage}[t]{0.29\columnwidth}\centering\strut
Actinobacteria\strut
\end{minipage} & \begin{minipage}[t]{0.38\columnwidth}\centering\strut
Actinomycetales\strut
\end{minipage}\tabularnewline
\begin{minipage}[t]{0.29\columnwidth}\centering\strut
Actinobacteria\strut
\end{minipage} & \begin{minipage}[t]{0.38\columnwidth}\centering\strut
Actinobacteria\_unclassified\strut
\end{minipage}\tabularnewline
\begin{minipage}[t]{0.29\columnwidth}\centering\strut
Actinobacteria\strut
\end{minipage} & \begin{minipage}[t]{0.38\columnwidth}\centering\strut
Actinomycetales\strut
\end{minipage}\tabularnewline
\begin{minipage}[t]{0.29\columnwidth}\centering\strut
Actinobacteria\strut
\end{minipage} & \begin{minipage}[t]{0.38\columnwidth}\centering\strut
Actinomycetales\strut
\end{minipage}\tabularnewline
\begin{minipage}[t]{0.29\columnwidth}\centering\strut
Actinobacteria\strut
\end{minipage} & \begin{minipage}[t]{0.38\columnwidth}\centering\strut
Actinomycetales\strut
\end{minipage}\tabularnewline
\begin{minipage}[t]{0.29\columnwidth}\centering\strut
Cytophagia\strut
\end{minipage} & \begin{minipage}[t]{0.38\columnwidth}\centering\strut
Cytophagales\strut
\end{minipage}\tabularnewline
\begin{minipage}[t]{0.29\columnwidth}\centering\strut
Alphaproteobacteria\strut
\end{minipage} & \begin{minipage}[t]{0.38\columnwidth}\centering\strut
Rhodospirillales\strut
\end{minipage}\tabularnewline
\begin{minipage}[t]{0.29\columnwidth}\centering\strut
Flavobacteriia\strut
\end{minipage} & \begin{minipage}[t]{0.38\columnwidth}\centering\strut
Flavobacteriales\strut
\end{minipage}\tabularnewline
\bottomrule
\end{longtable}

\begin{longtable}[]{@{}cc@{}}
\toprule
\begin{minipage}[b]{0.41\columnwidth}\centering\strut
Family\strut
\end{minipage} & \begin{minipage}[b]{0.42\columnwidth}\centering\strut
Genus\strut
\end{minipage}\tabularnewline
\midrule
\endhead
\begin{minipage}[t]{0.41\columnwidth}\centering\strut
Comamonadaceae\strut
\end{minipage} & \begin{minipage}[t]{0.42\columnwidth}\centering\strut
Comamonadaceae\_unclassified\strut
\end{minipage}\tabularnewline
\begin{minipage}[t]{0.41\columnwidth}\centering\strut
Actinomycetales\_unclassified\strut
\end{minipage} & \begin{minipage}[t]{0.42\columnwidth}\centering\strut
Actinomycetales\_unclassified\strut
\end{minipage}\tabularnewline
\begin{minipage}[t]{0.41\columnwidth}\centering\strut
Spartobacteria\_unclassified\strut
\end{minipage} & \begin{minipage}[t]{0.42\columnwidth}\centering\strut
Spartobacteria\_unclassified\strut
\end{minipage}\tabularnewline
\begin{minipage}[t]{0.41\columnwidth}\centering\strut
Chitinophagaceae\strut
\end{minipage} & \begin{minipage}[t]{0.42\columnwidth}\centering\strut
Sediminibacterium\strut
\end{minipage}\tabularnewline
\begin{minipage}[t]{0.41\columnwidth}\centering\strut
Saprospiraceae\strut
\end{minipage} & \begin{minipage}[t]{0.42\columnwidth}\centering\strut
Saprospiraceae\_unclassified\strut
\end{minipage}\tabularnewline
\begin{minipage}[t]{0.41\columnwidth}\centering\strut
Actinomycetales\_unclassified\strut
\end{minipage} & \begin{minipage}[t]{0.42\columnwidth}\centering\strut
Actinomycetales\_unclassified\strut
\end{minipage}\tabularnewline
\begin{minipage}[t]{0.41\columnwidth}\centering\strut
Actinomycetales\_unclassified\strut
\end{minipage} & \begin{minipage}[t]{0.42\columnwidth}\centering\strut
Actinomycetales\_unclassified\strut
\end{minipage}\tabularnewline
\begin{minipage}[t]{0.41\columnwidth}\centering\strut
Actinobacteria\_unclassified\strut
\end{minipage} & \begin{minipage}[t]{0.42\columnwidth}\centering\strut
Actinobacteria\_unclassified\strut
\end{minipage}\tabularnewline
\begin{minipage}[t]{0.41\columnwidth}\centering\strut
Microbacteriaceae\strut
\end{minipage} & \begin{minipage}[t]{0.42\columnwidth}\centering\strut
Microbacteriaceae\_unclassified\strut
\end{minipage}\tabularnewline
\begin{minipage}[t]{0.41\columnwidth}\centering\strut
Actinomycetales\_unclassified\strut
\end{minipage} & \begin{minipage}[t]{0.42\columnwidth}\centering\strut
Actinomycetales\_unclassified\strut
\end{minipage}\tabularnewline
\begin{minipage}[t]{0.41\columnwidth}\centering\strut
Microbacteriaceae\strut
\end{minipage} & \begin{minipage}[t]{0.42\columnwidth}\centering\strut
Microbacteriaceae\_unclassified\strut
\end{minipage}\tabularnewline
\begin{minipage}[t]{0.41\columnwidth}\centering\strut
Cyclobacteriaceae\strut
\end{minipage} & \begin{minipage}[t]{0.42\columnwidth}\centering\strut
Algoriphagus\strut
\end{minipage}\tabularnewline
\begin{minipage}[t]{0.41\columnwidth}\centering\strut
Acetobacteraceae\strut
\end{minipage} & \begin{minipage}[t]{0.42\columnwidth}\centering\strut
Roseomonas\strut
\end{minipage}\tabularnewline
\begin{minipage}[t]{0.41\columnwidth}\centering\strut
Flavobacteriaceae\strut
\end{minipage} & \begin{minipage}[t]{0.42\columnwidth}\centering\strut
Flavobacterium\strut
\end{minipage}\tabularnewline
\bottomrule
\end{longtable}

\begin{Shaded}
\begin{Highlighting}[]
\KeywordTok{as.data.frame}\NormalTok{(OTUsREL[,nonsoil.core.increasing}\OperatorTok{$}\NormalTok{OTU]) }\OperatorTok\StringTok{ }
\StringTok{  }\KeywordTok{rownames_to_column}\NormalTok{(}\StringTok{"sampleID"}\NormalTok{) }\OperatorTok\StringTok{ }
\StringTok{  }\KeywordTok{left_join}\NormalTok{(}\KeywordTok{rownames_to_column}\NormalTok{(design, }\StringTok{"sampleID"}\NormalTok{)) }\OperatorTok\StringTok{ }
\StringTok{  }\KeywordTok{gather}\NormalTok{(OTU, rel_abund, }\OperatorTok{-}\NormalTok{station, }\OperatorTok{-}\NormalTok{molecule, }\OperatorTok{-}\NormalTok{type, }\OperatorTok{-}\NormalTok{distance, }\OperatorTok{-}\NormalTok{sampleID) }\OperatorTok\StringTok{ }
\StringTok{  }\KeywordTok{filter}\NormalTok{(molecule }\OperatorTok{==}\StringTok{ "DNA"}\NormalTok{) }\OperatorTok\StringTok{ }\KeywordTok{left_join}\NormalTok{(OTU.tax) }\OperatorTok\StringTok{ }
\StringTok{  }\KeywordTok{mutate}\NormalTok{(}\DataTypeTok{taxon =} \KeywordTok{paste}\NormalTok{(Class, Order)) }\OperatorTok\StringTok{ }
\StringTok{  }\KeywordTok{ggplot}\NormalTok{(}\KeywordTok{aes}\NormalTok{(}\DataTypeTok{x =}\NormalTok{ distance, }\DataTypeTok{y =}\NormalTok{ rel_abund, }\DataTypeTok{color =}\NormalTok{ taxon)) }\OperatorTok{+}\StringTok{ }
\StringTok{  }\KeywordTok{geom_point}\NormalTok{(}\DataTypeTok{alpha =} \FloatTok{0.5}\NormalTok{) }\OperatorTok{+}\StringTok{ }
\StringTok{  }\KeywordTok{geom_smooth}\NormalTok{(}\DataTypeTok{method =} \StringTok{"loess"}\NormalTok{, }\DataTypeTok{se =} \OtherTok{FALSE}\NormalTok{) }\OperatorTok{+}\StringTok{ }
\StringTok{  }\KeywordTok{scale_x_reverse}\NormalTok{() }\OperatorTok{+}
\StringTok{  }\KeywordTok{theme}\NormalTok{(}\DataTypeTok{legend.position =} \StringTok{"bottom"}\NormalTok{) }\OperatorTok{+}
\StringTok{  }\KeywordTok{guides}\NormalTok{(}\DataTypeTok{color =} \KeywordTok{guide_legend}\NormalTok{(}\DataTypeTok{ncol =} \DecValTok{1}\NormalTok{)) }\OperatorTok{+}
\StringTok{  }\KeywordTok{labs}\NormalTok{(}\DataTypeTok{x =} \StringTok{"Reservoir Transect (m)"}\NormalTok{,}
       \DataTypeTok{y =} \StringTok{"Relative Abundance"}\NormalTok{)}
\end{Highlighting}
\end{Shaded}

\begin{verbatim}
## Warning: Column `OTU` joining character vector and factor, coercing into
## character vector
\end{verbatim}

\begin{verbatim}
## Warning: Removed 12 rows containing non-finite values (stat_smooth).
\end{verbatim}

\begin{verbatim}
## Warning: Removed 12 rows containing missing values (geom_point).
\end{verbatim}

\begin{center}\includegraphics{ReservoirGradient_files/figure-latex/sig_taxa-1} \end{center}

\begin{Shaded}
\begin{Highlighting}[]
\KeywordTok{as.data.frame}\NormalTok{(OTUsREL[,soil.core.increasing}\OperatorTok{$}\NormalTok{OTU]) }\OperatorTok\StringTok{ }
\StringTok{  }\KeywordTok{rownames_to_column}\NormalTok{(}\StringTok{"sampleID"}\NormalTok{) }\OperatorTok\StringTok{ }
\StringTok{  }\KeywordTok{left_join}\NormalTok{(}\KeywordTok{rownames_to_column}\NormalTok{(design, }\StringTok{"sampleID"}\NormalTok{)) }\OperatorTok\StringTok{ }
\StringTok{  }\KeywordTok{gather}\NormalTok{(OTU, rel_abund, }\OperatorTok{-}\NormalTok{station, }\OperatorTok{-}\NormalTok{molecule, }\OperatorTok{-}\NormalTok{type, }\OperatorTok{-}\NormalTok{distance, }\OperatorTok{-}\NormalTok{sampleID) }\OperatorTok\StringTok{ }
\StringTok{  }\KeywordTok{filter}\NormalTok{(molecule }\OperatorTok{==}\StringTok{ "DNA"}\NormalTok{) }\OperatorTok\StringTok{ }\KeywordTok{left_join}\NormalTok{(OTU.tax) }\OperatorTok\StringTok{ }
\StringTok{  }\KeywordTok{mutate}\NormalTok{(}\DataTypeTok{taxon =} \KeywordTok{paste}\NormalTok{(Class, Order)) }\OperatorTok\StringTok{ }
\StringTok{  }\KeywordTok{ggplot}\NormalTok{(}\KeywordTok{aes}\NormalTok{(}\DataTypeTok{x =}\NormalTok{ distance, }\DataTypeTok{y =}\NormalTok{ rel_abund, }\DataTypeTok{color =}\NormalTok{ taxon)) }\OperatorTok{+}\StringTok{ }
\StringTok{  }\KeywordTok{geom_point}\NormalTok{(}\DataTypeTok{alpha =} \FloatTok{0.5}\NormalTok{) }\OperatorTok{+}\StringTok{ }
\StringTok{  }\KeywordTok{geom_smooth}\NormalTok{(}\DataTypeTok{method =} \StringTok{"loess"}\NormalTok{, }\DataTypeTok{se =} \OtherTok{FALSE}\NormalTok{) }\OperatorTok{+}
\StringTok{  }\KeywordTok{scale_x_reverse}\NormalTok{() }\OperatorTok{+}
\StringTok{  }\KeywordTok{theme}\NormalTok{(}\DataTypeTok{legend.position =} \StringTok{"bottom"}\NormalTok{) }\OperatorTok{+}
\StringTok{  }\KeywordTok{guides}\NormalTok{(}\DataTypeTok{color =} \KeywordTok{guide_legend}\NormalTok{(}\DataTypeTok{ncol =} \DecValTok{1}\NormalTok{)) }\OperatorTok{+}
\StringTok{  }\KeywordTok{labs}\NormalTok{(}\DataTypeTok{x =} \StringTok{"Reservoir Transect (m)"}\NormalTok{,}
       \DataTypeTok{y =} \StringTok{"Relative Abundance"}\NormalTok{)}
\end{Highlighting}
\end{Shaded}

\begin{verbatim}
## Warning: Column `OTU` joining character vector and factor, coercing into
## character vector
\end{verbatim}

\begin{verbatim}
## Warning: Removed 42 rows containing non-finite values (stat_smooth).
\end{verbatim}

\begin{verbatim}
## Warning: Removed 42 rows containing missing values (geom_point).
\end{verbatim}

\begin{center}\includegraphics{ReservoirGradient_files/figure-latex/sig_taxa-2} \end{center}

\begin{Shaded}
\begin{Highlighting}[]
\KeywordTok{as.data.frame}\NormalTok{(OTUsREL[,soil.core.decreasing}\OperatorTok{$}\NormalTok{OTU]) }\OperatorTok\StringTok{ }
\StringTok{  }\KeywordTok{rownames_to_column}\NormalTok{(}\StringTok{"sampleID"}\NormalTok{) }\OperatorTok\StringTok{ }
\StringTok{  }\KeywordTok{left_join}\NormalTok{(}\KeywordTok{rownames_to_column}\NormalTok{(design, }\StringTok{"sampleID"}\NormalTok{)) }\OperatorTok\StringTok{ }
\StringTok{  }\KeywordTok{gather}\NormalTok{(OTU, rel_abund, }\OperatorTok{-}\NormalTok{station, }\OperatorTok{-}\NormalTok{molecule, }\OperatorTok{-}\NormalTok{type, }\OperatorTok{-}\NormalTok{distance, }\OperatorTok{-}\NormalTok{sampleID) }\OperatorTok\StringTok{ }
\StringTok{  }\KeywordTok{filter}\NormalTok{(molecule }\OperatorTok{==}\StringTok{ "DNA"}\NormalTok{) }\OperatorTok\StringTok{ }\KeywordTok{left_join}\NormalTok{(OTU.tax) }\OperatorTok\StringTok{ }
\StringTok{  }\KeywordTok{mutate}\NormalTok{(}\DataTypeTok{taxon =} \KeywordTok{paste}\NormalTok{(Class, Order)) }\OperatorTok\StringTok{ }
\StringTok{  }\KeywordTok{ggplot}\NormalTok{(}\KeywordTok{aes}\NormalTok{(}\DataTypeTok{x =}\NormalTok{ distance, }\DataTypeTok{y =}\NormalTok{ rel_abund, }\DataTypeTok{color =}\NormalTok{ taxon)) }\OperatorTok{+}\StringTok{ }
\StringTok{  }\KeywordTok{geom_point}\NormalTok{(}\DataTypeTok{alpha =} \FloatTok{0.5}\NormalTok{) }\OperatorTok{+}\StringTok{ }
\StringTok{  }\KeywordTok{geom_smooth}\NormalTok{(}\DataTypeTok{method =} \StringTok{"loess"}\NormalTok{, }\DataTypeTok{se =} \OtherTok{FALSE}\NormalTok{) }\OperatorTok{+}\StringTok{ }
\StringTok{  }\KeywordTok{scale_x_reverse}\NormalTok{() }\OperatorTok{+}
\StringTok{  }\KeywordTok{theme}\NormalTok{(}\DataTypeTok{legend.position =} \StringTok{"bottom"}\NormalTok{) }\OperatorTok{+}
\StringTok{  }\KeywordTok{guides}\NormalTok{(}\DataTypeTok{color =} \KeywordTok{guide_legend}\NormalTok{(}\DataTypeTok{ncol =} \DecValTok{1}\NormalTok{)) }\OperatorTok{+}
\StringTok{  }\KeywordTok{labs}\NormalTok{(}\DataTypeTok{x =} \StringTok{"Reservoir Transect (m)"}\NormalTok{,}
       \DataTypeTok{y =} \StringTok{"Relative Abundance"}\NormalTok{)}
\end{Highlighting}
\end{Shaded}

\begin{verbatim}
## Warning: Column `OTU` joining character vector and factor, coercing into
## character vector
\end{verbatim}

\begin{verbatim}
## Warning: Removed 21 rows containing non-finite values (stat_smooth).
\end{verbatim}

\begin{verbatim}
## Warning: Removed 21 rows containing missing values (geom_point).
\end{verbatim}

\begin{center}\includegraphics{ReservoirGradient_files/figure-latex/sig_taxa-3} \end{center}

\begin{Shaded}
\begin{Highlighting}[]
\CommentTok{# how much do the different core components contribute to total abundances}
\NormalTok{in.lake.core.soil.REL <-}\StringTok{ }\KeywordTok{rowSums}\NormalTok{(in.lake.core.from.soils) }\OperatorTok{/}\StringTok{ }\KeywordTok{rowSums}\NormalTok{(w.dna)}
\NormalTok{in.lake.core.water.REL <-}\StringTok{ }\KeywordTok{rowSums}\NormalTok{(in.lake.core.not.soils) }\OperatorTok{/}\StringTok{ }\KeywordTok{rowSums}\NormalTok{(w.dna)}
\end{Highlighting}
\end{Shaded}

\subsection{Taxonomic Analysis}\label{taxonomic-analysis}

\subsection{Comparisons of relabunds}\label{comparisons-of-relabunds}

Now, lets see which taxa increase or decrease substantially along the
gradient. I calculated the fold change in relative abundance of all
these taxa along the gradient relative to their max abundance in soils.
Thus, the OTUs that are most abundant near the soils will have a
declining slope toward the dam. The OTUs that are perhaps seeded from
the soils into the lake will have an increasing slope toward the dam.

\begin{Shaded}
\begin{Highlighting}[]
\NormalTok{high.activity.soil.core <-}\StringTok{ }\KeywordTok{as.data.frame}\NormalTok{(core.soil.taxa.DNA.REL.bounds) }\OperatorTok
\StringTok{  }\KeywordTok{rownames_to_column}\NormalTok{(}\StringTok{"OTU"}\NormalTok{) }\OperatorTok\StringTok{ }
\StringTok{  }\KeywordTok{filter}\NormalTok{(RNA.max }\OperatorTok{>}\StringTok{ }\DecValTok{0}\NormalTok{) }\OperatorTok\StringTok{ }\KeywordTok{arrange}\NormalTok{(}\KeywordTok{desc}\NormalTok{(RNA.max)) }\OperatorTok\StringTok{ }
\StringTok{  }\KeywordTok{left_join}\NormalTok{(OTU.tax)}
\end{Highlighting}
\end{Shaded}

\begin{verbatim}
## Warning: Column `OTU` joining character vector and factor, coercing into
## character vector
\end{verbatim}

\begin{Shaded}
\begin{Highlighting}[]
\NormalTok{high.activity.water.core <-}\StringTok{ }\KeywordTok{as.data.frame}\NormalTok{(core.water.taxa.DNA.REL.bounds) }\OperatorTok
\StringTok{  }\KeywordTok{rownames_to_column}\NormalTok{(}\StringTok{"OTU"}\NormalTok{) }\OperatorTok\StringTok{ }
\StringTok{  }\KeywordTok{filter}\NormalTok{(RNA.max }\OperatorTok{>}\StringTok{ }\DecValTok{0}\NormalTok{) }\OperatorTok\StringTok{ }\KeywordTok{arrange}\NormalTok{(}\KeywordTok{desc}\NormalTok{(RNA.max)) }\OperatorTok\StringTok{ }
\StringTok{  }\KeywordTok{left_join}\NormalTok{(OTU.tax)}
\end{Highlighting}
\end{Shaded}

\begin{verbatim}
## Warning: Column `OTU` joining character vector and factor, coercing into
## character vector
\end{verbatim}

\begin{Shaded}
\begin{Highlighting}[]
\NormalTok{mean.soil.abunds.soil.core <-}\StringTok{ }\NormalTok{OTUsREL[}\KeywordTok{which}\NormalTok{(design}\OperatorTok{$}\NormalTok{type }\OperatorTok{==}\StringTok{ "soil"}\NormalTok{), high.activity.soil.core}\OperatorTok{$}\NormalTok{OTU] }\OperatorTok\StringTok{ }
\StringTok{  }\NormalTok{colMeans }\OperatorTok\StringTok{ }\KeywordTok{data.frame}\NormalTok{(}\DataTypeTok{mean_soil_relabund =}\NormalTok{ .) }\OperatorTok\StringTok{ }
\StringTok{  }\KeywordTok{rownames_to_column}\NormalTok{(}\StringTok{"OTU"}\NormalTok{) }\OperatorTok\StringTok{ }\KeywordTok{arrange}\NormalTok{(}\KeywordTok{desc}\NormalTok{(mean_soil_relabund))}
\NormalTok{max.soil.abunds.soil.core <-}\StringTok{ }\NormalTok{OTUsREL[}\KeywordTok{which}\NormalTok{(design}\OperatorTok{$}\NormalTok{type }\OperatorTok{==}\StringTok{ "soil"}\NormalTok{), high.activity.soil.core}\OperatorTok{$}\NormalTok{OTU] }\OperatorTok\StringTok{ }
\StringTok{  }\KeywordTok{apply}\NormalTok{(}\DataTypeTok{X =}\NormalTok{ ., }\DataTypeTok{MARGIN =} \DecValTok{2}\NormalTok{, max) }\OperatorTok\StringTok{ }\KeywordTok{data.frame}\NormalTok{(}\DataTypeTok{max_soil_relabund =}\NormalTok{ .) }\OperatorTok\StringTok{ }
\StringTok{  }\KeywordTok{rownames_to_column}\NormalTok{(}\StringTok{"OTU"}\NormalTok{) }\OperatorTok\StringTok{ }\KeywordTok{arrange}\NormalTok{(}\KeywordTok{desc}\NormalTok{(max_soil_relabund))}

\NormalTok{mean.soil.abunds.water.core <-}\StringTok{ }\NormalTok{OTUsREL[}\KeywordTok{which}\NormalTok{(design}\OperatorTok{$}\NormalTok{type }\OperatorTok{==}\StringTok{ "soil"}\NormalTok{), high.activity.water.core}\OperatorTok{$}\NormalTok{OTU] }\OperatorTok\StringTok{ }
\StringTok{  }\NormalTok{colMeans }\OperatorTok\StringTok{ }\KeywordTok{data.frame}\NormalTok{(}\DataTypeTok{mean_soil_relabund =}\NormalTok{ .) }\OperatorTok\StringTok{ }
\StringTok{  }\KeywordTok{rownames_to_column}\NormalTok{(}\StringTok{"OTU"}\NormalTok{) }\OperatorTok\StringTok{ }\KeywordTok{arrange}\NormalTok{(}\KeywordTok{desc}\NormalTok{(mean_soil_relabund))}
\NormalTok{max.soil.abunds.water.core <-}\StringTok{ }\NormalTok{OTUsREL[}\KeywordTok{which}\NormalTok{(design}\OperatorTok{$}\NormalTok{type }\OperatorTok{==}\StringTok{ "soil"}\NormalTok{), high.activity.water.core}\OperatorTok{$}\NormalTok{OTU] }\OperatorTok\StringTok{ }
\StringTok{  }\KeywordTok{apply}\NormalTok{(}\DataTypeTok{X =}\NormalTok{ ., }\DataTypeTok{MARGIN =} \DecValTok{2}\NormalTok{, max) }\OperatorTok\StringTok{ }\KeywordTok{data.frame}\NormalTok{(}\DataTypeTok{max_soil_relabund =}\NormalTok{ .) }\OperatorTok\StringTok{ }
\StringTok{  }\KeywordTok{rownames_to_column}\NormalTok{(}\StringTok{"OTU"}\NormalTok{) }\OperatorTok\StringTok{ }\KeywordTok{arrange}\NormalTok{(}\KeywordTok{desc}\NormalTok{(max_soil_relabund))}


\NormalTok{soil.vs.lake.abunds <-}\StringTok{ }\NormalTok{high.activity.soil.core }\OperatorTok\StringTok{ }
\StringTok{  }\KeywordTok{left_join}\NormalTok{(mean.soil.abunds.soil.core) }\OperatorTok\StringTok{ }\KeywordTok{left_join}\NormalTok{(max.soil.abunds.soil.core) }\OperatorTok\StringTok{ }
\StringTok{  }\KeywordTok{mutate}\NormalTok{(}\DataTypeTok{soil_is_source =} \KeywordTok{ifelse}\NormalTok{(max_soil_relabund }\OperatorTok{>}\StringTok{ }\FloatTok{1e-3} \OperatorTok{&}\StringTok{ }\NormalTok{RNA.max }\OperatorTok{>}\StringTok{ }\FloatTok{1e-3}\NormalTok{, T, F)) }\OperatorTok\StringTok{ }
\StringTok{  }\KeywordTok{mutate}\NormalTok{(}\DataTypeTok{Taxon =} \KeywordTok{ifelse}\NormalTok{(Genus }\OperatorTok{==}\StringTok{ "unclassified"}\NormalTok{, }\KeywordTok{paste}\NormalTok{(Family, }\StringTok{"sp."}\NormalTok{), Genus))}

\NormalTok{combined.relabunds <-}\StringTok{ }\NormalTok{max.soil.abunds.soil.core }\OperatorTok
\StringTok{  }\KeywordTok{left_join}\NormalTok{(}\KeywordTok{rownames_to_column}\NormalTok{(}\KeywordTok{as.data.frame}\NormalTok{(}\KeywordTok{t}\NormalTok{(in.lake.core.from.soils.REL)), }\StringTok{"OTU"}\NormalTok{))}
\KeywordTok{rownames}\NormalTok{(combined.relabunds) <-}\StringTok{ }\NormalTok{combined.relabunds}\OperatorTok{$}\NormalTok{OTU}
\NormalTok{combined.relabunds <-}\StringTok{ }\NormalTok{combined.relabunds[,}\OperatorTok{-}\DecValTok{1}\NormalTok{]}

\NormalTok{otus.fold.change <-}\StringTok{ }\KeywordTok{na.omit}\NormalTok{(combined.relabunds }\OperatorTok{/}\StringTok{ }\NormalTok{combined.relabunds}\OperatorTok{$}\NormalTok{max_soil_relabund) }\CommentTok{# Calculate fold changes}

\NormalTok{fold_change_summary <-}\StringTok{ }\NormalTok{otus.fold.change }\OperatorTok\StringTok{ }\KeywordTok{rownames_to_column}\NormalTok{(}\StringTok{"OTU"}\NormalTok{) }\OperatorTok\StringTok{ }
\StringTok{  }\KeywordTok{select}\NormalTok{(}\OperatorTok{-}\NormalTok{max_soil_relabund) }\OperatorTok\StringTok{ }
\StringTok{  }\KeywordTok{gather}\NormalTok{(}\StringTok{"sample"}\NormalTok{, }\StringTok{"fold_change"}\NormalTok{, }\OperatorTok{-}\NormalTok{OTU) }\OperatorTok\StringTok{ }
\StringTok{  }\KeywordTok{left_join}\NormalTok{(}\KeywordTok{select}\NormalTok{(}\KeywordTok{rownames_to_column}\NormalTok{(design.dna, }\StringTok{"sample"}\NormalTok{), }\OperatorTok{-}\NormalTok{station, }\OperatorTok{-}\NormalTok{molecule, }\OperatorTok{-}\NormalTok{type)) }\OperatorTok\StringTok{ }
\StringTok{  }\KeywordTok{group_by}\NormalTok{(OTU) }\OperatorTok\StringTok{ }
\StringTok{  }\KeywordTok{summarize}\NormalTok{(}\DataTypeTok{max_change =} \KeywordTok{max}\NormalTok{(fold_change), }\DataTypeTok{min_change =} \KeywordTok{min}\NormalTok{(fold_change))}

\NormalTok{otus.fold.change }\OperatorTok\StringTok{ }\KeywordTok{rownames_to_column}\NormalTok{(}\StringTok{"OTU"}\NormalTok{) }\OperatorTok\StringTok{ }
\StringTok{  }\KeywordTok{select}\NormalTok{(}\OperatorTok{-}\NormalTok{max_soil_relabund) }\OperatorTok\StringTok{ }
\StringTok{  }\KeywordTok{gather}\NormalTok{(}\StringTok{"sample"}\NormalTok{, }\StringTok{"fold_change"}\NormalTok{, }\OperatorTok{-}\NormalTok{OTU) }\OperatorTok\StringTok{ }
\StringTok{  }\KeywordTok{left_join}\NormalTok{(}\KeywordTok{select}\NormalTok{(}\KeywordTok{rownames_to_column}\NormalTok{(design.dna, }\StringTok{"sample"}\NormalTok{), }\OperatorTok{-}\NormalTok{station, }\OperatorTok{-}\NormalTok{molecule, }\OperatorTok{-}\NormalTok{type)) }\OperatorTok\StringTok{ }
\StringTok{  }\KeywordTok{ggplot}\NormalTok{(}\KeywordTok{aes}\NormalTok{(}\DataTypeTok{x =}\NormalTok{ distance, }\DataTypeTok{y =}\NormalTok{ fold_change, }\DataTypeTok{color =}\NormalTok{ OTU)) }\OperatorTok{+}\StringTok{ }
\StringTok{  }\KeywordTok{geom_hline}\NormalTok{(}\KeywordTok{aes}\NormalTok{(}\DataTypeTok{yintercept =} \DecValTok{1}\NormalTok{), }\DataTypeTok{color =} \StringTok{"gray50"}\NormalTok{, }\DataTypeTok{alpha =} \FloatTok{0.5}\NormalTok{, }\DataTypeTok{size =} \DecValTok{2}\NormalTok{) }\OperatorTok{+}
\StringTok{  }\KeywordTok{geom_jitter}\NormalTok{(}\DataTypeTok{alpha =} \FloatTok{0.05}\NormalTok{) }\OperatorTok{+}\StringTok{ }
\StringTok{  }\KeywordTok{geom_smooth}\NormalTok{(}\DataTypeTok{alpha =} \FloatTok{0.5}\NormalTok{, }\DataTypeTok{method =} \StringTok{"lm"}\NormalTok{, }\DataTypeTok{se =}\NormalTok{ F) }\OperatorTok{+}\StringTok{ }
\StringTok{  }\KeywordTok{scale_y_log10}\NormalTok{(}\DataTypeTok{labels =}\NormalTok{ scales}\OperatorTok{::}\NormalTok{comma) }\OperatorTok{+}
\StringTok{  }\KeywordTok{scale_x_reverse}\NormalTok{() }\OperatorTok{+}
\StringTok{  }\KeywordTok{annotation_logticks}\NormalTok{(}\DataTypeTok{long =} \KeywordTok{unit}\NormalTok{(.}\DecValTok{1}\NormalTok{, }\StringTok{"in"}\NormalTok{), }\DataTypeTok{sides =} \StringTok{"l"}\NormalTok{) }\OperatorTok{+}
\StringTok{  }\KeywordTok{theme}\NormalTok{(}\DataTypeTok{legend.position =} \StringTok{"none"}\NormalTok{) }\OperatorTok{+}
\StringTok{  }\KeywordTok{labs}\NormalTok{(}\DataTypeTok{x =} \StringTok{"Reservoir Transect (m)"}\NormalTok{, }\DataTypeTok{y =} \StringTok{"Fold-change in abundance"}\NormalTok{)}
\end{Highlighting}
\end{Shaded}

\begin{verbatim}
## Warning: Transformation introduced infinite values in continuous y-axis
\end{verbatim}

\begin{verbatim}
## Warning: Transformation introduced infinite values in continuous y-axis
\end{verbatim}

\begin{verbatim}
## Warning: Removed 49 rows containing non-finite values (stat_smooth).
\end{verbatim}

\begin{center}\includegraphics{ReservoirGradient_files/figure-latex/fold_change_plot-1} \end{center}

\begin{Shaded}
\begin{Highlighting}[]
\CommentTok{# otus.fold.change %>% rownames_to_column("OTU") %>% }
\CommentTok{#   select(-max_soil_relabund) %>% }
\CommentTok{#   gather("sample", "fold_change", -OTU) %>% }
\CommentTok{#   left_join(select(rownames_to_column(design.dna, "sample"), -station, -molecule, -type))}

\NormalTok{foldchanges <-}\StringTok{ }\KeywordTok{t}\NormalTok{(otus.fold.change)[}\OperatorTok{-}\DecValTok{1}\NormalTok{,]}
\NormalTok{foldchangelms <-}\StringTok{ }\KeywordTok{apply}\NormalTok{(foldchanges, }\DataTypeTok{MARGIN =} \DecValTok{2}\NormalTok{, }
    \DataTypeTok{FUN =} \ControlFlowTok{function}\NormalTok{(x) }\KeywordTok{summary}\NormalTok{(}\KeywordTok{lm}\NormalTok{(x }\OperatorTok{~}\StringTok{ }\NormalTok{design.dna}\OperatorTok{$}\NormalTok{distance))}\OperatorTok{$}\NormalTok{coefficients[}\KeywordTok{c}\NormalTok{(}\DecValTok{1}\NormalTok{,}\DecValTok{2}\NormalTok{,}\DecValTok{8}\NormalTok{)])}
\KeywordTok{rownames}\NormalTok{(foldchangelms) <-}\StringTok{ }\KeywordTok{c}\NormalTok{(}\StringTok{"intercept"}\NormalTok{, }\StringTok{"slope"}\NormalTok{, }\StringTok{"pval"}\NormalTok{)}

\NormalTok{soil.core.decresing <-}\StringTok{ }\KeywordTok{as.data.frame}\NormalTok{(}\KeywordTok{t}\NormalTok{(foldchangelms)) }\OperatorTok\StringTok{ }
\StringTok{  }\KeywordTok{rownames_to_column}\NormalTok{(}\StringTok{"OTU"}\NormalTok{) }\OperatorTok\StringTok{ }
\StringTok{  }\KeywordTok{filter}\NormalTok{( slope }\OperatorTok{>}\StringTok{ }\DecValTok{0}\NormalTok{) }\OperatorTok\StringTok{   }\CommentTok{# rel abund decreases toward dam}
\StringTok{  }\KeywordTok{left_join}\NormalTok{(OTU.tax) }\OperatorTok\StringTok{ }\KeywordTok{select}\NormalTok{(}\OperatorTok{-}\NormalTok{intercept, }\OperatorTok{-}\NormalTok{slope, }\OperatorTok{-}\NormalTok{pval, }\KeywordTok{everything}\NormalTok{()) }\OperatorTok\StringTok{ }
\StringTok{  }\KeywordTok{arrange}\NormalTok{(}\KeywordTok{desc}\NormalTok{(slope))}
\end{Highlighting}
\end{Shaded}

\begin{verbatim}
## Warning: Column `OTU` joining character vector and factor, coercing into
## character vector
\end{verbatim}

\begin{Shaded}
\begin{Highlighting}[]
\NormalTok{soil.core.increasing <-}\StringTok{ }\KeywordTok{as.data.frame}\NormalTok{(}\KeywordTok{t}\NormalTok{(foldchangelms)) }\OperatorTok\StringTok{ }
\StringTok{  }\KeywordTok{rownames_to_column}\NormalTok{(}\StringTok{"OTU"}\NormalTok{) }\OperatorTok\StringTok{ }
\StringTok{  }\KeywordTok{filter}\NormalTok{( slope }\OperatorTok{<}\StringTok{ }\DecValTok{0}\NormalTok{) }\OperatorTok\StringTok{   }\CommentTok{# rel abund increases toward dam}
\StringTok{  }\KeywordTok{left_join}\NormalTok{(OTU.tax) }\OperatorTok\StringTok{ }\KeywordTok{select}\NormalTok{(}\OperatorTok{-}\NormalTok{intercept, }\OperatorTok{-}\NormalTok{slope, }\OperatorTok{-}\NormalTok{pval, }\KeywordTok{everything}\NormalTok{()) }\OperatorTok\StringTok{ }
\StringTok{  }\KeywordTok{arrange}\NormalTok{((slope))}
\end{Highlighting}
\end{Shaded}

\begin{verbatim}
## Warning: Column `OTU` joining character vector and factor, coercing into
## character vector
\end{verbatim}

\begin{Shaded}
\begin{Highlighting}[]
\NormalTok{soil.decrease.tab <-}\StringTok{ }\NormalTok{soil.core.decresing }\OperatorTok\StringTok{ }\KeywordTok{select}\NormalTok{(}\OperatorTok{-}\NormalTok{OTU, }\OperatorTok{-}\NormalTok{Domain) }\OperatorTok\StringTok{  }\KeywordTok{flextable}\NormalTok{()}
\NormalTok{soil.increase.tab <-}\StringTok{ }\NormalTok{soil.core.increasing }\OperatorTok\StringTok{ }\KeywordTok{select}\NormalTok{(}\OperatorTok{-}\NormalTok{OTU, }\OperatorTok{-}\NormalTok{Domain) }\OperatorTok\StringTok{ }\KeywordTok{flextable}\NormalTok{()}

\KeywordTok{read_docx}\NormalTok{() }\OperatorTok\StringTok{ }
\StringTok{  }\KeywordTok{body_end_section_continuous}\NormalTok{() }\OperatorTok\StringTok{ }
\StringTok{  }\KeywordTok{body_add_par}\NormalTok{(}\StringTok{"Increasing away from stream inlet"}\NormalTok{, }\DataTypeTok{style =} \StringTok{"heading 2"}\NormalTok{) }\OperatorTok\StringTok{ }
\StringTok{  }\KeywordTok{body_add_flextable}\NormalTok{(soil.increase.tab) }\OperatorTok\StringTok{ }
\StringTok{  }\KeywordTok{body_add_par}\NormalTok{(}\StringTok{"Decreasing away from stream inlet"}\NormalTok{, }\DataTypeTok{style =} \StringTok{"heading 2"}\NormalTok{) }\OperatorTok\StringTok{ }
\StringTok{  }\KeywordTok{body_add_flextable}\NormalTok{(soil.decrease.tab) }\OperatorTok\StringTok{ }
\StringTok{  }\KeywordTok{body_end_section_landscape}\NormalTok{() }\OperatorTok\StringTok{ }
\StringTok{  }\KeywordTok{print}\NormalTok{(}\DataTypeTok{target =} \StringTok{"tables/soil-core-change-tables.docx"}\NormalTok{)}
\end{Highlighting}
\end{Shaded}

\begin{verbatim}
## [1] "/Users/nawis/GitHub/ReservoirGradient/tables/soil-core-change-tables.docx"
\end{verbatim}

\subsection{Word Table}\label{word-table}

\begin{Shaded}
\begin{Highlighting}[]
\NormalTok{soil.tab <-}\StringTok{ }\NormalTok{core.soil }\OperatorTok\StringTok{ }\KeywordTok{arrange}\NormalTok{(}\KeywordTok{desc}\NormalTok{(RNA.max)) }\OperatorTok\StringTok{ }\KeywordTok{flextable}\NormalTok{() }\OperatorTok\StringTok{ }\KeywordTok{autofit}\NormalTok{()}
\NormalTok{water.tab <-}\StringTok{ }\NormalTok{core.water }\OperatorTok\StringTok{ }\KeywordTok{arrange}\NormalTok{(}\KeywordTok{desc}\NormalTok{(RNA.max)) }\OperatorTok\StringTok{ }\KeywordTok{flextable}\NormalTok{() }\OperatorTok\StringTok{ }\KeywordTok{autofit}\NormalTok{()}

\KeywordTok{read_docx}\NormalTok{() }\OperatorTok\StringTok{ }
\StringTok{  }\KeywordTok{body_add_par}\NormalTok{(}\StringTok{"Table S1"}\NormalTok{, }\DataTypeTok{style =} \StringTok{"heading 1"}\NormalTok{) }\OperatorTok\StringTok{ }
\StringTok{  }\KeywordTok{body_end_section_continuous}\NormalTok{() }\OperatorTok\StringTok{ }
\StringTok{  }\KeywordTok{body_add_par}\NormalTok{(}\StringTok{"Core Reservoir Microbiome (present in soils)"}\NormalTok{, }\DataTypeTok{style =} \StringTok{"heading 2"}\NormalTok{) }\OperatorTok\StringTok{ }
\StringTok{  }\KeywordTok{body_add_flextable}\NormalTok{(soil.tab) }\OperatorTok\StringTok{ }
\StringTok{  }\KeywordTok{body_add_par}\NormalTok{(}\StringTok{"Core Reservoir Microbiome (absent from soils)"}\NormalTok{, }\DataTypeTok{style =} \StringTok{"heading 2"}\NormalTok{) }\OperatorTok\StringTok{ }
\StringTok{  }\KeywordTok{body_add_flextable}\NormalTok{(water.tab) }\OperatorTok\StringTok{ }
\StringTok{  }\KeywordTok{body_end_section_landscape}\NormalTok{() }\OperatorTok\StringTok{ }
\StringTok{  }\KeywordTok{print}\NormalTok{(}\DataTypeTok{target =} \StringTok{"tables/core_tables.docx"}\NormalTok{)}
\end{Highlighting}
\end{Shaded}

\begin{verbatim}
## [1] "/Users/nawis/GitHub/ReservoirGradient/tables/core_tables.docx"
\end{verbatim}

\subsection{Soil vs.~Lake Comparisons}\label{soil-vs.lake-comparisons}

\begin{Shaded}
\begin{Highlighting}[]
\NormalTok{soil.vs.lake.abunds }\OperatorTok\StringTok{ }
\StringTok{  }\KeywordTok{mutate}\NormalTok{(}\DataTypeTok{Genus =} \KeywordTok{str_replace}\NormalTok{(Genus, }\StringTok{"_unclassified"}\NormalTok{, }\StringTok{" sp."}\NormalTok{)) }\OperatorTok\StringTok{ }
\StringTok{  }\KeywordTok{filter}\NormalTok{(max_soil_relabund }\OperatorTok{>}\StringTok{ }\DecValTok{0}\NormalTok{) }\OperatorTok\StringTok{ }
\StringTok{  }\KeywordTok{ggplot}\NormalTok{(}\KeywordTok{aes}\NormalTok{(}\DataTypeTok{x =}\NormalTok{ max_soil_relabund, }\DataTypeTok{y =}\NormalTok{ RNA.max)) }\OperatorTok{+}
\StringTok{  }\KeywordTok{geom_vline}\NormalTok{(}\DataTypeTok{xintercept =} \FloatTok{1e-3}\NormalTok{, }\DataTypeTok{alpha =} \FloatTok{0.1}\NormalTok{) }\OperatorTok{+}
\StringTok{  }\KeywordTok{geom_hline}\NormalTok{(}\DataTypeTok{yintercept =} \FloatTok{1e-3}\NormalTok{, }\DataTypeTok{alpha =} \FloatTok{0.1}\NormalTok{) }\OperatorTok{+}
\StringTok{  }\KeywordTok{geom_jitter}\NormalTok{(}\DataTypeTok{size =} \DecValTok{3}\NormalTok{, }\DataTypeTok{alpha =} \FloatTok{0.5}\NormalTok{, }\DataTypeTok{show.legend =}\NormalTok{ F) }\OperatorTok{+}
\StringTok{  }\KeywordTok{scale_x_log10}\NormalTok{(}\DataTypeTok{lim =} \KeywordTok{c}\NormalTok{(}\FloatTok{1e-6}\NormalTok{, }\FloatTok{1e-2}\NormalTok{)) }\OperatorTok{+}\StringTok{ }
\StringTok{  }\KeywordTok{scale_y_log10}\NormalTok{(}\DataTypeTok{lim =} \KeywordTok{c}\NormalTok{(}\FloatTok{1e-5}\NormalTok{, }\DecValTok{1}\NormalTok{)) }\OperatorTok{+}
\StringTok{  }\KeywordTok{annotation_logticks}\NormalTok{(}\DataTypeTok{long =} \KeywordTok{unit}\NormalTok{(.}\DecValTok{1}\NormalTok{, }\StringTok{"in"}\NormalTok{)) }\OperatorTok{+}
\StringTok{  }\KeywordTok{scale_color_manual}\NormalTok{(}\DataTypeTok{values =}\NormalTok{ my.cols) }\OperatorTok{+}
\StringTok{  }\KeywordTok{labs}\NormalTok{(}\DataTypeTok{x =} \StringTok{"Max Soil Relative Abundance"}\NormalTok{, }\DataTypeTok{y =} \StringTok{"Max Lake RNA }\CharTok{\textbackslash{}n}\StringTok{Relative Abundance"}\NormalTok{) }\OperatorTok{+}
\StringTok{  }\KeywordTok{geom_text_repel}\NormalTok{(}\DataTypeTok{size =} \FloatTok{4.5}\NormalTok{, }\KeywordTok{aes}\NormalTok{(}\DataTypeTok{label =}\NormalTok{ Genus), }\DataTypeTok{force =} \FloatTok{1.5}\NormalTok{, }\DataTypeTok{alpha =} \FloatTok{0.9}\NormalTok{, }\DataTypeTok{segment.alpha =} \FloatTok{0.8}\NormalTok{, }\DataTypeTok{box.padding =}\NormalTok{ .}\DecValTok{75}\NormalTok{, }\DataTypeTok{max.iter =} \DecValTok{100000}\NormalTok{)}
\end{Highlighting}
\end{Shaded}

\section{Ecosystem Functioning}\label{ecosystem-functioning}

\subsection{Fig 1: Microbial metabolism along reservoir
gradient}\label{fig-1-microbial-metabolism-along-reservoir-gradient}

Read in data

\begin{Shaded}
\begin{Highlighting}[]
\NormalTok{metab <-}\StringTok{ }\KeywordTok{read.table}\NormalTok{(}\StringTok{"data/res.grad.metab.txt"}\NormalTok{, }\DataTypeTok{sep=}\StringTok{"}\CharTok{\textbackslash{}t}\StringTok{"}\NormalTok{, }\DataTypeTok{header=}\OtherTok{TRUE}\NormalTok{)}
\KeywordTok{colnames}\NormalTok{(metab) <-}\StringTok{ }\KeywordTok{c}\NormalTok{(}\StringTok{"dist"}\NormalTok{, }\StringTok{"BP"}\NormalTok{, }\StringTok{"BR"}\NormalTok{)}
\NormalTok{BGE <-}\StringTok{ }\KeywordTok{round}\NormalTok{((metab}\OperatorTok{$}\NormalTok{BP}\OperatorTok{/}\NormalTok{(metab}\OperatorTok{$}\NormalTok{BP }\OperatorTok{+}\StringTok{ }\NormalTok{metab}\OperatorTok{$}\NormalTok{BR)),}\DecValTok{3}\NormalTok{)}
\NormalTok{metab <-}\StringTok{ }\KeywordTok{cbind}\NormalTok{(metab, BGE)}


\CommentTok{# Quadratic regression for BP}
\NormalTok{dist <-}\StringTok{ }\NormalTok{metab}\OperatorTok{$}\NormalTok{dist}
\NormalTok{dist2 <-}\StringTok{ }\NormalTok{metab}\OperatorTok{$}\NormalTok{dist}\OperatorTok{^}\DecValTok{2}
\NormalTok{BP.fit <-}\StringTok{ }\KeywordTok{lm}\NormalTok{(metab}\OperatorTok{$}\NormalTok{BP }\OperatorTok{~}\StringTok{ }\NormalTok{dist }\OperatorTok{+}\StringTok{ }\NormalTok{dist2)}
\NormalTok{BP.R2 <-}\StringTok{ }\KeywordTok{round}\NormalTok{(}\KeywordTok{summary}\NormalTok{(BP.fit)}\OperatorTok{$}\NormalTok{r.squared, }\DecValTok{2}\NormalTok{)}

\CommentTok{# Simple linear regression for BR}
\NormalTok{BR.fit <-}\StringTok{ }\KeywordTok{lm}\NormalTok{(metab}\OperatorTok{$}\NormalTok{BR }\OperatorTok{~}\StringTok{ }\NormalTok{metab}\OperatorTok{$}\NormalTok{dist)}
\NormalTok{BR.R2 <-}\StringTok{ }\KeywordTok{round}\NormalTok{(}\KeywordTok{summary}\NormalTok{(BR.fit)}\OperatorTok{$}\NormalTok{r.squared, }\DecValTok{2}\NormalTok{)}
\NormalTok{BR.int <-}\StringTok{ }\NormalTok{BR.fit}\OperatorTok{$}\NormalTok{coefficients[}\DecValTok{1}\NormalTok{]}
\NormalTok{BR.slp <-}\StringTok{ }\NormalTok{BR.fit}\OperatorTok{$}\NormalTok{coefficients[}\DecValTok{2}\NormalTok{]}

\CommentTok{# Simple linear regression for BGE}
\NormalTok{BGE.fit <-}\StringTok{ }\KeywordTok{lm}\NormalTok{(metab}\OperatorTok{$}\NormalTok{BGE }\OperatorTok{~}\StringTok{ }\NormalTok{metab}\OperatorTok{$}\NormalTok{dist)}
\NormalTok{BGE.R2 <-}\StringTok{ }\KeywordTok{round}\NormalTok{(}\KeywordTok{summary}\NormalTok{(BGE.fit)}\OperatorTok{$}\NormalTok{r.squared, }\DecValTok{2}\NormalTok{)}
\NormalTok{BGE.int <-}\StringTok{ }\NormalTok{BGE.fit}\OperatorTok{$}\NormalTok{coefficients[}\DecValTok{1}\NormalTok{]}
\NormalTok{BGE.slp <-}\StringTok{ }\NormalTok{BGE.fit}\OperatorTok{$}\NormalTok{coefficients[}\DecValTok{2}\NormalTok{]}

\NormalTok{BP.R2}
\NormalTok{BR.R2}
\NormalTok{BGE.R2}

\NormalTok{BP.plot <-}\StringTok{ }\KeywordTok{ggplot}\NormalTok{(metab, }\KeywordTok{aes}\NormalTok{(}\DataTypeTok{x =}\NormalTok{ dist, }\DataTypeTok{y =}\NormalTok{ BP)) }\OperatorTok{+}\StringTok{ }
\StringTok{  }\KeywordTok{geom_point}\NormalTok{() }\OperatorTok{+}\StringTok{ }
\StringTok{  }\KeywordTok{geom_smooth}\NormalTok{(}\DataTypeTok{method =} \StringTok{"lm"}\NormalTok{, }\DataTypeTok{formula =}\NormalTok{ y }\OperatorTok{~}\StringTok{ }\NormalTok{x }\OperatorTok{+}\StringTok{ }\KeywordTok{I}\NormalTok{(x}\OperatorTok{^}\DecValTok{2}\NormalTok{), }\DataTypeTok{color =} \StringTok{"black"}\NormalTok{) }\OperatorTok{+}
\StringTok{  }\KeywordTok{annotate}\NormalTok{(}\DataTypeTok{geom =} \StringTok{"text"}\NormalTok{, }\DataTypeTok{x =} \DecValTok{50}\NormalTok{, }\DataTypeTok{y =} \FloatTok{1.5}\NormalTok{, }\DataTypeTok{size =} \DecValTok{5}\NormalTok{, }
           \DataTypeTok{label =} \KeywordTok{paste0}\NormalTok{(}\StringTok{"R^2== "}\NormalTok{,BP.R2), }\DataTypeTok{parse =}\NormalTok{ T) }\OperatorTok{+}
\StringTok{  }\KeywordTok{labs}\NormalTok{(}\DataTypeTok{y =} \KeywordTok{expression}\NormalTok{(}\KeywordTok{paste}\NormalTok{(}\StringTok{'BP ('}\NormalTok{, mu ,}\StringTok{'M C h'}\OperatorTok{^-}\DecValTok{1}\OperatorTok{*}\StringTok{ ')'}\NormalTok{)), }
       \DataTypeTok{x =} \StringTok{"Reservoir Transect (m)"}\NormalTok{) }\OperatorTok{+}
\StringTok{  }\KeywordTok{scale_x_reverse}\NormalTok{(}\DataTypeTok{limits =} \KeywordTok{c}\NormalTok{(}\DecValTok{400}\NormalTok{,}\DecValTok{0}\NormalTok{))}
\NormalTok{BR.plot <-}\StringTok{ }\KeywordTok{ggplot}\NormalTok{(metab, }\KeywordTok{aes}\NormalTok{(}\DataTypeTok{x =}\NormalTok{ dist, }\DataTypeTok{y =}\NormalTok{ BR)) }\OperatorTok{+}\StringTok{ }
\StringTok{  }\KeywordTok{geom_point}\NormalTok{() }\OperatorTok{+}\StringTok{ }
\StringTok{  }\KeywordTok{geom_smooth}\NormalTok{(}\DataTypeTok{method =} \StringTok{"lm"}\NormalTok{, }\DataTypeTok{formula =}\NormalTok{ y }\OperatorTok{~}\StringTok{ }\NormalTok{x, }\DataTypeTok{color =} \StringTok{"black"}\NormalTok{) }\OperatorTok{+}\StringTok{ }
\StringTok{  }\KeywordTok{annotate}\NormalTok{(}\StringTok{"text"}\NormalTok{, }\DataTypeTok{x =} \DecValTok{50}\NormalTok{, }\DataTypeTok{y =} \FloatTok{1.5}\NormalTok{, }\DataTypeTok{size =} \DecValTok{5}\NormalTok{, }
           \DataTypeTok{label =} \KeywordTok{paste0}\NormalTok{(}\StringTok{"R^2== "}\NormalTok{,BR.R2), }\DataTypeTok{parse =}\NormalTok{ T ) }\OperatorTok{+}
\StringTok{  }\KeywordTok{labs}\NormalTok{(}\DataTypeTok{y =} \KeywordTok{expression}\NormalTok{(}\KeywordTok{paste}\NormalTok{(}\StringTok{'BR ('}\NormalTok{, mu ,}\StringTok{'M C h'}\OperatorTok{^-}\DecValTok{1}\OperatorTok{*}\StringTok{ ')'}\NormalTok{)), }
       \DataTypeTok{x =} \StringTok{"Reservoir Transect (m)"}\NormalTok{) }\OperatorTok{+}
\StringTok{  }\KeywordTok{scale_x_reverse}\NormalTok{(}\DataTypeTok{limits =} \KeywordTok{c}\NormalTok{(}\DecValTok{400}\NormalTok{,}\DecValTok{0}\NormalTok{))}
\NormalTok{BGE.plot <-}\StringTok{ }\KeywordTok{ggplot}\NormalTok{(metab, }\KeywordTok{aes}\NormalTok{(}\DataTypeTok{x =}\NormalTok{ dist, }\DataTypeTok{y =}\NormalTok{ BGE)) }\OperatorTok{+}\StringTok{ }
\StringTok{  }\KeywordTok{geom_point}\NormalTok{() }\OperatorTok{+}\StringTok{ }
\StringTok{  }\KeywordTok{geom_smooth}\NormalTok{(}\DataTypeTok{method =} \StringTok{"lm"}\NormalTok{, }\DataTypeTok{formula =}\NormalTok{ y }\OperatorTok{~}\StringTok{ }\NormalTok{x }\OperatorTok{+}\StringTok{ }\KeywordTok{I}\NormalTok{(x}\OperatorTok{^}\DecValTok{2}\NormalTok{), }\DataTypeTok{color =} \StringTok{"black"}\NormalTok{) }\OperatorTok{+}
\StringTok{  }\KeywordTok{annotate}\NormalTok{(}\StringTok{"text"}\NormalTok{, }\DataTypeTok{x =} \DecValTok{50}\NormalTok{, }\DataTypeTok{y =}\NormalTok{ .}\DecValTok{5}\NormalTok{, }\DataTypeTok{size =} \DecValTok{5}\NormalTok{, }
           \DataTypeTok{label =} \KeywordTok{paste0}\NormalTok{(}\StringTok{"R^2== "}\NormalTok{,BGE.R2), }\DataTypeTok{parse =}\NormalTok{ T ) }\OperatorTok{+}
\StringTok{  }\KeywordTok{labs}\NormalTok{(}\DataTypeTok{y =} \StringTok{"BGE"}\NormalTok{, }
       \DataTypeTok{x =} \StringTok{"Reservoir Transect (m)"}\NormalTok{) }\OperatorTok{+}
\StringTok{  }\KeywordTok{scale_x_reverse}\NormalTok{(}\DataTypeTok{limits =} \KeywordTok{c}\NormalTok{(}\DecValTok{400}\NormalTok{,}\DecValTok{0}\NormalTok{))}
\end{Highlighting}
\end{Shaded}

\begin{Shaded}
\begin{Highlighting}[]
\KeywordTok{plot_grid}\NormalTok{(BP.plot }\OperatorTok{+}\StringTok{ }\KeywordTok{theme}\NormalTok{(}\DataTypeTok{axis.title.x =} \KeywordTok{element_blank}\NormalTok{(), }\DataTypeTok{axis.text.x =} \KeywordTok{element_blank}\NormalTok{(), }
                          \DataTypeTok{plot.margin =} \KeywordTok{unit}\NormalTok{(}\KeywordTok{c}\NormalTok{(}\DecValTok{1}\NormalTok{, }\DecValTok{1}\NormalTok{, }\OperatorTok{-}\DecValTok{1}\NormalTok{, }\DecValTok{0}\NormalTok{), }\StringTok{"cm"}\NormalTok{)), }
\NormalTok{          BR.plot }\OperatorTok{+}\StringTok{ }\KeywordTok{theme}\NormalTok{(}\DataTypeTok{axis.title.x =} \KeywordTok{element_blank}\NormalTok{(), }\DataTypeTok{axis.text.x =} \KeywordTok{element_blank}\NormalTok{(),}
                          \DataTypeTok{plot.margin =} \KeywordTok{unit}\NormalTok{(}\KeywordTok{c}\NormalTok{(}\OperatorTok{-}\DecValTok{1}\NormalTok{, }\DecValTok{1}\NormalTok{, }\OperatorTok{-}\DecValTok{1}\NormalTok{, }\DecValTok{0}\NormalTok{), }\StringTok{"cm"}\NormalTok{)), }
\NormalTok{          BGE.plot }\OperatorTok{+}\StringTok{ }\KeywordTok{theme}\NormalTok{(}\DataTypeTok{plot.margin =} \KeywordTok{unit}\NormalTok{(}\KeywordTok{c}\NormalTok{(}\OperatorTok{-}\DecValTok{1}\NormalTok{, }\DecValTok{1}\NormalTok{, }\DecValTok{0}\NormalTok{, }\DecValTok{0}\NormalTok{), }\StringTok{"cm"}\NormalTok{)), }
          \DataTypeTok{align =} \StringTok{"hv"}\NormalTok{, }\DataTypeTok{ncol =} \DecValTok{1}\NormalTok{, }\DataTypeTok{labels =} \StringTok{"AUTO"}\NormalTok{)}
\end{Highlighting}
\end{Shaded}

\begin{center}\includegraphics{ReservoirGradient_files/figure-latex/metab_plot-1} \end{center}

\subsection{Relation of ecosystem functions and community
structure}\label{relation-of-ecosystem-functions-and-community-structure}

\begin{Shaded}
\begin{Highlighting}[]
\CommentTok{# detrend the spatial signal}
\NormalTok{bp.resid <-}\StringTok{ }\KeywordTok{resid}\NormalTok{(}\KeywordTok{lm}\NormalTok{(BP }\OperatorTok{~}\StringTok{ }\NormalTok{dist }\OperatorTok{+}\StringTok{ }\KeywordTok{I}\NormalTok{(dist)}\OperatorTok{^}\DecValTok{2}\NormalTok{, }\DataTypeTok{data =}\NormalTok{ metab))}
\NormalTok{br.resid <-}\StringTok{ }\KeywordTok{resid}\NormalTok{(}\KeywordTok{lm}\NormalTok{(BR }\OperatorTok{~}\StringTok{ }\NormalTok{dist, }\DataTypeTok{data =}\NormalTok{ metab))}


\NormalTok{metab.resids <-}\StringTok{ }\NormalTok{metab}
\NormalTok{metab.resids}\OperatorTok{$}\NormalTok{BR_resid <-}\StringTok{ }\NormalTok{br.resid }\OperatorTok{+}\StringTok{ }\KeywordTok{mean}\NormalTok{(metab}\OperatorTok{$}\NormalTok{BR)}
\NormalTok{metab.resids}\OperatorTok{$}\NormalTok{BP_resid <-}\StringTok{ }\NormalTok{bp.resid }\OperatorTok{+}\StringTok{ }\KeywordTok{mean}\NormalTok{(metab}\OperatorTok{$}\NormalTok{BP)}

\NormalTok{transient.metabolism <-}\StringTok{ }\KeywordTok{data.frame}\NormalTok{(}\DataTypeTok{transients =}\NormalTok{ terr.REL, }\DataTypeTok{dist =}\NormalTok{ design.dna}\OperatorTok{$}\NormalTok{distance) }\OperatorTok\StringTok{ }
\StringTok{  }\KeywordTok{left_join}\NormalTok{(metab.resids) }


\NormalTok{bp.mod.quad <-}\StringTok{ }\KeywordTok{lm}\NormalTok{(BP_resid }\OperatorTok{~}\StringTok{ }\NormalTok{transients }\OperatorTok{+}\StringTok{ }\KeywordTok{I}\NormalTok{(transients}\OperatorTok{^}\DecValTok{2}\NormalTok{), }\DataTypeTok{data =}\NormalTok{ transient.metabolism)}
\NormalTok{bp.mod.lin <-}\StringTok{ }\KeywordTok{lm}\NormalTok{(BP_resid }\OperatorTok{~}\StringTok{ }\NormalTok{transients, }\DataTypeTok{data =}\NormalTok{ transient.metabolism)}
\NormalTok{bp.mod.int <-}\StringTok{ }\KeywordTok{lm}\NormalTok{(BP_resid }\OperatorTok{~}\StringTok{ }\DecValTok{1}\NormalTok{, }\DataTypeTok{data =}\NormalTok{ transient.metabolism)}
\KeywordTok{anova}\NormalTok{(bp.mod.int, bp.mod.lin, bp.mod.quad)}
\KeywordTok{AIC}\NormalTok{(bp.mod.quad, bp.mod.lin, bp.mod.int)}

\NormalTok{br.mod.quad <-}\StringTok{ }\KeywordTok{lm}\NormalTok{(BR_resid }\OperatorTok{~}\StringTok{ }\NormalTok{transients }\OperatorTok{+}\StringTok{ }\KeywordTok{I}\NormalTok{(transients}\OperatorTok{^}\DecValTok{2}\NormalTok{), }\DataTypeTok{data =}\NormalTok{ transient.metabolism)}
\NormalTok{br.mod.lin <-}\StringTok{ }\KeywordTok{lm}\NormalTok{(BR_resid }\OperatorTok{~}\StringTok{ }\NormalTok{transients, }\DataTypeTok{data =}\NormalTok{ transient.metabolism)}
\NormalTok{br.mod.int <-}\StringTok{ }\KeywordTok{lm}\NormalTok{(BR_resid }\OperatorTok{~}\StringTok{ }\DecValTok{1}\NormalTok{, }\DataTypeTok{data =}\NormalTok{ transient.metabolism)}
\KeywordTok{anova}\NormalTok{(br.mod.int, br.mod.lin, br.mod.quad)}
\KeywordTok{AIC}\NormalTok{(br.mod.int, br.mod.lin, br.mod.quad)}

\NormalTok{bge.mod.quad <-}\StringTok{ }\KeywordTok{lm}\NormalTok{(BGE }\OperatorTok{~}\StringTok{ }\NormalTok{transients }\OperatorTok{+}\StringTok{ }\KeywordTok{I}\NormalTok{(transients}\OperatorTok{^}\DecValTok{2}\NormalTok{), }\DataTypeTok{data =}\NormalTok{ transient.metabolism)}
\NormalTok{bge.mod.lin <-}\StringTok{ }\KeywordTok{lm}\NormalTok{(BGE }\OperatorTok{~}\StringTok{ }\NormalTok{transients, }\DataTypeTok{data =}\NormalTok{ transient.metabolism)}
\NormalTok{bge.mod.int <-}\StringTok{ }\KeywordTok{lm}\NormalTok{(BGE }\OperatorTok{~}\StringTok{ }\DecValTok{1}\NormalTok{, }\DataTypeTok{data =}\NormalTok{ transient.metabolism)}
\KeywordTok{anova}\NormalTok{(bge.mod.int, bge.mod.lin, bge.mod.quad)}
\KeywordTok{AIC}\NormalTok{(bge.mod.int, bge.mod.lin, bge.mod.quad)}

\KeywordTok{round}\NormalTok{(}\KeywordTok{summary}\NormalTok{(br.mod.quad)}\OperatorTok{$}\NormalTok{r.squared, }\DecValTok{2}\NormalTok{)}
\KeywordTok{round}\NormalTok{(}\KeywordTok{summary}\NormalTok{(bp.mod.quad)}\OperatorTok{$}\NormalTok{r.squared, }\DecValTok{2}\NormalTok{)}


\NormalTok{total_core <-}\StringTok{ }\KeywordTok{rowSums}\NormalTok{(OTUsREL[design}\OperatorTok{$}\NormalTok{molecule }\OperatorTok{==}\StringTok{ "DNA"} \OperatorTok{&}\StringTok{ }\NormalTok{design}\OperatorTok{$}\NormalTok{type }\OperatorTok{==}\StringTok{ "water"}\NormalTok{,}
                               \KeywordTok{subset}\NormalTok{(}\KeywordTok{rbind.data.frame}\NormalTok{(high.activity.water.core, }
\NormalTok{                                                       high.activity.soil.core), RNA.max }\OperatorTok{>}\StringTok{ }\NormalTok{.}\DecValTok{01}\NormalTok{)}\OperatorTok{$}\NormalTok{OTU])}

\KeywordTok{summary}\NormalTok{(}\KeywordTok{lm}\NormalTok{(BP }\OperatorTok{~}\StringTok{ }\NormalTok{transients }\OperatorTok{*}\StringTok{ }\NormalTok{dist, transient.metabolism))}
\KeywordTok{summary}\NormalTok{(}\KeywordTok{lm}\NormalTok{(BR }\OperatorTok{~}\StringTok{ }\NormalTok{transients }\OperatorTok{*}\StringTok{ }\NormalTok{dist, transient.metabolism))}


\KeywordTok{data.frame}\NormalTok{(}
  \DataTypeTok{soil_core =} \KeywordTok{rowSums}\NormalTok{(OTUsREL[design}\OperatorTok{$}\NormalTok{molecule }\OperatorTok{==}\StringTok{ "DNA"} \OperatorTok{&}\StringTok{ }\NormalTok{design}\OperatorTok{$}\NormalTok{type }\OperatorTok{==}\StringTok{ "water"}\NormalTok{,}
           \KeywordTok{subset}\NormalTok{(soil.vs.lake.abunds, RNA.max }\OperatorTok{>}\StringTok{ }\NormalTok{.}\DecValTok{01}\NormalTok{)}\OperatorTok{$}\NormalTok{OTU]), }
  \DataTypeTok{dist =}\NormalTok{ design.dna}\OperatorTok{$}\NormalTok{distance) }\OperatorTok\StringTok{ }
\StringTok{  }\KeywordTok{left_join}\NormalTok{(metab.resids) }\OperatorTok\StringTok{ }\KeywordTok{select}\NormalTok{(}\OperatorTok{-}\NormalTok{BGE, }\OperatorTok{-}\NormalTok{BP, }\OperatorTok{-}\NormalTok{BR) }\OperatorTok\StringTok{ }\KeywordTok{gather}\NormalTok{(metab, value, }\OperatorTok{-}\NormalTok{soil_core, }\OperatorTok{-}\NormalTok{dist) }\OperatorTok\StringTok{ }
\StringTok{  }\KeywordTok{ggplot}\NormalTok{(}\KeywordTok{aes}\NormalTok{(}\DataTypeTok{x =}\NormalTok{ soil_core, }\DataTypeTok{y =}\NormalTok{ value, }\DataTypeTok{color =}\NormalTok{ metab, }\DataTypeTok{fill =}\NormalTok{ metab)) }\OperatorTok{+}
\StringTok{  }\KeywordTok{geom_point}\NormalTok{(}\DataTypeTok{size =} \DecValTok{2}\NormalTok{) }\OperatorTok{+}\StringTok{ }
\StringTok{  }\KeywordTok{geom_smooth}\NormalTok{(}\DataTypeTok{alpha =}\NormalTok{ .}\DecValTok{25}\NormalTok{, }\DataTypeTok{method =} \StringTok{'lm'}\NormalTok{, }\DataTypeTok{formula =}\NormalTok{ y }\OperatorTok{~}\StringTok{ }\NormalTok{x }\OperatorTok{+}\StringTok{ }\KeywordTok{I}\NormalTok{(x}\OperatorTok{^}\DecValTok{2}\NormalTok{)) }\OperatorTok{+}
\StringTok{  }\KeywordTok{labs}\NormalTok{(}\DataTypeTok{x =} \StringTok{"Relative Abundance of Soil-derived Core"}\NormalTok{,}
       \DataTypeTok{y =} \KeywordTok{expression}\NormalTok{(}\KeywordTok{paste}\NormalTok{(}\StringTok{'Metabolism ('}\NormalTok{, mu ,}\StringTok{'M C h'}\OperatorTok{^-}\DecValTok{1}\OperatorTok{*}\StringTok{ ')'}\NormalTok{))) }\OperatorTok{+}
\StringTok{  }\KeywordTok{scale_color_viridis}\NormalTok{(}\StringTok{"Ecosystem Function"}\NormalTok{, }\DataTypeTok{discrete =}\NormalTok{ T, }\DataTypeTok{begin =}\NormalTok{ .}\DecValTok{1}\NormalTok{, }\DataTypeTok{end =}\NormalTok{ .}\DecValTok{6}\NormalTok{, }\DataTypeTok{option =} \StringTok{"D"}\NormalTok{) }\OperatorTok{+}
\StringTok{  }\KeywordTok{scale_fill_viridis}\NormalTok{(}\StringTok{"Ecosystem Function"}\NormalTok{, }\DataTypeTok{discrete =}\NormalTok{ T, }\DataTypeTok{begin =}\NormalTok{ .}\DecValTok{1}\NormalTok{, }\DataTypeTok{end =}\NormalTok{ .}\DecValTok{6}\NormalTok{, }\DataTypeTok{option =} \StringTok{"D"}\NormalTok{) }\OperatorTok{+}
\StringTok{  }\KeywordTok{ggsave}\NormalTok{(}\StringTok{"figures/06_soilcore-function.pdf"}\NormalTok{, }\DataTypeTok{bg =} \StringTok{"white"}\NormalTok{, }\DataTypeTok{width =} \DecValTok{7}\NormalTok{, }\DataTypeTok{height =} \DecValTok{6}\NormalTok{)}

\KeywordTok{data.frame}\NormalTok{(}
  \DataTypeTok{water_core =} \KeywordTok{rowSums}\NormalTok{(OTUsREL[design}\OperatorTok{$}\NormalTok{molecule }\OperatorTok{==}\StringTok{ "DNA"} \OperatorTok{&}\StringTok{ }\NormalTok{design}\OperatorTok{$}\NormalTok{type }\OperatorTok{==}\StringTok{ "water"}\NormalTok{,}
                               \KeywordTok{subset}\NormalTok{(high.activity.water.core, RNA.max }\OperatorTok{>}\StringTok{ }\NormalTok{.}\DecValTok{01}\NormalTok{)}\OperatorTok{$}\NormalTok{OTU]), }
  \DataTypeTok{dist =}\NormalTok{ design.dna}\OperatorTok{$}\NormalTok{distance) }\OperatorTok\StringTok{ }
\StringTok{  }\KeywordTok{left_join}\NormalTok{(metab.resids) }\OperatorTok\StringTok{ }\KeywordTok{select}\NormalTok{(}\OperatorTok{-}\NormalTok{BGE,}\OperatorTok{-}\NormalTok{BR,}\OperatorTok{-}\NormalTok{BP) }\OperatorTok\StringTok{ }\KeywordTok{gather}\NormalTok{(metab, value, }\OperatorTok{-}\NormalTok{water_core, }\OperatorTok{-}\NormalTok{dist) }\OperatorTok\StringTok{ }
\StringTok{  }\KeywordTok{ggplot}\NormalTok{(}\KeywordTok{aes}\NormalTok{(}\DataTypeTok{x =}\NormalTok{ water_core, }\DataTypeTok{y =}\NormalTok{ value, }\DataTypeTok{color =}\NormalTok{ metab, }\DataTypeTok{fill =}\NormalTok{ metab)) }\OperatorTok{+}
\StringTok{  }\KeywordTok{geom_point}\NormalTok{(}\DataTypeTok{size =} \DecValTok{2}\NormalTok{) }\OperatorTok{+}\StringTok{ }
\StringTok{  }\KeywordTok{geom_smooth}\NormalTok{(}\DataTypeTok{alpha =}\NormalTok{ .}\DecValTok{25}\NormalTok{, }\DataTypeTok{method =} \StringTok{'lm'}\NormalTok{, }\DataTypeTok{formula =}\NormalTok{ y }\OperatorTok{~}\StringTok{ }\NormalTok{x }\OperatorTok{+}\StringTok{ }\KeywordTok{I}\NormalTok{(x}\OperatorTok{^}\DecValTok{2}\NormalTok{)) }\OperatorTok{+}
\StringTok{  }\KeywordTok{labs}\NormalTok{(}\DataTypeTok{x =} \StringTok{"Relative Abundance of non-soil-derived Core"}\NormalTok{,}
       \DataTypeTok{y =} \KeywordTok{expression}\NormalTok{(}\KeywordTok{paste}\NormalTok{(}\StringTok{'Metabolism ('}\NormalTok{, mu ,}\StringTok{'M C h'}\OperatorTok{^-}\DecValTok{1}\OperatorTok{*}\StringTok{ ')'}\NormalTok{))) }\OperatorTok{+}
\StringTok{  }\KeywordTok{scale_color_viridis}\NormalTok{(}\StringTok{"Ecosystem Function"}\NormalTok{, }\DataTypeTok{discrete =}\NormalTok{ T, }\DataTypeTok{begin =}\NormalTok{ .}\DecValTok{1}\NormalTok{, }\DataTypeTok{end =}\NormalTok{ .}\DecValTok{6}\NormalTok{, }\DataTypeTok{option =} \StringTok{"D"}\NormalTok{) }\OperatorTok{+}
\StringTok{  }\KeywordTok{scale_fill_viridis}\NormalTok{(}\StringTok{"Ecosystem Function"}\NormalTok{, }\DataTypeTok{discrete =}\NormalTok{ T, }\DataTypeTok{begin =}\NormalTok{ .}\DecValTok{1}\NormalTok{, }\DataTypeTok{end =}\NormalTok{ .}\DecValTok{6}\NormalTok{, }\DataTypeTok{option =} \StringTok{"D"}\NormalTok{) }\OperatorTok{+}
\StringTok{  }\KeywordTok{ggsave}\NormalTok{(}\StringTok{"figures/06_nonsoilcore-function.pdf"}\NormalTok{, }\DataTypeTok{bg =} \StringTok{"white"}\NormalTok{, }\DataTypeTok{width =} \DecValTok{7}\NormalTok{, }\DataTypeTok{height =} \DecValTok{6}\NormalTok{)}



\KeywordTok{data.frame}\NormalTok{(}\DataTypeTok{transients =} \KeywordTok{resid}\NormalTok{(}\KeywordTok{lm}\NormalTok{(terr.REL }\OperatorTok{~}\StringTok{ }\NormalTok{design.dna}\OperatorTok{$}\NormalTok{distance)) }\OperatorTok{+}\StringTok{ }\KeywordTok{mean}\NormalTok{(terr.REL), }\DataTypeTok{dist =}\NormalTok{ design.dna}\OperatorTok{$}\NormalTok{distance) }\OperatorTok\StringTok{ }
\StringTok{  }\KeywordTok{left_join}\NormalTok{(metab.resids) }\OperatorTok\StringTok{ }\KeywordTok{select}\NormalTok{(}\OperatorTok{-}\NormalTok{BGE, }\OperatorTok{-}\NormalTok{BP, }\OperatorTok{-}\NormalTok{BR) }\OperatorTok\StringTok{ }\KeywordTok{gather}\NormalTok{(metab, value, }\OperatorTok{-}\NormalTok{transients, }\OperatorTok{-}\NormalTok{dist) }\OperatorTok\StringTok{ }
\StringTok{  }\KeywordTok{ggplot}\NormalTok{(}\KeywordTok{aes}\NormalTok{(}\DataTypeTok{x =}\NormalTok{ transients, }\DataTypeTok{y =}\NormalTok{ value, }\DataTypeTok{color =}\NormalTok{ metab, }\DataTypeTok{fill =}\NormalTok{ metab)) }\OperatorTok{+}
\StringTok{  }\KeywordTok{geom_point}\NormalTok{(}\DataTypeTok{size =} \DecValTok{2}\NormalTok{, }\DataTypeTok{show.legend =}\NormalTok{ F) }\OperatorTok{+}\StringTok{ }
\StringTok{  }\KeywordTok{geom_smooth}\NormalTok{(}\DataTypeTok{alpha =}\NormalTok{ .}\DecValTok{25}\NormalTok{, }\DataTypeTok{method =} \StringTok{'lm'}\NormalTok{, }\DataTypeTok{formula =}\NormalTok{ y }\OperatorTok{~}\StringTok{ }\NormalTok{x, }\DataTypeTok{show.legend =}\NormalTok{ F) }\OperatorTok{+}
\StringTok{  }\KeywordTok{annotation_logticks}\NormalTok{(}\DataTypeTok{sides =} \StringTok{"b"}\NormalTok{) }\OperatorTok{+}
\StringTok{  }\KeywordTok{labs}\NormalTok{(}\DataTypeTok{x =} \StringTok{"Relative Abundance of Transient Taxa"}\NormalTok{,}
       \DataTypeTok{y =} \KeywordTok{expression}\NormalTok{(}\KeywordTok{paste}\NormalTok{(}\StringTok{'Metabolism ('}\NormalTok{, mu ,}\StringTok{'M C h'}\OperatorTok{^-}\DecValTok{1}\OperatorTok{*}\StringTok{ ')'}\NormalTok{))) }\OperatorTok{+}
\StringTok{  }\KeywordTok{scale_color_viridis}\NormalTok{(}\StringTok{"Ecosystem Function"}\NormalTok{, }\DataTypeTok{discrete =}\NormalTok{ T, }\DataTypeTok{begin =}\NormalTok{ .}\DecValTok{1}\NormalTok{, }\DataTypeTok{end =}\NormalTok{ .}\DecValTok{6}\NormalTok{, }\DataTypeTok{option =} \StringTok{"D"}\NormalTok{) }\OperatorTok{+}
\StringTok{  }\KeywordTok{scale_fill_viridis}\NormalTok{(}\StringTok{"Ecosystem Function"}\NormalTok{, }\DataTypeTok{discrete =}\NormalTok{ T, }\DataTypeTok{begin =}\NormalTok{ .}\DecValTok{1}\NormalTok{, }\DataTypeTok{end =}\NormalTok{ .}\DecValTok{6}\NormalTok{, }\DataTypeTok{option =} \StringTok{"D"}\NormalTok{) }\OperatorTok{+}
\StringTok{  }\KeywordTok{scale_y_continuous}\NormalTok{(}\DataTypeTok{limits =} \KeywordTok{c}\NormalTok{(}\DecValTok{0}\NormalTok{,}\DecValTok{3}\NormalTok{)) }\OperatorTok{+}
\StringTok{  }\KeywordTok{theme}\NormalTok{(}\DataTypeTok{plot.margin =} \KeywordTok{unit}\NormalTok{(}\KeywordTok{c}\NormalTok{(}\DecValTok{1}\NormalTok{,}\DecValTok{1}\NormalTok{,}\DecValTok{0}\NormalTok{,}\DecValTok{0}\NormalTok{), }\StringTok{"cm"}\NormalTok{)) }\OperatorTok{+}
\StringTok{  }\KeywordTok{ggsave}\NormalTok{(}\StringTok{"figures/06_transients-function.pdf"}\NormalTok{, }\DataTypeTok{bg =} \StringTok{"white"}\NormalTok{, }\DataTypeTok{width =} \DecValTok{7}\NormalTok{, }\DataTypeTok{height =} \DecValTok{6}\NormalTok{)}


\NormalTok{core.metab <-}\StringTok{ }\KeywordTok{data.frame}\NormalTok{(}
  \DataTypeTok{total_core =} \KeywordTok{rowSums}\NormalTok{(OTUsREL[design}\OperatorTok{$}\NormalTok{molecule }\OperatorTok{==}\StringTok{ "DNA"} \OperatorTok{&}\StringTok{ }\NormalTok{design}\OperatorTok{$}\NormalTok{type }\OperatorTok{==}\StringTok{ "water"}\NormalTok{,}
                               \KeywordTok{subset}\NormalTok{(}\KeywordTok{rbind.data.frame}\NormalTok{(high.activity.water.core, }
\NormalTok{                                                       high.activity.soil.core), RNA.max }\OperatorTok{>}\StringTok{ }\NormalTok{.}\DecValTok{01}\NormalTok{)}\OperatorTok{$}\NormalTok{OTU]), }
  \DataTypeTok{dist =}\NormalTok{ design.dna}\OperatorTok{$}\NormalTok{distance) }\OperatorTok\StringTok{ }
\StringTok{  }\KeywordTok{left_join}\NormalTok{(metab.resids)}

\KeywordTok{summary}\NormalTok{(}\KeywordTok{lm}\NormalTok{(BP }\OperatorTok{~}\StringTok{ }\NormalTok{total_core }\OperatorTok{*}\StringTok{ }\NormalTok{dist, core.metab))}
\KeywordTok{summary}\NormalTok{(}\KeywordTok{lm}\NormalTok{(BR }\OperatorTok{~}\StringTok{ }\NormalTok{total_core }\OperatorTok{+}\StringTok{ }\NormalTok{dist, core.metab))}


\NormalTok{core.metab <-}\StringTok{ }\KeywordTok{data.frame}\NormalTok{(}
  \DataTypeTok{total_core =} \KeywordTok{rowSums}\NormalTok{(OTUsREL[design}\OperatorTok{$}\NormalTok{molecule }\OperatorTok{==}\StringTok{ "DNA"} \OperatorTok{&}\StringTok{ }\NormalTok{design}\OperatorTok{$}\NormalTok{type }\OperatorTok{==}\StringTok{ "water"}\NormalTok{,}
                               \KeywordTok{subset}\NormalTok{(}\KeywordTok{rbind.data.frame}\NormalTok{(high.activity.water.core, }
\NormalTok{                                                       high.activity.soil.core), RNA.max }\OperatorTok{>}\StringTok{ }\NormalTok{.}\DecValTok{01}\NormalTok{)}\OperatorTok{$}\NormalTok{OTU]), }
  \DataTypeTok{dist =}\NormalTok{ design.dna}\OperatorTok{$}\NormalTok{distance) }\OperatorTok\StringTok{ }
\StringTok{  }\KeywordTok{left_join}\NormalTok{(metab.resids)}
\NormalTok{core.metab}\OperatorTok{$}\NormalTok{total_core_resid <-}\StringTok{ }\KeywordTok{resid}\NormalTok{(}\KeywordTok{lm}\NormalTok{(total_core }\OperatorTok{~}\StringTok{ }\NormalTok{dist }\OperatorTok{+}\StringTok{ }\KeywordTok{I}\NormalTok{(dist}\OperatorTok{^}\DecValTok{2}\NormalTok{), core.metab)) }\OperatorTok{+}\StringTok{ }\KeywordTok{mean}\NormalTok{(core.metab}\OperatorTok{$}\NormalTok{total_core)}
\KeywordTok{summary}\NormalTok{(}\KeywordTok{lm}\NormalTok{(BP_resid }\OperatorTok{~}\StringTok{ }\NormalTok{total_core, core.metab))}
\KeywordTok{summary}\NormalTok{(}\KeywordTok{lm}\NormalTok{(BR_resid }\OperatorTok{~}\StringTok{ }\NormalTok{total_core }\OperatorTok{+}\StringTok{ }\KeywordTok{I}\NormalTok{(total_core}\OperatorTok{^}\DecValTok{2}\NormalTok{), core.metab))}


\NormalTok{core.metab }\OperatorTok\StringTok{ }\KeywordTok{select}\NormalTok{(}\OperatorTok{-}\NormalTok{BGE, }\OperatorTok{-}\NormalTok{BP, }\OperatorTok{-}\NormalTok{BR, }\OperatorTok{-}\NormalTok{total_core) }\OperatorTok\StringTok{ }\KeywordTok{gather}\NormalTok{(metab, value, }\OperatorTok{-}\NormalTok{total_core_resid, }\OperatorTok{-}\NormalTok{dist) }\OperatorTok\StringTok{ }
\StringTok{  }\KeywordTok{ggplot}\NormalTok{(}\KeywordTok{aes}\NormalTok{(}\DataTypeTok{x =}\NormalTok{ total_core_resid, }\DataTypeTok{y =}\NormalTok{ value, }\DataTypeTok{color =}\NormalTok{ metab, }\DataTypeTok{fill =}\NormalTok{ metab)) }\OperatorTok{+}
\StringTok{  }\KeywordTok{geom_point}\NormalTok{(}\DataTypeTok{size =} \DecValTok{2}\NormalTok{, }\DataTypeTok{show.legend =}\NormalTok{ F) }\OperatorTok{+}\StringTok{ }
\StringTok{  }\KeywordTok{geom_smooth}\NormalTok{(}\DataTypeTok{alpha =}\NormalTok{ .}\DecValTok{25}\NormalTok{, }\DataTypeTok{method =} \StringTok{'lm'}\NormalTok{, }\DataTypeTok{formula =}\NormalTok{ y }\OperatorTok{~}\StringTok{ }\NormalTok{x, }\DataTypeTok{show.legend =}\NormalTok{ F) }\OperatorTok{+}
\StringTok{  }\KeywordTok{labs}\NormalTok{(}\DataTypeTok{x =} \StringTok{"Relative Abundance of Core Taxa"}\NormalTok{,}
       \DataTypeTok{y =} \KeywordTok{expression}\NormalTok{(}\KeywordTok{paste}\NormalTok{(}\StringTok{'Metabolism ('}\NormalTok{, mu ,}\StringTok{'M C h'}\OperatorTok{^-}\DecValTok{1}\OperatorTok{*}\StringTok{ ')'}\NormalTok{))) }\OperatorTok{+}
\StringTok{  }\KeywordTok{scale_color_viridis}\NormalTok{(}\StringTok{"Ecosystem Function"}\NormalTok{, }\DataTypeTok{discrete =}\NormalTok{ T, }\DataTypeTok{begin =}\NormalTok{ .}\DecValTok{1}\NormalTok{, }\DataTypeTok{end =}\NormalTok{ .}\DecValTok{6}\NormalTok{, }\DataTypeTok{option =} \StringTok{"D"}\NormalTok{) }\OperatorTok{+}
\StringTok{  }\KeywordTok{scale_fill_viridis}\NormalTok{(}\StringTok{"Ecosystem Function"}\NormalTok{, }\DataTypeTok{discrete =}\NormalTok{ T, }\DataTypeTok{begin =}\NormalTok{ .}\DecValTok{1}\NormalTok{, }\DataTypeTok{end =}\NormalTok{ .}\DecValTok{6}\NormalTok{, }\DataTypeTok{option =} \StringTok{"D"}\NormalTok{) }\OperatorTok{+}
\StringTok{  }\KeywordTok{scale_y_continuous}\NormalTok{(}\DataTypeTok{limits =} \KeywordTok{c}\NormalTok{(}\DecValTok{0}\NormalTok{,}\DecValTok{3}\NormalTok{)) }\OperatorTok{+}
\StringTok{  }\KeywordTok{theme}\NormalTok{(}\DataTypeTok{plot.margin =} \KeywordTok{unit}\NormalTok{(}\KeywordTok{c}\NormalTok{(}\DecValTok{1}\NormalTok{,}\DecValTok{1}\NormalTok{,}\DecValTok{0}\NormalTok{,}\DecValTok{0}\NormalTok{), }\StringTok{"cm"}\NormalTok{)) }\OperatorTok{+}
\StringTok{  }\KeywordTok{ggsave}\NormalTok{(}\StringTok{"figures/06_core-function.pdf"}\NormalTok{, }\DataTypeTok{bg =} \StringTok{"white"}\NormalTok{, }\DataTypeTok{width =} \DecValTok{7}\NormalTok{, }\DataTypeTok{height =} \DecValTok{6}\NormalTok{)}
\end{Highlighting}
\end{Shaded}


\end{document}
